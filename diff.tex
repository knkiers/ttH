\documentclass[aps,preprint,tightenlines,floatfix,superscriptaddress,nofootinbib,showpacs]{revtex4-1}
%DIF LATEXDIFF DIFFERENCE FILE
%DIF DEL ../manuscript_older/Paper_ttbarH_Draft_3_2_16.tex   Thu Mar  3 06:02:17 2016
%DIF ADD Paper_ttbarH_Draft.tex                              Thu Mar  3 12:23:21 2016
\pdfoutput=1
\usepackage{graphicx}
%\usepackage{dcolumn}
\usepackage{amsfonts}
\usepackage{hhline}
\usepackage{multirow}
\usepackage{array}
%\usepackage{longtable}
\usepackage{amsmath}
\usepackage{amssymb}
\usepackage{maybemath}
%\usepackage[italic]{hepparticles}
\usepackage{float}
\usepackage[lofdepth,lotdepth]{subfig}
\usepackage[countmax]{subfloat}
\usepackage{cancel}
%\usepackage[nodisplayskipstretch]{setspace}
%\setstretch{1.5}
%\usepackage{caption}
%\usepackage{subcaption}
\setlength{\oddsidemargin}{-1in}
\addtolength{\oddsidemargin}{24mm} \setlength{\textwidth}{170mm}
\setlength{\topmargin}{-0.55in} \setlength{\headheight}{10mm}
\setlength{\headsep}{0mm} \setlength{\textheight}{230mm}
\def\beq{\begin{equation}}
\def\eeq{\end{equation}}
\def\bea{\begin{eqnarray}}
\def\eea{\end{eqnarray}}
\def\nn{\nonumber}
\def\sss{\scriptstyle}
\def\lft{{\scriptstyle L}}
\def\rht{{\scriptstyle R}}
\def\roughly#1{\mathrel{\raise.3ex\hbox
{$#1$\kern-.75em\lower1ex\hbox{$\sim$}}}}
\def\lsim{\roughly<}
\def\gsim{\roughly>}
\def\tbslash{\tbar\hspace{-10pt}\not{}\hspace{4pt}}
\def\tslash{t\hspace{-10pt}\not{}\hspace{4pt}}
\def\lpslash{l^+\hspace{-10pt}\not{}\hspace{4pt}}
\def\lmslash{l^-\hspace{-10pt}\not{}\hspace{4pt}}
\def\nslash{n\hspace{-10pt}\not{}\hspace{4pt}}
\def\ntbslash{n_{\tbar}\hspace{-10pt}\not{}\hspace{4pt}}
%%%%%%%%%%%%%%%%%%%%%%%%%%%%%%%%%%%%%%%%%%%%%%%%%%%%%%%%%%%%%%%%%%%%%%%%%%%%%%%
\newcommand{\note}[1]{\marginpar{{\small\begin{center}{\it #1}\end{center}}}}
%%%%%%%%%%%%%%%%%%%%%%%%%%%%%%%%%%%%%%%%%%%%%%%%%%%%%%%%%%%%%%%%%%%%%%%%%%%%%%%
\def\tbar{\bar{t}}
%\def\tbar{{\overline{t}}}
\def\bbar{\bar{b}}
\def\qbar{\bar{q}}
\def\nubar{{\bar{\nu}}_{\ell}}
\def\tbbc{t \to b \bbar c}
\def\tauprocess{\tau\rightarrow K\pi\pi\nu_{\tau}}
\def\ppprocess{pp\to t\,\left(\rightarrow b {\ell}^+ \nu_{\ell}\right) \tbar\,\left(\rightarrow\bbar {\ell}^-\nubar\right)\,H}
\def\bppprocess{{\boldmath $pp\to t \tbar H\to \left(b {\ell}^+ \nu_{\ell}\right)\left(\bbar {\ell}^-\nubar\right)H$}}
\def\ggprocess{gg\to t \tbar H\to\left(b l^+ \nu_l\right)\left(\bbar l^-\nubar\right)H}
\def\qqprocess{q\qbar\to t \tbar H\to\left(b l^+ \nu_l\right)\left(\bbar l^-\nubar\right)H}
\def\kp{\kappa_t}
\def\kpt{\tilde{\kappa}_t}
\def\MG{\mbox{MadGraph 5}}
\def\tm#1{\texttt{TM-{#1}}}
\def\ex#1{\texttt{EX-{#1}}}
\def\Ahat{\hat A_i^{\sigma}}
\def\madg{MadGraph~5}
\def\TPa{\epsilon(t,\tbar,n_t,n_{\tbar})}
\def\TPb{\epsilon(Q,\tbar,n_t,n_{\tbar})}
\def\TPc{\epsilon(Q,t,n_t,n_{\tbar})}
%%%%%%%%%%%%%%%%%%%%%%%%%%%%%%%%%%%%%%%%%%%%%%%%%%%%%%%%%%%%%%%%%%%%%%%%
%%%%%
%for editing purposes
\usepackage[normalem]{ulem}
\usepackage{color}
\definecolor{BrickRed}{cmyk}{0,0.89,0.94,0.28}
\definecolor{DarkGreen}{cmyk}{1,0,1,0.5}
\definecolor{Blue}{cmyk}{1,1,0,0}
\definecolor{BurntOrange}{cmyk}{0,0.51,1,0}

\def\hldl#1{\textcolor{BrickRed}{\textsf{#1}}}
\def\hlps#1{\textcolor{DarkGreen}{\textsf{#1}}}
\def\hlkk#1{\textcolor{Blue}{\textsf{#1}}}
\def\hlas#1{\textcolor{BurntOrange}{\textsf{#1}}}
\def\soutdl{\bgroup\markoverwith{\textcolor{BrickRed}{\rule[0.5ex]{2pt}{0.4pt}}}\ULon}
\def\soutps{\bgroup\markoverwith{\textcolor{DarkGreen}{\rule[0.5ex]{2pt}{0.4pt}}}\ULon}
\def\soutkk{\bgroup\markoverwith{\textcolor{Blue}{\rule[0.5ex]{2pt}{0.4pt}}}\ULon}
\def\soutas{\bgroup\markoverwith{\textcolor{BurntOrange}{\rule[0.5ex]{2pt}{0.4pt}}}\ULon}

\def\mhldl#1{\textcolor{BrickRed}{\ensuremath{#1}}}
\def\mhlps#1{\textcolor{DarkGreen}{\ensuremath{#1}}}
\def\mhlkk#1{\textcolor{Blue}{\ensuremath{#1}}}
\def\mhlas#1{\textcolor{BurntOrange}{\ensuremath{#1}}}
\def\msoutdl#1{\text{\soutps{\ensuremath{#1}}}}
\def\msoutps#1{\text{\soutps{\ensuremath{#1}}}}
\def\msoutkk#1{\text{\soutps{\ensuremath{#1}}}}
\def\msoutas#1{\text{\soutps{\ensuremath{#1}}}}
%%%%%%%%%%%%%%%%%%%%%%%%%%%%%%%%%%%%%%%%%%%%%%%%%%%%%%%%%%%%%%%%%%%%%%%%
%DIF PREAMBLE EXTENSION ADDED BY LATEXDIFF
%DIF UNDERLINE PREAMBLE %DIF PREAMBLE
\RequirePackage[normalem]{ulem} %DIF PREAMBLE
\RequirePackage{color}\definecolor{RED}{rgb}{1,0,0}\definecolor{BLUE}{rgb}{0,0,1} %DIF PREAMBLE
\providecommand{\DIFadd}[1]{{\protect\color{blue}\uwave{#1}}} %DIF PREAMBLE
\providecommand{\DIFdel}[1]{{\protect\color{red}\sout{#1}}}                      %DIF PREAMBLE
%DIF SAFE PREAMBLE %DIF PREAMBLE
\providecommand{\DIFaddbegin}{} %DIF PREAMBLE
\providecommand{\DIFaddend}{} %DIF PREAMBLE
\providecommand{\DIFdelbegin}{} %DIF PREAMBLE
\providecommand{\DIFdelend}{} %DIF PREAMBLE
%DIF FLOATSAFE PREAMBLE %DIF PREAMBLE
\providecommand{\DIFaddFL}[1]{\DIFadd{#1}} %DIF PREAMBLE
\providecommand{\DIFdelFL}[1]{\DIFdel{#1}} %DIF PREAMBLE
\providecommand{\DIFaddbeginFL}{} %DIF PREAMBLE
\providecommand{\DIFaddendFL}{} %DIF PREAMBLE
\providecommand{\DIFdelbeginFL}{} %DIF PREAMBLE
\providecommand{\DIFdelendFL}{} %DIF PREAMBLE
%DIF END PREAMBLE EXTENSION ADDED BY LATEXDIFF

\begin{document}
\vspace*{2cm}

\title{{\boldmath Pseudoscalar top-Higgs coupling:  Exploration of $\mathrm{CP}$-odd observables to 
resolve the sign ambiguity
}}

\def\tayloru{\affiliation{\it Physics and Engineering Department,
    Taylor University, \\ 236 West Reade Ave., Upland, IN 46989, USA \vspace*{1mm}}}

%\def\tayloru{\affiliation{\it Physics and Engineering Department,
%    Taylor University, \\ 236 West Reade Ave., Upland, IN 46989, USA \vspace*{8mm}}}

\def\laplata{\affiliation{\it IFLP, CONICET -- Dpto. de F\'{\i}sica,
    Universidad Nacional de La Plata, C.C. 67, 1900 La Plata,
    Argentina \vspace*{8mm}}}

\author{Nicolas Mileo}
\email{mileo@fisica.unlp.edu.ar}
\laplata

\author{Ken Kiers}
\email{knkiers@taylor.edu}
\tayloru

\author{Alejandro Szynkman}
\email{szynkman@fisica.unlp.edu.ar}
\laplata

\author{Daniel Crane \vspace*{4mm}}
\email{dkcrane@mtu.edu}
\tayloru

\author{Ethan Gegner}
\email{ethan\_gegner@taylor.edu}
\tayloru

\date{\vspace*{2mm}\today \\\bigskip\bigskip}

\begin{abstract}
\vspace*{4mm} We present a collection of $\mathrm{CP}$-odd observables
for the process $\ppprocess$ that are linearly dependent on the scalar
($\kp$) and pseudoscalar ($\kpt$) top-Higgs coupling and hence
sensitive to the corresponding relative sign. The proposed observables
are based on triple product (TP) structures that we extract
from the expression of the differential cross section in terms of the
spin vectors of the top and antitop quarks. In order to explore other
possibilities, we progressively modify these TPs, first
by combining them, and then by replacing the spin vectors by the
lepton momenta or the $t$ and $\tbar$ momenta by their visible parts.
Assuming an integrated luminosity that is consistent with that
  envisioned for the HL-LHC, we find that the most
promising observable can disentangle the hypotheses $\kp=1,\kpt=\pm 1$
by more than the $\sim 20\sigma$ statistical level.  In
the case of observables that do not need the reconstruction of the $t$
and $\tbar$ momenta, the power of discrimination is up to the $\sim
16\sigma$ statistical level for the same number of events. We also show that the
capability of the most promising observables for separating the
$\mathrm{CP}$-mixed hypotheses preveals even when a number of events
plausible within the short term LHC is considered. 

%We also show that the sensitivity of the most promising observables is not spoiled when a number of events affordable by the short term LHC is considered.

%We also show that even when the number of events is reduced by an order of magnitude the most promising observables are still useful for testing the couplings
\end{abstract}
\maketitle
%%%%%%%%%%%%%%%%%%%%%%%%%%%%%%%%%%%%%%%%%%%%%%%%%
\section{Introduction}
\label{sec1}
After the discovery of a new boson $H$ by the ATLAS \cite{atlasH} and
CMS \cite{cmsH} collaborations, it has become of crucial importance
 to determine its physical
 properties with the highest possible precision.
 The study of the new boson's couplings
 to fermions is of great relevance
 and will allow us to better understand this particle's $\mathrm{CP}$-transformation
   properties, as well as the
extent to which this particle is consistent with the Higgs boson
predicted by the Standard Model (SM) of particle physics.
It is of particular importance to test the coupling of the putative 
Higgs boson to
the top quark.  This coupling governs
the main Higgs boson production mechanism (which proceeds via gluon fusion)
and it contributes to the
important Higgs boson decay mode to two photons.
It is also
involved in the scalar-field naturalness problem -- giving rise to the
leading dependence on the cut-off energy scale in the corrections to
the Higgs mass -- and it may play an important role in the mechanism for electroweak symmetry breaking.\par

Given that the main Higgs boson production
process is dominated by a top quark loop and that
the diphoton and digluon decay channels are also mediated by a top
loop, these processes provide constraints on the scalar and
pseudoscalar $tH$ couplings, $\kp$ and $\kpt$
\cite{constraints1,constraints2, constraints3,constraints4}.
     However,
these constraints assume that there are no other sources contributing
to the corresponding effective couplings; furthermore, in the case of the
diphoton decay channel (which also involves a $W$ boson loop), it is also
assumed that the
coupling of the Higgs boson to the $W$ is standard. In this sense, the
constraints derived from measurements of Higgs boson production and decay rates
 are indirect constraints. Electric dipole moments can also
impose stringent indirect constraints on $\kpt$ by assuming that there
are no new physics (NP) particles contributing to the loops of the
relevant diagrams and in the case of the EDM of the electron that the 
electron-Higgs coupling is that
predicted by the SM \cite{constraints1,edm}. In order to probe
the $tH$ coupling directly, processes with smaller cross sections need
to be taken into account.  \par

In contrast to the $\tau H$ coupling, which can be studied through the
decay $H\rightarrow \tau^+\tau^-$ \cite{tau},
the $tH$ coupling can only be tested directly via production processes, since the
Higgs boson is kinematically forbidden from decaying to a $t\tbar$ pair.
Two types of processes are of particular interest in this regard -- the
production of a Higgs boson in association with a $t\tbar$ pair
and in association with a single top or antitop. The cross section
for associated Higgs production with a single top (antitop)
is smaller than that for production with a $t\tbar$ pair, and involves
the interference between a diagram in which the Higgs is radiated from
the top (antitop) leg and one with the Higgs emitted from the
intermediate virtual $W$ boson. Interestingly, this implies that the contraints
on $\kp$ and $\kpt$ derived from $tH$ and $\tbar H$ production are
dependent on the assumption made regarding the coupling of the Higgs boson to the
$W$ gauge boson, $\kappa_W$. Nevertheless, it is important to note that the
interference between the above mentioned diagrams
can be exploited to determine the
relative sign between $\kp$ and $\kappa_W$ (see for example
Ref.~\cite{tHmaltoni}).  Associated Higgs production with a
$t\tbar$ pair has been studied by several authors, and
various observables sensitive to the couplings $\kp$ and $\kpt$ have
been proposed. Examples of such observables (all of which are
$\mathrm{CP}$-even) are the cross section, 
invariant mass distributions, the transverse Higgs momentum
distribution and the azimuthal angular separation between the $t$ and $\tbar$,
to name a few \cite{Guadagnoli,*Boosted,*Li,*Golden,*Khatibi}. Also, an approach based on weighted
moments and optimal observables has been developed in Ref.~\cite
{Gunion1,*Gunion2,*Gunion3,*Gunion} to discriminate the hypothesis of
a $\mathrm{CP}$-even Higgs from that of a $\mathrm{CP}$-mixed state
within the context of an $e^+ e^-$ as well as a $pp$ collider.
 Now, $\mathrm{CP}$-even observables are not
sensitive to the relative sign between the scalar and pseudoscalar
couplings $\kp$ and $\kpt$. Such observables are quadratically
dependent on these couplings and thus only provide an indirect measure
of $\mathrm{CP}$ violation. In order to be sensitive to the relative
sign between $\kp$ and $\kpt$, $\mathrm{CP}$-odd observables must be
considered. \par

Since the top quark decays before it can hadronize, its spin
information is passed on to the angular distributions of its decay
products in such a way that these particles work as spin analyzers.
As is well known, in the case of semileptonic top decay,
the charged lepton is the most powerful in this regard.
It is also known that the top quark and antiquark spins
are highly correlated in $t\tbar$ production,
a feature that
is manifested in the double angular distributions of the decay
products of the $t$ and $\tbar$ systems \cite{Mahlon1,*Mahlon2,*Mahlon3,*atwood}.
%The spin correlations
%between the $t$ and $\tbar$ are
%dependent in turn on the $t\tbar$ production mechanism, and in
In the case of
$t\tbar H$ associated production, the $t\tbar$ spin correlations
are also sensitive to the manner in which
the top couples to the Higgs boson. 
In fact, observables that exploit
the differences in the $t\tbar$ spin configurations were used in
Ref.~\cite{Biswas} to improve the discrimination of the $t\tbar H$ signal
from the dominant irreducible background $t\tbar b\bbar$, which does not
involve the Higgs boson. 

In this paper, we define a set of observables that are linearly dependent on
$\kp$ and $\kpt$ and are thus sensitive to the relative sign of these
couplings. The proposed observables are based on a particular set of
triple product (TP) structures that we extract naturally from the
expression for the differential cross section for $\ppprocess$,
making use of the fact that the $t$ and $\tbar$ decay products
contain spin information and are sensitive to the
nature of the $tH$ coupling, as noted above.
By using spinor techniques we relate the top
and antitop spin vectors to final state particle
momenta and separate the production process from the decay. This
allows to identify straightforwardly the contributions linearly
sensitive to the couplings.
 Further, the TPs correlations in these
contributions incorporate the $t$ and $\tbar$ spin vectors. Starting
with these TPs, we not only recover the observables given in
\cite{Ellis,Guadagnoli} but also propose additional possibilities that
increase the sensitivity. In order to establish a hierarchy in the
sensitivity of the TPs under analysis we investigate three different
types of observables by using simulated events: asymmetries, mean
values and angular distributions. We note that TP correlations have
been used in \cite{Valencia1,*Valencia2} in the context of top-quark
production and decay and in \cite{Valencia3} in the framework of
anomalous color dipole operators.  \par

The remainder of this paper is organized as follows. In
Sec.~\ref{sec2} we study the theoretical framework for the process
$\ppprocess$ and derive a general expression for the differential
cross section from which a first set of TP correlations is
extracted. In Sec.~\ref{sec3} we probe the sensitivity of these TPs to
the $tH$ coupling by using various $\mathrm{CP}$-odd
observables. Subsequent sections are dedicated to explore another
possibilities of $\mathrm{CP}$-odd observables. In particular,
observables based on TPs that incorporate the Higgs momentum are
discussed in Sec.~\ref{sec4}, whereas observables obtained without
using the $t$ and $\tbar$ momenta are studied in
Sec.~\ref{sec5}. Finally, Sec.~\ref{sec6} is devoted to the discussion
on the experimental feasibility of the most promising observables
encountered here. The main conclusions are summarized in
Sec.~\ref{sec7}.

%%%%%%%%%%%%%%%%%%%%%%%%%%%%%%%%%%%%%%%%%%%%%%%%%
%\newpage
%%%%%%%%%%%%%%%% Quitarlo %%%%%%%%%%%%%%%%%%%%%%%
\setlength{\abovedisplayskip}{10.6pt}
\setlength{\belowdisplayskip}{10.6pt}
%%%%%%%%%%%%%%%%%%%%%%%%%%%%%%%%%%%%%%%%%%%%%%%%%
%\section{Process \MakeLowercase{{\boldmath $pp\to t \bar{t}$}}$H\to$\MakeLowercase{{\boldmath $\left(b l^+ \nu_l \right) \left(\bar{b} l^-\bar{\nu}_l\right)$}}$H$. Theoretical framework}
\section{Theoretical framework for \MakeLowercase{{\boldmath $pp\to t(\to b {\ell}^+ \nu_{\ell})\,\bar{t}(\to\bar{b} {\ell}^-\bar{\nu}_{\ell})$}}$\,H$}
\label{sec2}


At the LHC $t\bar{t}H$ production proceeds via $q\bar{q}$
annihilation and $gg$ fusion processes. The relevant leading-order Feynman
diagrams are displayed in Fig.~\ref{fig1}, where the first two
rows show the $q\bar{q}$ and $gg$ $s$-channel diagrams, and
the last one depicts the $gg$ $t$-channel diagrams.  Three more
$gg$-initiated diagrams are obtained by exchanging the gluon
lines in the third row.
We describe the $tH$ coupling with the effective Lagrangian
%
\beq
\label{eq1}
\mathcal{L}_{t\bar{t}H}=-\frac{m_t}{v}(\kp \tbar t+i\kpt \tbar
\gamma_5 t)H,
\eeq
%
where $v=246~\mathrm{GeV}$ is the SM Higgs vacuum
expectation value, and the coefficients $\kp$ and $\kpt$ parameterize 
the scalar and pseudoscalar interaction, respectively. The
SM case is obtained for $\kp=1$ and $\kpt=0$, while the values $\kp=0$
and $\kpt\neq 0$ parameterize a $\mathrm{CP}$-odd Higgs boson.

Before turning to a discussion of $\mathrm{CP}$-odd observables,
it is useful to consider a few theoretical aspects of
the process $\ppprocess$, in which the top and antitop both decay
semileptonically.
In the following subsections we derive a ``factorized''
expression for the gluon fusion contribution to this process
and then use this expression
to isolate various mathematical quantities
that will be useful as we construct $\mathrm{CP}$-odd observables.\par

%%%%%%%%%%%%%%%%%%%% Figura 1%%%%%%%%%%%%%%%%%%%%%%%%%%%%%
%/home/nico/jaxodraw-2.1-0/
\begin{center}
\begin{figure}[H]
\centering
%\hspace*{-0.4cm}
\subfloat{\includegraphics[scale=0.45]{qqs1_II.pdf}}
\hspace*{0.05\textwidth}
%\label{fig1a}}
\subfloat{\includegraphics[scale=0.45]{qqs2_II.pdf}}
%\label{fig1b}}
\\[0.032\textwidth]
\subfloat{\includegraphics[scale=0.45]{ggs1_II.pdf}}
\hspace*{0.05\textwidth}
%\label{fig1b}}
\subfloat{\includegraphics[scale=0.45]{ggs2_II.pdf}}
%\label{fig1c}}
\\[0.032\textwidth]
\subfloat{\includegraphics[scale=0.45]{ggt1_II.pdf}}
%\label{fig1b}}
\hspace*{0.025\textwidth}
\subfloat{\includegraphics[scale=0.45]{ggt2_II.pdf}}
%\label{fig1c}}
\hspace*{0.025\textwidth}
\subfloat{\includegraphics[scale=0.45]{ggt3_II.pdf}}
%\label{fig1b}}
\vspace*{0.02\textwidth}
\caption{Tree-level Feynman diagrams contributing to $t\tbar H$ production
  at the LHC. Three more diagrams are obtained by exchanging the gluon
  lines in the $t$-channel diagrams.}
\label{fig1}
\end{figure}
\end{center}
%%%%%%%%%%%%%%%%%%%%%%%%%%%%%%%%%%%%%%%%%%%%%%%%%%%%%%%%%%%%
\subsection{Factorized expression for the scattering cross section}
\label{subsec:factorize}

In this subsection we focus on the $gg$-initiated
contributions
to $t\tbar H$ production, since these dominate over the
the quark-antiquark annihilation contributions. As we shall show below, assuming the
narrow width approximation for the top and antitop quarks, the unpolarized
differential cross section for $gg\to t(\to
b{\ell}^+\nu_{\ell})\,\tbar(\to \bbar {\ell}^- \nubar)\,H$ may be
written in the following ``factorized'' form,\footnote{The reader
  is referred to the discussion following Eq.~(\ref{eq17}) for some qualifying
  remarks regarding the ``factorization'' of this expression.}
%
\beq
\label{eq2}
%dd\sigma(gg\to (bl^+\nu_l)(\bbar l^- \nubar) H)=\sum_{\substack{bl^+\nu_l \\ \tiny{\mathrm{spins}}}}\,\sum_{\substack{\bbar l^-\nubar \\ \mathrm{\tiny{spins}}}}\left(\frac{2}{\Gamma_t}\right)^2\,d\sigma(gg\to t(n_t)\tbar (n_{\tbar})H)\,d\Gamma(t(n_t)\to bl^+\nu_l)\,d\Gamma(\tbar (n_{\tbar})\to \bbar l^-\nubar)
d\sigma =\sum_{\substack{b{\ell}^+\nu_l \\ \tiny{\mathrm{spins}}}}\,
   \sum_{\substack{\bbar {\ell}^-\nubar \\ \mathrm{\tiny{spins}}}}\left(\frac{2}{\Gamma_t}\right)^2\,
   d\sigma(gg\to t(n_t)\tbar (n_{\tbar})H)\,
   d\Gamma(t\to b{\ell}^+\nu_{\ell})\,
   d\Gamma(\tbar \to \bbar {\ell}^-\nubar),
\eeq  
%
where $d\sigma(gg\to t(n_t)\tbar (n_{\tbar})H)$ is the differential
cross section for the production of a top and antitop quark,
with spin vectors $n_t$ and $n_{\bar{t}}$, respectively, along with a Higgs
boson.  Also, $d\Gamma(t\to b{\ell}^+\nu_{\ell})$ and
$d\Gamma(\tbar \to \bbar {\ell}^-\nubar)$ are the
partial differential decay widths for an unpolarized top and
anti-top quark.  The four-vectors
$n_t$ and $n_{\tbar}$ are not arbitrary, but are
given by particular combinations of the
momenta of the $t,\tbar, \ell^+$ and $\ell^-$~\cite{Arens},
%
\bea
\label{eq3}
n_t&=&-\frac{p_t}{m_t}+\frac{m_t}{(p_t\cdot p_{{\ell}^+})}p_{{\ell}^+}\\
\label{eq4}
n_{\tbar}&=&\,\frac{p_{\tbar}}{m_t}-\frac{m_t}{(p_{\tbar}\cdot p_{{\ell}^-})}p_{{\ell}^-}.
\eea
%
Expressions similar to Eq.~(\ref{eq2}) have been derived previously for the
production of short-lived particles in $e^-e^+$ colliders
\cite{kawasaki} and for $t\tbar$ production both in
$e^-e^+$ colliders \cite{Arens} and $pp$ colliders
\cite{ale1,*ale2,*ale3,*ale4}. \par

%%%%%%%%%%%%%%%%% FIGURA 2 %%%%%%%%%%%%%%%%%%%%%%
\begin{center}
\begin{figure}[H]
\centering
\includegraphics[scale=0.6]{esquematico_II.pdf}
\vspace*{0.02\textwidth}
\caption{Schematic representation of the process $g_ag_b\to t(\to
  b_i{\ell}^+\nu_{\ell})\tbar(\to \bbar_j {\ell}^- \nubar) H$. The
  indices $i,j$ denote the colour of the quarks while $a,b$ are gluon
  indices.}
\label{fig2}
\end{figure}
\end{center}
%%%%%%%%%%%%%%%%%%%%%%%%%%%%%%%%%%%%%%%%%%%%%%%%%

To derive the above expressions, we begin by considering the
schematic representation for the process $g_ag_b\to t(\to
b_i{\ell}^+\nu_{\ell})\,\tbar(\to \bbar_j {\ell}^- \nubar)\,H$
that is sketched in Fig.~\ref{fig2}.
Here $a$ and $b$ denote the initial-state gluons and
$i$ and $j$ refer to the colours of the top and antitop quarks.
%\par
%
The amplitude for this process may be written in the following
compact form
%
\beq
\label{eq6}
\mathcal{M}^{ab,ij}=\bar{\psi}_t\,\mathcal{A}^{ab,ij}\,\psi_{\tbar}\,,
\eeq
%
where the spinors $\bar{\psi}_t$ and $\psi_{\tbar}$
contain all of the information regarding the decay of the
virtual top and anti-top, respectively,
and where the quantity $A^{ab,ij}$ is given by
%
\beq
\label{eq5}
\mathcal{A}^{ab,ij}\equiv A^{ab,ij}_{\mu\nu}(\epsilon_{\lambda_a})^{\mu}(\epsilon_{\lambda_b})^{\nu}=\sum_{k=1}^8 \mathcal{A}^{ab,ij}_k =\kp \sum_{k=1}^8 \mathcal{S}^{ab,ij}_k + i\kpt \sum_{k=1}^8 \mathcal{P}^{ab,ij}_k.
\eeq 
%
The sum over $k$ in the above expression
corresponds to the eight gluon-initiated
diagrams indicated in Fig.~\ref{fig1}; also,
$\epsilon_{\lambda_a}$ and $\epsilon_{\lambda_b}$ are the
polarization vectors corresponding to $g_a$ and $g_b$, respectively.
In the last equality in Eq.~(\ref{eq5})
we have explicitly separated the amplitude into
two sums, with one sum corresponding to the scalar contributions
and the other to the pseudoscalar ones.
Taking all of the final-state particles to be massless, we can use the
spinor techniques
developed in \cite{Kleiss} to write $\bar{\psi}_t$ and
$\psi_{\tbar}$ as follows\footnote{These spinor techniques can also be used
  for massive final-state particles. Given the
  energy scale involved in the process in question, however, the assumption
  of massless final-state particles is sensible and greatly
  simplifies the derivation of Eq.~(\ref{eq2}).}
%
\bea
\label{eq7}
\bar{\psi}_t &=& -g^2\, \mathbb{P}_t(t)\,\mathbb{P}_W(t-b)\,
   \langle b-|\nu_{\ell}+\rangle \langle {\ell}^+\!+|(\tslash+m_t)\\
\label{eq8}
\psi_{\tbar} &=& g^2\, \mathbb{P}_t(\tbar)\,\mathbb{P}_W(\tbar-\bbar)\,
   \langle\nubar+|\bbar-\rangle (\tbslash-m_t)|{\ell}^-+\rangle,
\eea
%
where $|i+(-)\rangle \equiv (1/2)(1\pm \gamma^5)\,\psi_i$
represents a right-handed (left-handed) chiral spinor for final-state
particle $i$ and $\langle i+(-)|$ represents the corresponding adjoint spinor.
Also, $\mathbb{P}_t(q)=(q^2-m^2_t+im_t\Gamma_t)^{-1}$ and
$\mathbb{P}_W(q)=(q^2-m^2_W+im_W\Gamma_W)^{-1}$, and
we have denoted the momenta of the various particles by the symbols
that refer to the names of those particles~\cite{Mangano}.

Using the expressions defined above for $\bar{\psi}_t$ and
$\psi_{\tbar}$, we can write the amplitude $\mathcal{M}^{ab,ij}$
in a form that is (in a sense) factorized.
As a first step, we insert
Eqs.~(\ref{eq7}) and (\ref{eq8}) into Eq.~(\ref{eq6}), yielding
%
\beq
\label{eq9}
\mathcal{M}^{ab,ij}=\!-g^4\,\mathbb{P}_t(t)\,\mathbb{P}_t(\tbar)\,\mathbb{P}_W(t-b)\,\mathbb{P}_W(\tbar-\bbar)\,\langle b-|\nu_{\ell}+\rangle \langle\nubar+|\bbar-\rangle \sqrt{\strut2(t\cdot {\ell}^+)}\sqrt{\strut2(\tbar\cdot {\ell}^-)}\,[\bar{\phi}_t \mathcal{A}^{ab,ij}\phi_{\tbar}],
%\frac{1}{(t^2-m^2_t+im_t\Gamma_t)}\,\frac{1}{(\tbar^2-m^2_t+im_t\Gamma_t)}\,\frac{1}{((t-b)^2-M^2_W)}\frac{1}{((\tbar-\bbar)^2-M^2_W)}\,\langle b-|\nu+\rangle\,\langle\nubar+|\bbar-\rangle
\eeq
%
where the spinors $\phi_{t}$ and $\phi_{\tbar}$ are defined as
%
\beq
\label{eq10}
\phi_t=\frac{(\tslash +m_t)}{\sqrt{\strut2(t\cdot {\ell}^+)}}|{\ell}^++\rangle
\eeq
%
\beq
\label{eq11}
\phi_{\tbar}=\frac{(\tbslash -m_t)}{\sqrt{\strut2(\tbar\cdot {\ell}^-)}}|{\ell}^-+\rangle \,.
\eeq
%
Note that in writing down the above expressions we have adopted the narrow-width
approximation for the top and antitop quarks and for the $W^\pm$ gauge
bosons.\footnote{Since Eq.~(\ref{eq9}) contains the top quark propagator
  term
  $\mathbb{P}_t(t)$, for example, $|\mathcal{M}^{ab,ij}|^2$ contains
  the factor $((t^2-m^2_t)^2+m^2_t\Gamma^2_t)^{-1}$,
  which is replaced by $(\pi/m_t\Gamma_t)\delta(t^2-m^2_t)$
  in the narrow-width approximation.  Thus, except for the propagator terms $\mathbb{P}_t(t)$ and $\mathbb{P}_t(\tbar)$,
   we take the four-vector $t$ appearing in Eqs.~(\ref{eq9})-(\ref{eq11}) to be
  on shell, satisfying $t^2 = m_t^2$.} 
Working out the projection operators $\phi_t\,\bar{\phi}_t$
and $\phi_{\tbar}\,\bar{\phi}_{\tbar}$, we have
%
\beq
\label{eq12}
\phi_t\,\bar{\phi}_t=\frac{1}{2}(1+\nslash_{\!t}\gamma^5)(\tslash +m_t)
\eeq
%
and
%
\beq
\label{eq13}
\phi_{\tbar}\,\bar{\phi}_{\tbar}=\frac{1}{2}(1+\nslash_{\!\tbar}\gamma^5)(\tbslash -m_t),
\eeq
%
with $n_t$ and $n_{\tbar}$ being the four-vectors defined in
Eqs.~(\ref{eq3}) and (\ref{eq4}).  Thus, $\phi_t$ and $\phi_{\tbar}$
may be regarded as
describing a top quark with spin vector $n_t$ and an antitop quark
with spin vector $n_{\tbar}$, respectively.

As a final step toward factorizing the amplitude $\mathcal{M}^{ab,ij}$,
we note that the amplitude for a top quark with spin vector
$n_t$ to decay into $b{\ell}^+\nu_{\ell}$ is given by
%
\beq
\label{eq14}
\mathcal{M}(t(n_t)\to b{\ell}^+\nu_{\ell})=ig^2\mathbb{P}_W(t-b)\langle b-|\nu_{\ell}+\rangle\sqrt{\strut2(t\cdot {\ell}^+)} \,,
\eeq
%
and likewise,
%
\beq
\label{eq15}
\mathcal{M}(\tbar(n_{\tbar})\to \bar{b}{\ell}^-\bar{\nu}_{\ell})=ig^2\mathbb{P}_W(\tbar-\bar{b})\langle \bar{\nu}_{\ell}+|\bar{b}-\rangle\sqrt{\strut2(\tbar\cdot {\ell}^-)}.
\eeq
%
Furthermore, the term inside the square brackets in
Eq.~(\ref{eq9}) is the amplitude for producing a top quark
with spin vector $n_t$, along with an anti-top with spin vector
$n_{\tbar}$ and a Higgs boson,
%
\beq
\label{eq16}
\mathcal{M}(g_ag_b \to t^i(n_t)\tbar^j(n_{\tbar})H)=\bar{\phi}_t \mathcal{A}^{ab,ij}\phi_{\tbar}.
\eeq
%
Combining Eqs.~(\ref{eq14})-(\ref{eq16}), we can write Eq.~(\ref{eq9})
in a form that appears to be factorized,
%
\beq
\label{eq17}
\mathcal{M}^{ab,ij}=\mathbb{P}_t(t)\mathbb{P}_t(\tbar)\,\mathcal{M}(t(n_t)\to b{\ell}^+\nu_{\ell})\,\mathcal{M}(\tbar(n_{\tbar})\to \bar{b}{\ell}^-\bar{\nu}_{\ell})\,\mathcal{M}(g_ag_b \to t^i(n_t)\tbar^j(n_{\tbar})H) \,.
\eeq
%
It is important
to note that, even though the above expression has the appearance of being
factorized into production and decay parts, this apparent
factorization is a bit misleading.  In particular, the amplitude for
$t\tbar H$ production contains the top and antitop quark
spin four-vectors $n_t$ and $n_{\tbar}$, which
depend on final-state kinematical quantities [see
Eqs.~(\ref{eq3}) and (\ref{eq4})].
With this qualification in mind, we may now use the
amplitude in Eq.~(\ref{eq17})
to determine the corresponding scattering cross section.
After some manipulation of the phase space variables
to take advantage of the presence of the
propagator terms, $\mathbb{P}_t(t)$ and $\mathbb{P}_t(\tbar)$,
we arrive at the expression in Eq.~(\ref{eq2}).\footnote{The reader may note that in the differential widths of $t\to b{\ell}^+\nu_{\ell}$ and $\tbar\to \bar{b}{\ell}^-\bar{\nu}_{\ell}$ appearing in 
Eq.~(\ref{eq2}), the spin states of the top an antitop have been averaged. Interestingly, under the assumption of massless final-state particles, the amplitudes $\mathcal{M}(t(-n_t)\to b{\ell}^+\nu_{\ell})$ and $\mathcal{M}(\tbar(-n_{\tbar})\to \bar{b}{\ell}^-\bar{\nu}_{\ell})$ vanish. } This expression
also has the appearance of being factorized, but
qualifying remarks, similar to those above, apply.

%%%%%%%%%%%%%%%%%%% Quitarlo %%%%%%%%%%%%%%%%%%%%
\setlength{\abovedisplayskip}{10.2pt}
\setlength{\belowdisplayskip}{10.2pt}
%%%%%%%%%%%%%%%%%%%%%%%%%%%%%%%%%%%%%%%%%%%%%%%%%

\subsection{Origin of triple product terms}
\label{subsec:origin}

The expression derived above for the scattering cross
section [see Eq.~(\ref{eq2}), as well as Eq.~(\ref{eq17})]
provides significant insight into
how one might analyze $\ppprocess$ in order to determine
the nature of the top-Higgs coupling.  In particular,
let us focus on the production amplitude,
$\mathcal{M}(g_ag_b \to t^i(n_t)\tbar^j(n_{\tbar})H))$, which forms
part of the overall amplitude in Eq.~(\ref{eq17}).
The absolute value squared of the production amplitude
is used to determine $d\sigma(gg\to t(n_t)\tbar (n_{\tbar})H)$,
which in turn forms part of the expression for the ``factorized'' cross section
in Eq.~(\ref{eq2}).  Summing over colour and gluon indices
we have
%
\beq
\label{eq18}
\sum_{\substack{a,b \\ i,j}}|\mathcal{M}(g_ag_b \to t^i(n_t)\tbar^j(n_{\tbar})H)|^2=\sum_{\substack{a,b \\ i,j}}\left|\sum^{8}_{k=1}C^{ab,ij}_k\,\bar{\phi}_t (\kp\mathcal{S}_k+i\kpt\mathcal{P}_k)\phi_{\tbar}\right|^2,
\eeq
%
where we have separated the colour structure of each diagram by defining
$\mathcal{S}^{\,ab,ij}_k= C^{ab,ij}_k \mathcal{S}_k$ and
$\mathcal{P}^{\,ab,ij}_k= C^{ab,ij}_k \mathcal{P}_k$
[see Eqs.~(\ref{eq5}) and (\ref{eq16})]. Also, the factors
$g^2_s m_t/v$ and $-ig^2_s m_t/v$ arising from the vertices of the $t$-
and $s$-channel diagrams respectively have been included in the
definition of $C^{ab,ij}_k$ for convenience. The terms linear in $\kp$
and $\kpt$ can be written as
%
\beq
\label{eq19}
\mathcal{O}(\kp\kpt)\to \frac{1}{2}\kp\kpt \sum_{k,r}\mathbb{C}_{kr}\mathrm{Im}
\left\lbrace \mathrm{Tr}\left[ (1+\nslash_t \gamma^5)(\tslash+m_t)\mathcal{S}_k(1+\nslash_{\tbar}\gamma^5
 )(\tbslash -m_t)\tilde{\mathcal{P}}_r \right] \right\rbrace ,
\eeq
%
where the factor $\mathbb{C}_{kr}=\sum_{ab,ij}C^{ab,ij}_k
C^{ab,ij*}_r$ is real and where $\tilde{\mathcal{P}}_r = \gamma^0
\mathcal{P}^{\dagger}_r \gamma^0$.
The only terms that yield non-zero contributions
in the above sum are those with an
odd number of $\gamma^5$ matrices; these lead to triple-product
(TP) structures
of the form $\epsilon_{\alpha\beta\gamma\delta}\,p^{\alpha}_ap^{\beta}_bp^{\gamma}_cp^{\delta}_d$,
where $p_a$-$p_d$ represent various four momenta associated with the process.
In contrast,
it can be seen from Eq.~(\ref{eq18}) that the terms proportional to
$\kp^2$ and $\tilde{\kappa}^2_t$ descend from traces containing
an even number of $\gamma^5$
matrices and can be written in terms of scalar products of the
available momenta.\par
%\vspace*{-2mm}

With the above considerations in mind, it is useful to
write a general expression for the differential cross
section $d\sigma(gg\to t(n_t)\tbar (n_{\tbar})H)$
in terms of the momenta $q=(q_1-q_2)/2$,
$Q=(q_1+q_2)/2$, $t$, $\bar{t}$, $n_t$ and $n_{\tbar}$, where
$q_{1,2}$ denote the momenta of the initial-state gluons. Note that with this
choice, $q\cdot Q=0$ and $Q^2=-q^2=M^2_{t\tbar H}/4$, where $M_{t\tbar
  H}$ is the invariant mass of the $t\tbar H$ system.
Fifteen TPs can be constructed from these six
four-vectors,\footnote{We note that these fifteen
  TPs are not linearly independent (see the epsilon relations
  discussed in Ref.~\cite{identities}).} so that
%
\beq
\label{eq20}
d\sigma(gg\to t(n_t)\tbar (n_{\tbar})H)= \kp^2\,f_1(p_i\cdot p_j)+\tilde{\kappa}^2_t\,f_2(p_i\cdot p_j)+\kp\kpt\,\sum^{15}_{l=1}g_l(p_i\cdot p_j)\,\epsilon_l,
\eeq   
%
where
$\epsilon_l=\epsilon_{\alpha\beta\gamma\delta}\,p^{\alpha}_ap^{\beta}_bp^{\gamma}_cp^{\delta}_d$
denotes the $l$th TP (we adopt the convention $\epsilon_{0123}=+1$) and where $p_i$ and $p_j$ refer to any of the
six momenta.  The
functions $f_{1,2}$ and $g_k$ depend only on the possible scalar
products and are therefore even under a parity transformation
($\mathrm{P}$). However, the terms linear in $\kp\kpt$ are
$\mathrm{P}$-odd due to the presence of the $\mathrm{P}$-odd
TPs. Hence, only the functions $f_{1,2}$ will contribute to the total
cross-section, whereas the TP terms will be sensitive to the sign of
the anomalous coupling $\kpt$. Of the fifteen TPs mentioned above,
we will focus on those that contain both of the spin
vectors $n_t$ and $n_{\tbar}$, but do not include $q$.
The decision not to consider $q$-dependent TPs is motivated by the fact
that $q$ cannot be expressed in terms of the momenta of final state
particles (as $Q$ can, by virtue of energy-momentum conservation). The
decision to focus on TPs that contain both $n_t$ and $n_{\tbar}$
is rooted in the fact that the spins of pair-produced top and antitop quarks
are highly correlated at hadron colliders 
(even though the quarks themselves are unpolarized).
Observables that combine the decay products of the
$t$ and $\tbar$ will be sensitive to this spin
correlation~\cite{Bernreuther}.  A similar behaviour is expected in $t\tbar H$
production, where it can be shown that single-spin asymmetries
vanish~\cite{Ellis,Biswas}. Hence, in order to construct observables
sensitive to the structure of the $tH$ coupling, we will restrict our attention
to those
TPs that include information on the decay products of both the top and
anti-top quarks. Only five of the fifteen TPs in Eq.~(\ref{eq20}) do
not involve the four vector $q$ and, among these, only three
include both $n_t$ and $n_{\tbar}\,$.  Thus, we will restrict our attention
to the following TPs
%
\beq
\label{eqa1}
\epsilon_1\equiv\epsilon(t,\tbar,n_t,n_{\tbar}),\,\,
\vspace*{1mm}
\eeq
%
\beq
\label{eqa2}
\epsilon_2\equiv\epsilon(Q,\tbar,n_t,n_{\tbar}),
\vspace*{1mm}
\eeq
%
\beq
\label{eqa3}
\epsilon_3\equiv\epsilon(Q,t,n_t,n_{\tbar}).
\vspace*{1mm}
\eeq
%
\par  

Before turning to a consideration of various CP-odd observables,
we remark that even though all of the above discussion took place
within the context of $gg$-initiated production, similar conclusions
are obtained for $q\qbar$-initiated production. In particular, the
definitions of the spin vectors in Eqs.~(\ref{eq3})-(\ref{eq4}) and
the general form of $d\sigma$ introduced in Eq.~(\ref{eq20}) are valid
in both cases.
%%%%%%%%%%%%%%%%%%%%%%%%%%%%%%%%%%%%%%%%%%%%%%%%%
%\newpage
%%%%%%%%%%%%%%%%%%%%%%%%%%%%%%%%%%%%%%%%%%%%%%%%%
%\setlength{\belowdisplayskip}{10pt} \setlength{\belowdisplayshortskip}{10pt}
%\setlength{\abovedisplayskip}{10pt} \setlength{\abovedisplayshortskip}{10pt}
\bigskip
\section{$\mathrm{\mathbf{CP}}$-odd observables}
\label{sec3}
In this section we present three types of observables based on the TPs discussed
in Sec.~\ref{sec2}, namely, asymmetries, angular
distributions and mean values. These observables are sensitive not only to the
magnitude of the pseudoscalar coupling $\kpt$, but also to its
sign.  In order to test the various observables, we have
used $\mathtt{MadGraph5\_aMC@NLO}$ \cite{Madgraph} to simulate the process
$\ppprocess$ at parton level for different values of the couplings
$\kp$ and $\kpt$.  In all cases we have generated $10^5$ events
and have assumed a center-of-mass energy of
$14\,\mathrm{TeV}$.\footnote{Note that, since we generate
  the same number of events in
  each case, the corresponding integrated luminosities are different, since the
  cross section depends on the value of $\kpt$.}
We have also imposed the
following set of cuts: $p_T$ of leptons $> 10\,\mathrm{GeV}$, $|\eta|$
of leptons $< 2.5$, $|\eta|$ of b jets $< 2.5$ and $\Delta
R_{\ell\ell}>0.4$.  Note that we have used this
somewhat large number of events ($10^5$) in order to determine clearly
the extent to which the proposed observables are sensitive to the NP
coupling.  Section~\ref{sec6} contains an
analysis of the experimental feasibility of the more promising observables.

Before continuing on to our analysis, let us make a few comments
regarding the values that we choose for $\kp$ and $\kpt$.
First of all, we note that if the pseudoscalar
coupling $\kpt$ is the only source of physics beyond the SM,
then indirect contraints (based on the signal strength of $gg\to H \to \gamma\gamma$) 
disfavour $\kp < 0$ but do not
resolve the degeneracy in the sign of $\kpt$ \cite{Guadagnoli}. On
the other hand, if one assumes that the tensor structure of the
Higgs interactions are the
same as those of the SM and if one parameterizes these interactions via
one universal Higgs coupling to vector bosons, $\kappa_V$, and one
universal Higgs coupling to fermions, $\kappa_f$, then the measured signal
strengths provided by the ATLAS and CMS collaborations are compatible with
the values predicted by the SM, (namely, $\kappa_f=1$ and $\kappa_V=1$).
With these facts in mind, we will, for the most part,
set the value of the scalar coupling to its SM
value ($\kp=1$) and will allow the pseudoscalar coupling to take on
various values (including both possible signs). In particular, we
analyze the cases $\kpt=0,\pm 0.25, \pm 0.5, \pm 0.75,\pm 1$.
In addition, we also provide some analysis regarding the pure
$\mathrm{CP}$-odd case ($\kp=0,\kpt=1$).
%\newpage
%%%%%%%%%%%%%%%%%%%%%%%%%%%%%%%%%%%%%%%%%%%%%%%%%
\subsection{Asymmetry}
%\bigskip
\label{sec3.1}

The first type of $\mathrm{CP}$-odd observable that we will consider is an
asymmetry that compares the number of events
for which a given TP is positive to that for which it is negative.
Normalizing to the total number of events, we define
%
\beq
\label{eq21}
\mathcal{A}(\epsilon)=\frac{N(\epsilon > 0)-N(\epsilon < 0)}{N(\epsilon > 0)+N(\epsilon < 0)}.
\eeq 
%
By construction, $\mathcal{A}\in [-1,+1]$.
Based on the general expression given in Eq.~(\ref{eq20}), we expect
the following functional form for the asymmetry,
%
\beq
\label{eq22}
\mathcal{A}(\epsilon)=\frac{A\kp\kpt}{B\kappa^2_t+C\tilde{\kappa}^2_t},
\eeq 
%
which for $\kp=1$ can be parameterized as
%
\beq
\label{eq23}
\mathcal{A}(\epsilon)=\frac{a\kpt}{1+b\tilde{\kappa}^2_t},
\eeq 
%
where the parameter $a\equiv A/B$ determines the sensitivity to the
pseudoscalar coupling, whereas $b\equiv C/B$ quantifies the deviation
from linear behaviour.

Table~\ref{table1} shows numerical results for the
asymmetries associated with three different TPs,
$\epsilon_1$, $\epsilon_2$ and $\epsilon_3$, taking
$\kp=1$ and $\kpt=0,\pm 1$.  The asymmetry $\mathcal{A}$
is shown in each case, along with $\mathcal{A}/\sigma_{\mathcal{A}}$, where
$\sigma_{\mathcal{A}}$ is the corresponding
statistical uncertainty.  As is
evident from the table, the asymmetries in question provide
a clear separation between the SM and the 
$\mathrm{CP}$-mixed cases, with typical deviations being of
order $10\sigma$.  Furthermore, the asymmetries
for the SM case are each statistically consistent with zero,
as one would expect.
The three asymmetries also allow one to
determine the sign of $\kpt$,
with deviations between the $\kpt = \pm 1$ cases typically
being of order $20\sigma$.
{\bf (We couldn't figure out how to clarify this. If you find a way go ahead\ldots.)}
The sensitivity of
the asymmetry is quite similar for the three TPs, as can be seen by
including other values of $\kpt$ and using the expression in
Eq.~(\ref{eq23}) as a fitting function (see Fig.~\ref{fig3}).
Performing such a fit, 
we obtain $(a=-0.057\pm 0.006, b=0.5\pm 0.2),
(a=-0.056\pm 0.006, b=0.5 \pm 0.2)$ and $(a=0.058\pm 0.006, b=0.6 \pm
0.2)$ for $\epsilon_1$, $\epsilon_2$ and $\epsilon_3$,
respectively.

The results shown in Table~\ref{table1}
and Fig.~\ref{fig3} all assume a $pp$ initial state, which is
actually a combination of events coming from $gg$ and $q\qbar$
initial states.
While this combination of initial states is the appropriate scenario
to consider, it
is interesting to consider the relative contributions to the asymmetry
coming from the
$gg$ and $q\qbar$ initial states.  Figure~\ref{fig4} shows
three curves for the ``$\epsilon_1$'' case, one for $gg$-initiated
events, one for $q\qbar$-initiated events, and one for the usual
combination of these events (the ``$pp$'' initial state).
Interestingly, we see from Fig.~\ref{fig4} that the asymmetry for this TP
is enhanced for
$gg$-initiated production, while it is reduced and of opposite sign
for the $q\qbar$-initiated events. The asymmetry for the $pp$ case is
evidently
dominated by the $gg$ contribution, but is somewhat smaller
in magnitude due to the
$q\qbar$ contribution.
\newcolumntype{C}[1]{>{\centering\arraybackslash}p{#1}}
\renewcommand{\arraystretch}{1.4}
\begin{table}[H]
  \caption{Asymmetries for three different scenarios,
    obtained by using $10^5$ simulated events for the
    TPs $\epsilon_1=\TPa, \epsilon_2=\TPb$ and $\epsilon_3=\TPc$.
    The three scenarios
  correspond to the SM ($\kp=1$ and $\kpt=\pm 0$) and
  two $\mathrm{CP}$-mixed cases (defined by
  $\kp=1$ and $\kpt=\pm 1$).}
\label{table1}
\begin{center}
\begin{tabular}{|C{1cm}|C{1cm}||C{2cm}|C{2cm}||C{2cm}|C{2cm}||C{2cm}|C{2cm}|}
%\begin{tabular}{|c|r||r|c||r|c||r|c|}
\hhline{|========|}
%\hhline{|--------|}
$\kappa_t$&$\tilde{\kappa}_t$~~&$\mathcal{A}(\epsilon_1)$~~&$\mathcal{A}(\epsilon_1)/\sigma_{\mathcal{A}}$& $\mathcal{A}(\epsilon_2)$~~&$\mathcal{A}(\epsilon_2)/\sigma_{\mathcal{A}}$&$\mathcal{A}(\epsilon_3)$~~&$\mathcal{A}(\epsilon_3)/\sigma_{\mathcal{A}}$  \\ 
\hhline{|========|} 
%\hhline{|--------|}
$1$ & $-1$~~~ & $0.0315$~ & $10.0$ & $0.0332$~ & $10.5$~ & $-0.0307$~~~ & $-9.7$~~~\\[0.6mm]
\hline
$1$ & $0$ & $-0.0021$~~~ & $-0.7$~~~ & $0.0009$~ & $0.3$~ & $-0.0011$~~~ & $-0.3$~~~\\[0.6mm]
\hline
$1$ & $1$ & $-0.0379$~~~ & $-12.0$~~~ & $-0.0411$~~~& $-13.0$~~~ & $\,0.0378$~ & $12.0$ \\[0.6mm]
\hhline{|========|}
%\hhline{|--------|}
\end{tabular}
\end{center} 
\end{table}
%%%%%%%%%%%%%%%%%%%%%%%%% FIGURA 3 %%%%%%%%%%%%%%%%%%%%%%%%%
\begin{center}
\vspace*{2.5mm}
\begin{figure}[H]
%\centering
\hspace*{-0.45cm}
\subfloat{\includegraphics[scale=0.45]{ATP1_nuevo.pdf}}
\hspace*{0.002\textwidth}
\subfloat{\includegraphics[scale=0.45]{ATP2_nuevo.pdf}} \\
%\hspace*{0.03\textwidth}
\centering
\subfloat{\includegraphics[scale=0.45]{ATP3_nuevo.pdf}}
\caption{Asymmetries for the TPs
  $\epsilon_1=\epsilon(t,\tbar,n_t,n_{\tbar})$ (top-left),
  $\epsilon_2=\epsilon(Q,\tbar,n_t,n_{\tbar})$ (top-right) and
  $\epsilon_3=\epsilon(Q,t,n_t,n_{\tbar})$ (bottom). The points
  represent the values for $\kpt=0,\pm 0.25, \pm 0.5, \pm 0.75,\pm 1$
  and the red solid line is the fitting curve.}
\label{fig3}
\end{figure}
\end{center}
%%%%%%%%%%%%%%%%%%%%%%%%%%%%%%%%%%%%%%%%%%%%
%
%%%%%%%%%%%%%%%%%%%%% FIGURA 4  %%%%%%%%%%%%%%%%%%%%%%%%%%%%
\begin{center}
\vspace*{-4mm}
\begin{figure}[H]
\centering
\includegraphics[scale=0.45]{ATP1juntos_nuevo.pdf}
\caption{Asymmetry for the TP
  $\epsilon_1=\epsilon(t,\tbar,n_t,n_{\tbar})$. The dashed line (red)
  corresponds to $gg$-initiated production, the dot-dashed line (grey)
  to $q\qbar$-initiated production and the solid line (blue) to $pp$
  production.}
\label{fig4}
\end{figure}
\end{center}
%%%%%%%%%%%%%%%%%%%%%%%%%%%%%%%%%%%%%%%%%%%%%%%%%%%%%%%%%%%%%
\vspace*{-6mm}

We have also tested various
linear combinations of the TPs $\epsilon_{1,2,3}$ and have found
that the asymmetry is enhanced for the following combination:
%
\beq
\setlength{\abovedisplayskip}{9.5pt}
\setlength{\belowdisplayskip}{9.5pt}
\label{eq24}
\epsilon_4=\epsilon_3-\epsilon_2=\epsilon(Q,t-\tbar,n_t,n_{\tbar}).
\eeq
%
Note that in the $Q$ rest frame, $\epsilon_4=Q^0
(\vec{t}-\vec{\tbar}\,)\cdot(\vec{n}_t\times \vec{n}_{\tbar})$ and the
sign of this TP is determined by the quantity
$(\vec{t}-\vec{\tbar}\,)\cdot(\vec{n}_t\times \vec{n}_{\tbar})$.
{\bf (KK: try to connect previous sentences to something later on\ldots.)}
The
values obtained for the asymmetry associated with this TP are shown in
Table~\ref{table2}. By comparing the results in Tables \ref{table1}
and \ref{table2}, we see that the capability of this
asymmetry to distinguish between the
two $\mathrm{CP}$-mixed hypotheses is increased by at least $3\sigma$.
\vspace{4mm}
\begin{table}[H]
\caption{Asymmetry for the TP $\epsilon_{4}$ for the SM case and the
  two $\mathrm{CP}$-mixed cases defined by $\kp=1,\kpt=\pm 1$. The
  values are obtained by using $10^5$ simulated events.}
\label{table2}
\begin{center}
\begin{tabular}{|C{1cm}|C{1cm}||C{2cm}|C{2cm}|}
%\begin{tabular}{|c|r||r|c||r|c||r|c|}
\hhline{|====|}
%\hhline{|--------|}
$\kappa_t$&$\tilde{\kappa}_t$~~&$\mathcal{A}(\epsilon_4)$~~&$\mathcal{A}(\epsilon_4)/\sigma_{\mathcal{A}}$ \\ 
\hhline{|====|} 
%\hhline{|--------|}
$1$ & $-1$~~~ & $-0.0371$~~~ & $-12$~~~ \\[0.6mm]
\hline
$1$ & $0$ & $0.0004$~ & $0.1$ \\[0.6mm]
\hline
$1$ & $1$ & $0.0461$~ & $14\,$ \\[0.6mm]
\hhline{|====|}
%\hhline{|--------|}
\end{tabular}
\end{center} 
\end{table}
\par Finally, it is worth noting that the asymmetries
described in this subsection are not useful for
discriminating between the SM hypothesis ($\kp=1,\kpt=0$) and the pure
pseudoscalar hypothesis ($\kp=0,\kpt=1$).  Since
the numerators of the 
asymmetries are all linear in both
$\kp$ and $\kpt$, they are expected to vanish in these cases. However, we
will show in the next subsection
that there exist angular distributions
derived from the TPs that are actually suitable for distinguishing between these
two hyphotheses.
%%%%%%%%%%%%%%%%%%%%%%%%%%%%%%%%%%%%%%%%%%%%%%%%%
\subsection{Angular Distributions}
\label{sec3.2}

Given a certain TP, it is possible to define associated angular
distributions that are sensitive to the pseudoscalar coupling
$\kpt$. In order to clarify this, let us first consider the TP
$\epsilon(t,\tbar,n_t,n_{\tbar})$. This TP can be written as
$\epsilon(t+\tbar,\tbar,n_t,n_{\tbar})$, so that in the reference frame
defined by $\vec{t}+\vec{\tbar} =0$ and $\vec{\tbar}\parallel \hat{z}$
we have
%
\beq
\label{eq25}
\epsilon(t+\tbar,\tbar,n_t,n_{\tbar})=M_{t\tbar}\,|\vec{\tbar}|\,(\vec{n}_t\times \vec{n}_{\tbar})_z=M_{t\tbar}\,|\vec{\tbar}||\vec{n}_t||\vec{n}_{\tbar}|\sin\theta_{n_t}\sin\theta_{n_{\tbar}}\sin \Delta \phi(n_t,n_{\tbar}),
\eeq
%
where $M_{t\tbar}$ is the invariant mass of the $t\tbar$ pair, the
angles $\theta_{n_t}$ and $\theta_{n_{\tbar}}$ denote the polar angles
of $\vec{n}_t$ and $\vec{n}_{\tbar}$, respectively, and 
$\Delta\phi(n_t,n_{\tbar})$ is the angular difference between the
projections of $\vec{n}_t$ and $\vec{n}_{\tbar}$ onto the plane
perpendicular to $\vec{\bar{t}}$. If we define the angle
$\Delta\phi(n_t,n_{\tbar})$  to be within the
range $[-\pi,\pi]$, we see from Eq.~(\ref{eq25}) that its sign will
determine the sign of the TP. Thus, the distribution of the
number of events with respect to the angle
$\Delta\phi(n_t,n_{\tbar})$ is related to the asymmetry of the TP,
%
\beq
\label{eq26}
\mathcal{A}(\epsilon)=1-2\frac{N(\epsilon < 0)}{N_T}\,\,\mbox{ and }\,\,\frac{N(\epsilon < 0)}{N_T}=\int^0_{-\pi}\frac{1}{N_T}\frac{dN}{d\Delta\phi(n_t,n_{\tbar})}\,d\Delta\phi(n_t,n_{\tbar}),
\eeq  
%
where $N_T$ is the total number of events.
{\bf (KK: Have we checked numerically that the above
  expression does indeed hold numerically for the various
  distributions?  It's an exact relation, correct?)}
Moreover, for a certain TP
one can derive different angular distributions by considering
different reference frames, although all of these will satisfy
Eq.~(\ref{eq26}) (note that $\mathcal{A}(\epsilon)$
is Lorentz invariant). Recalling the various TPs considered
in Sec.~\ref{sec2}, we examine the following angular distributions.
 \begin{enumerate} 
\item {\boldmath $\epsilon_1 = \TPa$.}  To probe $\epsilon_1$, we construct
  the distribution
  $d\sigma/d\Delta\phi_1(n_t,n_{\tbar})$ in the rest frame of $t\tbar$,
  taking $\vec{\tbar}$ to define the $z$-axis. The angle
  $\Delta\phi_1(n_t,n_{\tbar})$ is the angular difference between the
  projection of the spin vectors in the plane perpendicular to
  $\vec{\tbar}$.
\item {\boldmath $\epsilon_2 = \TPb$.}  In this case, we
  define the distribution
  $d\sigma/d\Delta\phi_2(n_t,n_{\tbar})$ in the rest frame of $Q$, taking
  $\vec{\tbar}$ to define the $z$-axis. The angle
  $\Delta\phi_2(n_t,n_{\tbar})$ is the angular difference between the
  projection of the spin vectors in the plane perpendicular to
  $\vec{\tbar}$.
\item {\boldmath $\epsilon_3 = \TPc$.}  The distribution
  $d\sigma/d\Delta\phi_3(n_t,n_{\tbar})$ is also defined
  in the rest frame of $Q$, but this time taking
  $\vec{t}$ to be along the $z$-axis. The angle $\Delta\phi_3(n_t,n_{\tbar})$
  is the angular difference between the projection of the spin vectors
  in the plane perpendicular to $\vec{t}$.
 \end{enumerate} 
\par
%
Figure~\ref{fig5} shows the normalized distributions obtained
for the first case listed above.
Four scenarios are considered, corresponding to the SM
($\kp= 1$ and $\kpt=0$), two cases in which the Higgs boson
has mixed $\mathrm{CP}$ couplings ($\kp= 1$ and $\kpt=\pm 1$)
and a case in which the Higgs boson is purely 
$\mathrm{CP}$-odd ($\kp= 0,\kpt=1$). Figure~\ref{fig6} shows
the analogous distributions for $\epsilon_2$. The distributions
corresponding to $\epsilon_3$ are similar to those of $\epsilon_2$,
except that the ``shifts'' are in the opposite directions for
the two mixed-$\mathrm{CP}$ cases.  Given the similarities of the
plots we do not include them here. 
%%%%%%%%%%%%%%%%%%%%%%%%% FIGURA 5 %%%%%%%%%%%%%%%%%%%%%%%%%
%%%% ruta turin facultad: /home/nico/Documentos/ttbarH/Material_final/ todas las figs para abajo
\begin{center}
\vspace*{1.4mm}
\begin{figure}[H]
%\centering
\hspace*{-0.55cm}
%\subfloat{\includegraphics[scale=0.443]{TP1_10_nuevo.pdf}}
\subfloat{\includegraphics[scale=0.45]{NEW_TP1_1_0.pdf}}
\hspace*{-0.006\textwidth}
\subfloat{\includegraphics[scale=0.45]{NEW_TP1_0_1.pdf}} \\
\hspace*{-0.55cm}
%\hspace*{0.03\textwidth}
\subfloat{\includegraphics[scale=0.45]{NEW_TP1_1_1.pdf}}
\hspace*{-0.006\textwidth}
\subfloat{\includegraphics[scale=0.45]{NEW_TP1_1_-1.pdf}}
\vspace*{1cm}
\caption{Angular distributions associated with the TP
  $\epsilon_1 = \epsilon(t,\tbar,n_t,n_{\tbar})$ for various values of
  $\kp$ and $\kpt$. The error bars correspond to the statistical
  uncertainties.}
\label{fig5}
\end{figure}
\end{center}
%%%%%%%%%%%%%%%%%%%%%%%%% FIGURA 6 %%%%%%%%%%%%%%%%%%%%%%%%%
\begin{center}
%\vspace*{-4mm}
\begin{figure}[H]
%\centering
\hspace*{-0.52cm}
\subfloat{\includegraphics[scale=0.45]{NEW_TP2_1_0.pdf}}
\hspace*{-0.006\textwidth}
\subfloat{\includegraphics[scale=0.45]{NEW_TP2_0_1.pdf}} \\
\hspace*{-0.56cm}
%\hspace*{0.03\textwidth}
\subfloat{\includegraphics[scale=0.45]{NEW_TP2_1_1.pdf}}
\hspace*{0.002\textwidth}
\subfloat{\includegraphics[scale=0.45]{NEW_TP2_1_-1.pdf}}
%\subfloat{\includegraphics[scale=0.45]{TP2_1-1_nuevo.pdf}}
\caption{Angular distributions associated with the TP $\epsilon_2 = \TPb$ for
  various values of $\kp$ and $\kpt$. The error bars indicate the statistical
  uncertainties.}
\label{fig6}
\end{figure}
\end{center}
\par As can be seen from Figs.~\ref{fig5} and \ref{fig6}, the
peaks of the 
distributions are shifted to the left or the right of the origin
in the mixed-$\mathrm{CP}$
cases ($\kp =1$ and $\kpt=\pm 1$).
The magnitude of the shift appears to be approximately the
same in both cases, but is in the
opposite direction for $\kp=\kpt=1$ compared to $\kp=-\kpt=1$, thus allowing
one to distinguish
the sign of the pseudoscalar coupling. The observed dependence
on the sign of $\kpt$ in these cases
is consistent with the fact that the numerator of $\mathcal{A}(\epsilon)$
is linear in $\kpt$ [see Eq.~(\ref{eq23})]
and that the quantity $N(\epsilon < 0)/N_T$ is related
to the angular distribution according to Eq.~(\ref{eq26}). The angular
distributions for the SM case ($\kp =1$ and $\kpt=0$) and
the pure pseudoscalar case ($\kp =0$ and $\kpt= 1$) are
visibly different from each other and from the mixed-$\mathrm{CP}$
scenarios. Comparing the SM and purely pseudoscalar cases,
we note that while the angular distributions for the former case
exhibit a minimum at $\Delta\phi_{1,2}(n_t,n_{\tbar})=0$, those
for the latter case exhibit a peak at this location.
Thus, these two scenarios can be distinguished from each other
via these angular distributions.  This is to be contrasted
with the situation for the asymmetries $\mathcal{A}(\epsilon)$, which vanish
in both cases.

In order to quantify the shifts discussed above, we have fitted
the simulated distributions with the following function, which was proposed in
Ref.~\cite{Ellis},
%
\beq
\label{eq27}
\frac{1}{\sigma}\frac{d\sigma}{d\Delta\phi_i(n_t,n_{\tbar})}=a_0 + a_1\cos(\Delta\phi_i(n_t,n_{\tbar})+\delta),\qquad\, i=1,2,3.
\eeq
%
To the extent that the above expression is exact, we note that
Eq.~(\ref{eq26}) {\bf (KK: corrected the equation number; please check\ldots )}
gives $\mathcal{A}(\epsilon_i)=4a_1 \sin\delta$. With
this fitting function, we obtain phase shifts $\delta$ that
are approximately between
$0.9$ and $1$ ($-1$ and $-0.9$) for $\kp=\kpt=1$ ($\kp=-\kpt=1$), both
for $\epsilon_1$ and $\epsilon_2$.\footnote{The results for the
  TP $\epsilon_3$ are relatively similar to
  those for $\epsilon_2$, except that the phase shifts have the opposite
  sign in the $\mathrm{CP}$-mixed cases.  Given this similarity
  we do not include the corresponding results for the $\epsilon_3$ distribution
  here. {\bf (KK: OK to move this to a footnote?)}}  However, the quality of the fits is
not very good, particularly for $\epsilon_1$.  The
$\chi^2/\mathrm{d.o.f}$ for the fits in this case are in the range
$1.69$-$3.86$, while for $\epsilon_2$ they are in the range $0.53$-$1.16$.
{\bf (KK: Did we do 4 fits for each $\epsilon$?  Or only 2?  If only 2, maybe
  saying ``in the range'' is not quite correct, since the two
  data points are simply the endpoints of the ``range''.  If
  the range includes all 4 fits for each, maybe we could state that
  explicitly, since the preceding sentences are only discussing the two
  mixed-CP scenarios.)}
The deviation from the functional
form proposed in Eq.~(\ref{eq27}) appears to be due primarily to the $\Delta
R_{ll}$ cut that we have imposed. In fact, when this cut is turned off, the
above ranges for the $\chi^2/\mathrm{d.o.f}$ become $0.75$-$1.14$ and
$0.44$-$1.07$ for the $\epsilon_1$ and $\epsilon_2$ distributions,
respectively. Tables \ref{table3} and \ref{table4} list the
results of the fits obtained when the $\Delta R_{\ell\ell}$ cut is
relaxed.  Figure~\ref{fig7} shows the corresponding
plots for a couple of the scenarios.
As is evident from Tables \ref{table3} and \ref{table4}, the
parameter $\delta$ is sensitive not only to the modulus of $\kpt$ but
also to its sign, as would be expected from Eq.~(\ref{eq26}).
The phase shift $\delta$ for the $\Delta\phi_1$ distribution
appears to exhibit a slightly higher sensitivity than
that obtained for the $\Delta\phi_2$ distribution, although the corresponding
numerical values obtained for the various scenarios are
compatible to within their statistical uncertainties. It is
important to stress, however, that the fits for the $\Delta\phi_2$ distributions
always yield smaller values for the $\chi^2/\mathrm{d.o.f}$.

\renewcommand{\arraystretch}{1.2}
\begin{table}[H]
\caption{Fit results for the angular distribution $d\sigma/(\sigma
  d\Delta\phi_1(n_t,n_{\tbar}))$ (related to the TP
  $\epsilon_1=\epsilon(t,\tbar,n_t,n_{\tbar})$) with the $\Delta
  R_{\ell\ell}$ cut turned off. Note that the sign of the parameter $a_1$
  changes for $\kp=0,\kp=1$, compared to the
  other cases.  {\bf (KK: Do we restrict $\delta$ to be between
    $\pm \pi/2$?  If not, we could require $a_1$ to always
    be positive, and absorb the sign change by a shift of $\delta$ by $\pi$
    or ssomething.  Maybe we should clarify the range allowed for $\delta$.)} }
\label{table3}
\begin{center}
\begin{tabular}{|C{1cm}|C{1cm}||C{3cm}|C{3cm}|C{3cm}|}
%\begin{tabular}{|c|r||r|c||r|c||r|c|}
\hhline{|=====|}
%\hhline{|--------|}
$\kappa_t$&$\tilde{\kappa}_t$~~&$a_0$~~&$a_1$& $\delta$~~ \\ 
\hhline{|=====|} 
%\hhline{|--------|}
$1$ & $-1$~~~ & $0.1592 \pm 0.0006$ & $-0.0139 \pm 0.0008$ & $0.81 \pm 0.07$ \\[0.6mm]
\hline
$1$ & $0$ & $0.1595 \pm 0.0006$ & $-0.0181 \pm 0.0008$ & $0.002 \pm 0.06\,\,$ \\[0.6mm]
\hline
$1$ & $1$ & $0.1591 \pm 0.0006$ & $-0.0131 \pm 0.0008 $ & $\,-0.82 \pm 0.07\quad$  \\[0.6mm]
\hline
$0$ & $1$ & $0.1591 \pm 0.0006$ & ~~$\,0.0102 \pm 0.0008$ & $0.11 \pm 0.08$ \\
\hhline{|=====|}
%\hhline{|--------|}
\end{tabular}
\end{center} 
\end{table}
%%%%%%%%%%%%%%%%%%%%%%%%%%%%%%%%%%%%%%%%%%%%%%%%%%%%%
\begin{table}[H]
\caption{Fit results for the angular distribution $d\sigma/(\sigma
  d\Delta\phi_2(n_t,n_{\tbar}))$ (related to the TP
  $\epsilon_2=\epsilon(Q,\tbar,n_t,n_{\tbar})$), with the $\Delta
  R_{\ell\ell}$ cut turned off. As was the case in Table~\ref{table3},
  the sign of the parameter $a_1$ changes for
  $\kp=0,\kp=1$.}
\label{table4}
\begin{center}
\begin{tabular}{|C{1cm}|C{1cm}||C{3cm}|C{3cm}|C{3cm}|}
%\begin{tabular}{|c|r||r|c||r|c||r|c|}
\hhline{|=====|}
%\hhline{|--------|}
$\kappa_t$&$\tilde{\kappa}_t$~~&$a_0$~~&$a_1$& $\delta$~~ \\ 
\hhline{|=====|} 
%\hhline{|--------|}
$1$ & $-1$~~~ & $0.1591 \pm 0.0006$ & $-0.0146 \pm 0.0008$ & $0.73 \pm 0.06$ \\[0.6mm]
\hline
$1$ & $0$ & $0.1594 \pm 0.0007$ & $-0.0190 \pm 0.0008$ & $0.005 \pm 0.06\,\,$ \\[0.6mm]
\hline
$1$ & $1$ & $0.1592 \pm 0.0006$ & $-0.0136 \pm 0.0008 $ & $\,-0.77 \pm 0.07\quad$  \\[0.6mm]
\hline
$0$ & $1$ & $0.1591 \pm 0.0006$ & ~~$\,0.0113 \pm 0.0008$ & $0.09 \pm 0.08$ \\
\hhline{|=====|}
%\hhline{|--------|}
\end{tabular}
\end{center} 
\end{table}
\par

%%%%%%%%%%%%%%%%%%%%%%%%% FIGURA 7 %%%%%%%%%%%%%%%%%%%%%%%%%
\begin{center}
%\vspace*{-4mm}
\begin{figure}[H]
%\centering
\hspace*{-0.52cm}
\subfloat{\includegraphics[scale=0.45]{TP1_11_nocuts_nuevo.pdf}}
\hspace*{-0.006\textwidth}
\subfloat{\includegraphics[scale=0.45]{TP1_1-1_nocuts_nuevo.pdf}} \\
\hspace*{-0.52cm}
%\hspace*{0.03\textwidth}
\subfloat{\includegraphics[scale=0.45]{TP2_11_nocuts_nuevo.pdf}}
\hspace*{-0.006\textwidth}
\subfloat{\includegraphics[scale=0.45]{TP2_1-1_nocuts_nuevo.pdf}}
\caption{Angular distributions $d\sigma/(\sigma
  d\Delta\phi_1(n_t,n_{\tbar}))$ (top) and $d\sigma/(\sigma
  d\Delta\phi_2(n_t,n_{\tbar}))$ (bottom) associated with the TPs
  $\epsilon_1=\TPa$ and $\epsilon_2=\TPb$, respectively, for the
  $\mathrm{CP}$-mixed cases $\kp=\kpt=1$ (left) and $\kp=-\kpt=1$
  (right).  The $\Delta R_{\ell\ell}$ cut was turned off when
  generating these results. The
  corresponding fit curves [see Eq.~(\ref{eq27})] are displayed in red.}
\label{fig7}
\end{figure}
\end{center}
\par

In Sec.~\ref{sec3.1} we defined a fourth triple product,
$\epsilon_4 = \epsilon_3-\epsilon_2$.  We have constructed
an angular distribution related to this TP as well.
Specifically, we have analyzed the
$\Delta\phi(n_t,n_{\tbar})$ distribution in the $Q$ rest frame,
taking
$H$ to define the $z$-axis.  By studying the distributions
for various values of $\kp$ and $\kpt$, we have
found that they are not well described by Eq.~(\ref{eq27})
and their range of variation is larger than that of the distributions
displayed in Figs.~\ref{fig5} and \ref{fig6}.
{\bf (KK: I'm not sure what this means.  Are they less cosine-like?
  Or are the amplitudes or phase shifts larger than those for the other
  TPs?)}
Unlike the distributions
related to $\epsilon_1$-$\epsilon_3$, those arising from $\epsilon_4$
exhibit small changes in their shapes {\bf (KK: I don't think I understand
  this, since we said that the range of variation was larger in the last
  sentence.  I think I'm missing something :) )} for all the considered
hypotheses and for this reason we have not included the corresponding
plots here. However, the larger range of variation of the $\epsilon_4$
distributions leads to higher values for the asymmetry (as can be seen
from Tables \ref{table1} and \ref{table2}) even when the changes in
the respective shapes are smaller {\bf (KK: I don't understand what
  this means.)} than in the case of the
distributions described by Eq.~(\ref{eq27}).  \par
%%%%%%%%%%%%%%%%%%%%%%%%%%%%%%%%%%%%%%%%%%%%%%%%%%%%%%%%%%%%%%%%%%%%%%%%%%
\subsection{Mean value}
\label{sec3.3}
We turn now to consider the last type of observable that we will
construct from the TPs, the mean value. As was the case for the
observables considered in Secs.~\ref{sec3.1} and \ref{sec3.2},
the mean value is sensitive to $\kpt$.  Given a certain
TP, we define its mean value in the following manner,
%
\beq
\label{eq28}
\langle \epsilon \rangle = \frac{\int\epsilon\, \{d\sigma (pp\to b\,\ell^+\nu_{\ell}\,\bbar\,\ell^-\bar{\nu}_{\ell}H)/ d\Phi\}\,d\Phi}{\int \{d\sigma (pp\to b\,\ell^+\nu_{\ell}\,\bbar\,\ell^-\bar{\nu}_{\ell}H)/ d\Phi\}\,d\Phi},
\eeq
%
{\bf (KK: PRD might change these to [\ldots].)}
where $\Phi$ is the Lorentz-invariant phase space corresponding to the
final state
$b\,\ell^+\nu_{\ell}\,\bbar\,\ell^-\bar{\nu}_{\ell}H$. From
Eq.~(\ref{eq20}) we see that only the terms linear in (both) $\kp$ and $\kpt$
will contribute to the mean value.  Thus, we expect this observable
to be sensitive not only to the magnitude of $\kp\kpt$,
but also to the relative sign of
the couplings.\par

The results obtained for the TPs $\epsilon_1=
\epsilon(t,\tbar,n_t,n_{\tbar})$, $\epsilon_2=
\epsilon(Q,\tbar,n_t,n_{\tbar})$ and $\epsilon_3
=\epsilon(Q,t,n_t,n_{\tbar})$ introduced in Sec.~\ref{sec2} are
displayed in Table~\ref{table5}.  For each TP we list the 
mean value divided by the
corresponding statistical uncertainty.
{\bf (KK: I simplified this sentence and
  removed the language about the estimator -- we can
  certainly put it back in, but to be consistent, wouldn't we
  need to say $\bar{\epsilon}/\sigma_{\bar{\epsilon}}$?}
%A deviation  of the estimator of $\langle
%\epsilon \rangle$, $\bar{\epsilon}$.
We see that the three observables are capable of distinguishing the SM
case from both $\mathrm{CP}$-mixed cases. Furthermore,
two $\mathrm{CP}$-mixed cases
are clearly disentangled, since the the observables are
sensitive to the sign of $\kpt$. The observables $\langle \epsilon_2
\rangle$ and $\langle \epsilon_3 \rangle$ appear to be slightly more
sensitive than $\langle \epsilon_1 \rangle$. On the other hand, the
mean value for the combination $\epsilon_4$ introduced in
Sec.~\ref{sec3.1} gives slightly smaller values than those listed in
Table \ref{table5}, yielding $-4.32$, $1.11$ and $7.23$ for the cases
$(\kp=1,\kpt=-1,0,1)$, respectively. As with the asymmetry, the
purely $\mathrm{CP}$-even and purely $\mathrm{CP}$-odd cases cannot be
distinguished by the mean value, since it is linear in both $\kp$ and
$\kpt$ (see Eqs.~(\ref{eq20}) and (\ref{eq28})). Comparing the results
in Table~\ref{table5} with
the results presented in Sec.~\ref{sec3.1}, we can conclude that
the sensitivity to the NP contribution is smaller for the mean values
of the TPs under consideration
than for the corresponding asymmetries.


\renewcommand{\arraystretch}{1.6}
\begin{table}[H]
\caption{Mean values obtained for the TPs $\epsilon_{1,2,3}$ for the
  SM case and two $\mathrm{CP}$-mixed cases.
  The values are obtained using a sample of $10^5$
  simulated events.}
\label{table5}
\begin{center}
\begin{tabular}{|C{1cm}|C{1cm}||C{2cm}||C{2cm}||C{2cm}|}
%\hhline{|=====|}
\hhline{|-----|}
$\kappa_t$ & $\tilde{\kappa}_t$ & $\langle \epsilon_1 \rangle /\sigma_{\bar{\epsilon}_1}$ & $\langle \epsilon_2 \rangle /\sigma_{\bar{\epsilon}_2}$ & $\langle \epsilon_3 \rangle /\sigma_{\bar{\epsilon}_3}$ \\ 
\hhline{|=====|} 
%\hhline{|-----|}
%\vspace*{.5mm}
\renewcommand{\arraystretch}{1.0}
$1$~ & $-1$~~~ & $4.26$~ & $4.94$~ & $-5.81$~~~\\[0.6mm]
\hline
$1$ & $0$~ & $-0.91$~~~ & $-0.22$~~~ & $1.25$~\\[0.6mm]
\hline
$1$ & $1$~ & $-7.98$~~~ & $-8.83$~~~& $8.75$~\\[0.6mm]
\hhline{|=====|}
%\hhline{|-----|}
\end{tabular}
\end{center} 
\end{table}

%%%%%%%%%%%%%%%%%%%%%%%%%%%%%%%%%%%%%%%%%%%%%%%%% 
\section{{\boldmath $\mathrm{CP}$}-odd observables not depending on \MakeLowercase{{\boldmath $t$}} and \MakeLowercase{{\boldmath $\tbar$}} spin vectors}
\label{sec4}
So far we have considered three TPs involving the momenta $t,\tbar$
and $Q$ and the spin vectors $n_t$ and $n_{\tbar}$ [defined in
Eqs.~(\ref{eq3})-(\ref{eq4}].  Furthermore, we have described the general
form of the differential cross-section in terms of these vectors in
Eq.~(\ref{eq20}). In this section we consider other
possibilities for the choice of the vectors from which the
$\mathrm{CP}$-odd observables can be constructed. From the definitions
in Eqs.~(\ref{eq3})-(\ref{eq4}), we see that the TPs
$\epsilon_{1,2,3}$ can be written as follows,
%
\beq
\label{eq29}
\TPa = \frac{m^2_t}{(t\cdot \ell^+)(\tbar\cdot \ell^-)}\,\epsilon(t,\tbar,\ell^-\!,\ell^+),
\eeq
%
\vspace*{1mm}
\beq
\label{eq30}
\TPb = \frac{m^2_t}{(t\cdot \ell^+)(\tbar\cdot \ell^-)}\left(\epsilon(t,\tbar,\ell^-\!,\ell^+)+\epsilon(H,\tbar,\ell^-\!,\ell^+)+\frac{(t\cdot\ell^+)}{m^2_t}\epsilon(H,\tbar,t,\ell^-\!)\right) \,,
\eeq
%
\vspace*{1mm}
\beq
\label{eq31}
\TPc = \frac{m^2_t}{(t\cdot \ell^+)(\tbar\cdot \ell^-)}\left(-\epsilon(t,\tbar,\ell^-\!,\ell^+)+\epsilon(H,t,\ell^-\!,\ell^+)+\frac{(\tbar\cdot\ell^-)}{m^2_t}\epsilon(H,\tbar,t,\ell^+\!)\right).
\vspace*{2mm}
\eeq
%
The above equations express the TPs studied in the last sections as
linear combination of TPs involving the momenta $t,\tbar,H,\ell^+$ and
$\ell^-$, with coefficients that are functions of phase space
variables. These five momenta give rise to five TPs whose sensitivity
can also be tested by means of the observables introduced in
Secs.~\ref{sec3.1}-\ref{sec3.3}.  We have found
that TPs that do not include both the lepton and anti-lepton momenta yield
negligible sensitivity to the value of $\kpt$.  For this reason,
we concentrate here on the
results obtained for the remaining TPs,
%
\beq
\label{eqa5}
\epsilon_5\equiv\epsilon(t,\tbar,\ell^-\!,\ell^+) \,,
\vspace*{1mm}
\eeq
%
\beq
\label{eqa6}
\epsilon_6\equiv\epsilon(H,t,\ell^-\!,\ell^+) \,,
\vspace*{1mm}
\eeq
%
\beq
\label{eqa7}
\epsilon_7\equiv\epsilon(H,\tbar,\ell^-\!,\ell^+) \,.
\vspace*{1mm}
\eeq
%
    {\bf (KK: Should we clarify somehow that $\epsilon_{4,5,6,7}$ are not the same as
      $\epsilon_{4,5,6,7}$ in Eq.~(20)?  That seems like a potential source of confusion.)}

\renewcommand{\arraystretch}{1.6}
\begin{table}[H]
\caption{Mean values obtained for the TPs $\epsilon_{5,6,7}$ for the
  SM case and two $\mathrm{CP}$-mixed cases with opposite sign in the
  pseudoscalar coupling. The values correspond to $10^5$ simulated
  events.}
\label{table6}
\begin{center}
\begin{tabular}{|C{1cm}|C{1cm}||C{2cm}||C{2cm}||C{2cm}|}
%\hhline{|=====|}
\hhline{|-----|}
$\kappa_t$ & $\tilde{\kappa}_t$ & $\langle \epsilon_5 \rangle /\sigma_{\bar{\epsilon}_5}$ & $\langle \epsilon_6 \rangle /\sigma_{\bar{\epsilon}_6}$ & $\langle \epsilon_7 \rangle /\sigma_{\bar{\epsilon}_7}$ \\ 
\hhline{|=====|} 
%\hhline{|-----|}
%\vspace*{.5mm}
\renewcommand{\arraystretch}{1.0}
$1$~ & $-1$~~~ & $3.98$~ & $-1.96$~~~ & $1.69$~ \\[0.6mm]
\hline
$1$ & $0$~ & $-0.43$~~~ & $1.25$~ & $0.74$~ \\[0.6mm]
\hline
$1$ & $1$~ & $-6.76$~~~ & $3.46$~ & $-3.29$~~~~\\[0.6mm]
\hhline{|=====|}
%\hhline{|-----|}
\end{tabular}
\end{center} 
\end{table}
\renewcommand{\arraystretch}{1.4}
\begin{table}[H]
\caption{Asymmetries for the TPs $\epsilon_{5,6,7}$ for the SM case
  and the two $\mathrm{CP}$-mixed cases defined by $\kp=1,\kpt=\pm
  1$. The values correspond to $10^5$ simulated events.}
\label{table7}
\begin{center}
\begin{tabular}{|C{1cm}|C{1cm}||C{2cm}|C{2cm}||C{2cm}|C{2cm}||C{2cm}|C{2cm}|}
%\begin{tabular}{|c|r||r|c||r|c||r|c|}
\hhline{|========|}
%\hhline{|--------|}
$\kappa_t$&$\tilde{\kappa}_t$~~&$\mathcal{A}(\epsilon_5)$~~&$\mathcal{A}(\epsilon_5)/\sigma_{\mathcal{A}}$& $\mathcal{A}(\epsilon_6)$~~&$\mathcal{A}(\epsilon_6)/\sigma_{\mathcal{A}}$&$\mathcal{A}(\epsilon_7)$~~&$\mathcal{A}(\epsilon_7)/\sigma_{\mathcal{A}}$  \\ 
\hhline{|========|} 
%\hhline{|--------|}
$1$ & $-1$~~~ & $0.0315$~ & $10.0$~ & $-0.0134$~~~ & $-4.2$~~~ & $0.0111$~ & $3.5$~\\[0.6mm]
\hline
$1$ & $0$ & $-0.0021$~~~ & $-0.7$~~~ & $-0.0011$~~~ & $-0.3$~~~ & $0.0009$~& $0.3$~\\[0.6mm]
\hline
$1$ & $1$ & $-0.0379$~~~ & $-12.0$~~~ & $0.0143$~ & $4.5$~ & $-0.0137$~~~ & $\,-4.3$~~~  \\[0.6mm]
\hhline{|========|}
%\hhline{|--------|}
\end{tabular}
\end{center} 
\end{table}
\par
%

Tables~\ref{table6} and
\ref{table7} summarize the results for the TPs $\epsilon_{5,6,7}$.
    {\bf (KK: Should we interchange the two tables?  In the previous section
      we considered asymmetries first, then mean values.  Seems like it might
      flow better to follow that same order here.)}
We see that the TP $\epsilon_5$ gives rise to asymmetries and mean
values that are clearly higher than those obtained from $\epsilon_6$
and $\epsilon_7$. This is in contrast to the TPs $\epsilon_{1,2,3}$,
for which the asymmetries and mean values are comparable for the TPs
(see Tables
\ref{table1} and \ref{table5}). We also note that the asymmetry for
$\epsilon_5$ is exactly the same as for $\epsilon_1$, as \DIFdelbegin \DIFdel{was }\DIFdelend \DIFaddbegin \DIFadd{is }\DIFaddend expected
from Eq.~(\ref{eq29})\DIFaddbegin \DIFadd{, }\DIFaddend since the proportionality factor relating them
is positive definite. Regarding the mean values, we see by comparing
Tables~\ref{table5} and \ref{table6} that the TPs $\epsilon_{1,2,3}$
appear to have a higher sensitivity to the pseudoscalar coupling than
$\epsilon_{5,6,7}$.  \par
\DIFaddbegin 

\DIFaddend It is important to mention that in the
$t\tbar$ rest frame the sign of the TP $\epsilon_5$ is defined through
the angle $\Delta\phi_{\ell^-\ell^+}$ \DIFdelbegin \DIFdel{(recall the discussion on
}\DIFdelend \DIFaddbegin [\DIFadd{see the discussion following
  }\DIFaddend Eq.~(\ref{eq28}) \DIFdelbegin \DIFdel{)}\DIFdelend \DIFaddbegin {\bf \DIFadd{(KK: should this be Eq~(\ref{eq25})?)}}]\DIFaddend ,
which is the angular difference between the
projections of the leptons\DIFaddbegin \DIFadd{' }\DIFaddend momenta onto the plane perpendicular to
$\vec{\tbar}$. By a similar argument to that discussed at the
beginning of Sec.~\ref{sec3.2}, we can construct an associated angular
distribution \DIFdelbegin \DIFdel{that, being constrained by the $\mathcal{A}(\epsilon_5)$,
will be also }\DIFdelend \DIFaddbegin [\DIFadd{see Eq.~(\ref{eq26})}] \DIFadd{that
will be }\DIFaddend sensitive to the sign of the pseudoscalar coupling. This
angular variable is \DIFdelbegin \DIFdel{nothing but }\DIFdelend \DIFaddbegin \DIFadd{the same as }\DIFaddend that proposed in \DIFaddbegin \DIFadd{Ref.~}\DIFaddend \cite{Ellis} as a
useful $\mathrm{CP}$-odd observable. Moreover, it is shown in
\DIFdelbegin \DIFdel{\mbox{%DIFAUXCMD
\cite{Ellis} }%DIFAUXCMD
that this }\DIFdelend \DIFaddbegin \DIFadd{Ref.~\mbox{%DIFAUXCMD
\cite{Ellis} }%DIFAUXCMD
that the corresponding
}\DIFaddend angular distribution follows the functional
form given in Eq.~(\ref{eq27}). Due to the fact that this distribution
is constrained by \DIFaddbegin \DIFadd{the asymmetry }\DIFaddend $\mathcal{A}(\epsilon_5)$
\DIFdelbegin \DIFdel{which }\DIFdelend \DIFaddbegin [\DIFadd{via Eq.~(\ref{eq26})}]\DIFadd{, }{\bf \DIFadd{(KK: correct, I think\ldots?)}}
\DIFadd{which in turn }\DIFaddend is equal to
$\mathcal{A}(\epsilon_1)$ and smaller than
$\mathcal{A}(\epsilon_{2,3})$, the corresponding shifts ($\delta$)
obtained for different values of $\kpt$ are expected to be of the same
order \DIFdelbegin \DIFdel{than }\DIFdelend \DIFaddbegin \DIFadd{as }\DIFaddend those exhibited by the $\Delta\phi_1(n_t,n_{\tbar})$
distribution and somewhat smaller than those \DIFdelbegin \DIFdel{observed in }\DIFdelend \DIFaddbegin \DIFadd{obtained for }\DIFaddend the
$\Delta\phi_2(n_t,n_{\tbar})$ distribution.
\DIFaddbegin {\bf \DIFadd{(KK: A few comments\ldots (i) From Table I, it doesn't seem like
  $\mathcal{A}(\epsilon_1)<\mathcal{A}(\epsilon_{3})$; (ii) I'm not
  sure about this argument
  regarding $\delta$ -- wouldn't the value of $\mathcal{A}$ depend
  on both the phase shift and the amplitude of the cosine?  For example,
  assuming
  the functional form from the fit, it depends on $a_1\sin(\delta)$; but
  that expression is not exact; (iii) I wonder if it might be best to
  drop the connection to $\epsilon_2$, which seems a bit more tenuous,
  and just make the comparison with $\epsilon_1$ and its $\delta$.  I suggest
  the sentence that follows\ldots.)}}
\DIFadd{Since this distribution
is constrained by the asymmetry $\mathcal{A}(\epsilon_5)$
}[\DIFadd{via Eq.~(\ref{eq26})}]\DIFadd{,
which in turn is equal to
$\mathcal{A}(\epsilon_1)$, the corresponding shifts ($\delta$)
obtained for different values of $\kpt$ are expected to be of the same
order as those exhibited by the $\Delta\phi_1(n_t,n_{\tbar})$
distribution.
}

\DIFaddend \par
\DIFdelbegin \DIFdel{In analogy to }\DIFdelend \DIFaddbegin 

\DIFadd{In analogy with }\DIFaddend the
combination of TPs considered in Sec.~\ref{sec3}, we have found a
combination of the TPs $\epsilon_{5,6,7}$ for which the asymmetry is
enhanced with respect to $\epsilon_5$-$\epsilon_7$,
%
\beq
\label{eq32}
\epsilon_8 = 2\epsilon_5 -\epsilon_6 +\epsilon_7 = \epsilon(t+\tbar+H,t-\tbar,\ell^+,\ell^-).
%2\epsilon(t,\tbar,\ell^-,\ell^+)+\epsilon(H,\tbar,\ell^-,\ell^+)-\epsilon(H,t,\ell^-,\ell^+)
\eeq
%
We see from Eq.~(\ref{eq32}) that in the $t\tbar H$ rest frame
$\epsilon_8=M_{t\tbar H}(\vec{t}-\vec{\tbar})\cdot (\vec{\ell}^+\!
\times \vec{\ell}^-)$, where $M_{t\tbar H}$ is the invariant mass of
the \DIFdelbegin \DIFdel{system }\DIFdelend $t\tbar H$ \DIFaddbegin \DIFadd{system}\DIFaddend . Hence, in \DIFdelbegin \DIFdel{this reference }\DIFdelend \DIFaddbegin \DIFadd{the $t\tbar H$ rest }\DIFaddend frame the sign of 
\DIFdelbegin \DIFdel{the
combination }\DIFdelend $\epsilon_8$ is \DIFdelbegin \DIFdel{set }\DIFdelend \DIFaddbegin \DIFadd{determined }\DIFaddend by the quantity
$(\vec{t}-\vec{\tbar})\cdot (\vec{\ell}^+\! \times
\vec{\ell}^-)$. \DIFdelbegin \DIFdel{Taking into account }\DIFdelend \DIFaddbegin \DIFadd{Comparing }\DIFaddend Eqs.~(\ref{eq24}) and (\ref{eq32})
\DIFdelbegin \DIFdel{along with the definition }\DIFdelend \DIFaddbegin \DIFadd{and noting that }\DIFaddend $Q=(t+\tbar+H)/2$, we see that the only
relevant difference between $\epsilon_4$ and $\epsilon_8$ is that in
the latter the spin vectors $n_t$ and $n_{\tbar}$ have been replaced
by the momenta of the leptons $\ell^+$ and $\ell^-$\DIFaddbegin \DIFadd{, }\DIFaddend respectively. The
values obtained for $\mathcal{A}(\epsilon_8)$ are shown in Table
\ref{table8}. Compared to the TPs $\epsilon_5$-$\epsilon_7$ and
$\epsilon_1, \epsilon_3$ (see Tables \ref{table1} and \ref{table7}),
the chosen combination exhibits a slightly higher sensitivity in the
asymmetry for resolving the $\mathrm{CP}$-mixed cases. This is not
true in the case of the TPs $\epsilon_2$ and $\epsilon_4$, for which
the asymmetry is larger (see Tables \ref{table1}, \ref{table2} and
\ref{table8}). In particular, the comparison between $\epsilon_4$ and
$\epsilon_8$ indicates that the use of the momenta of the leptons
instead of the spin vectors produces a decrease in the sensitivity of
the asymmetry. 
\DIFdelbegin %DIFDELCMD < \par %%%
\DIFdel{From the TPs }\DIFdelend \DIFaddbegin {\bf \DIFadd{(KK: I don't know that we should separate $\epsilon_2$ from
  $\epsilon_{1,3}$ in this way.  They basically agree within their
  uncertainties.  If we used another Monte Carlo data set, we might
  get a different ordering.  I would suggest the following rewording\ldots.)}}
\DIFadd{Compared to the TPs $\epsilon_1$-$\epsilon_3$ and
}\DIFaddend $\epsilon_5$-$\epsilon_7$ \DIFdelbegin \DIFdel{one can
derive various angular distributions in the same manner we have
discussed in }\DIFdelend \DIFaddbegin \DIFadd{(see Tables \ref{table1} and \ref{table7}),
the asymmetry for $\epsilon_8$ has a comparable or slightly higher
sensitivity for resolving the $\mathrm{CP}$-mixed cases.
Comparing with $\mathcal{A}(\epsilon_4)$, however,
we see that using the momenta of the leptons (in $\epsilon_8$)
instead of the spin vectors produces a decrease in the sensitivity of
the asymmetry.
}

\DIFadd{Having defined the TPs $\epsilon_5$-$\epsilon_7$, one can
proceed, as in }\DIFaddend Sec.~\ref{sec3.2}\DIFdelbegin \DIFdel{for $\epsilon_1$-$\epsilon_3$}\DIFdelend \DIFaddbegin \DIFadd{, to define corresponding
angular distributions}\DIFaddend .  Of
course, in this case the corresponding \DIFdelbegin \DIFdel{angle }\DIFdelend \DIFaddbegin \DIFadd{angles }\DIFaddend will be defined in terms
of the momenta of the leptons instead of using the spin vectors. \DIFdelbegin \DIFdel{These
distributions }\DIFdelend \DIFaddbegin \DIFadd{The
angular distributions based on $\epsilon_5$-$\epsilon_7$
}\DIFaddend have the same \DIFdelbegin \DIFdel{behaviour than }\DIFdelend \DIFaddbegin \DIFadd{overall behaviour as }\DIFaddend those derived from
$\epsilon_1$-$\epsilon_3$\DIFdelbegin \DIFdel{, but only the shift obtained in the case of
$\epsilon_5$ is compatible to the values given in Tables }\DIFdelend \DIFaddbegin \DIFadd{.  Using Eq.~(\ref{eq27})
to fit the distributions and comparing to the results obtained }\DIFaddend for
$\epsilon_1$-$\epsilon_3$\DIFdelbegin \DIFdel{. The shifts resulting from the fit of the
distributions related to }\DIFdelend \DIFaddbegin \DIFadd{, we find that the phase shifts ($\delta$)
are comparable for the $\epsilon_5$ angular distribution, but are
smaller for the
}\DIFaddend $\epsilon_6$ and $\epsilon_7$ \DIFdelbegin \DIFdel{are
smaller.  }\DIFdelend \DIFaddbegin \DIFadd{distributions.  }{\bf \DIFadd{(KK: rewording OK/correct?)}}
\DIFaddend Taking into account these facts we do not give further
details of the angular distributions for the TPs discussed in this
section.
\DIFaddbegin {\bf \DIFadd{(KK: I would suggest deleting this last sentence.)}}
\DIFaddend \par
\DIFdelbegin \DIFdel{The mean value }\DIFdelend \DIFaddbegin 

\DIFadd{The mean values }\DIFaddend of $\epsilon_8$ for the \DIFdelbegin \DIFdel{hypotheses under
consideration is }\DIFdelend \DIFaddbegin \DIFadd{scenarios under
consideration are }\DIFaddend comparable with the values listed in Table
\ref{table6} for $\epsilon_5$. Concerning the associated angular
distributions, their range of variation \DIFaddbegin {\bf \DIFadd{(KK: I'm not sure what
  this means)}} \DIFaddend is larger than in the case of
the distributions related to $\epsilon_5$-$\epsilon_7$ but exhibit
smaller changes in their shapes for the different \DIFdelbegin \DIFdel{hypothesis }\DIFdelend \DIFaddbegin \DIFadd{scenarios }\DIFaddend and for
this reason we do not include here the corresponding plots.
\DIFaddbegin {\bf \DIFadd{(KK: might need to reword this a bit after clarifying what 
  range of variation means.)}}
\DIFaddend \begin{table}[H]
\caption{Asymmetry for the TP $\epsilon_{8}$ for the SM case and the
  two $\mathrm{CP}$-mixed cases defined by $\kp=1,\kpt=\pm 1$. The
  values are obtained with $10^5$ simulated events.}
\label{table8}
\begin{center}
\begin{tabular}{|C{1cm}|C{1cm}||C{2cm}|C{2cm}|}
%\begin{tabular}{|c|r||r|c||r|c||r|c|}
\hhline{|====|}
%\hhline{|--------|}
$\kappa_t$&$\tilde{\kappa}_t$~~&$\mathcal{A}(\epsilon_8)$~~&$\mathcal{A}(\epsilon_8)/\sigma_{\mathcal{A}}$ \\ 
\hhline{|====|} 
%\hhline{|--------|}
$1$ & $-1$~~~ & $0.0331$~ & $10.5$~ \\[0.6mm]
\hline
$1$ & $0$ & $0.0023$~ & $0.7$~ \\[0.6mm]
\hline
$1$ & $1$ & $-0.0403$~~~ & $\,-12.7$~~~~ \\[0.6mm]
\hhline{|====|}
%\hhline{|--------|}
\end{tabular}
\end{center} 
\end{table}
\section{{\boldmath{$\mathrm{CP}$}}-odd observables not depending on \MakeLowercase{{\boldmath $t$}} and \MakeLowercase{{\boldmath $\tbar$}} momenta}
\label{sec5}
\DIFdelbegin \DIFdel{All the }\DIFdelend \DIFaddbegin \DIFadd{The }\DIFaddend observables discussed in the \DIFdelbegin \DIFdel{above sections }\DIFdelend \DIFaddbegin \DIFadd{preceding sections all }\DIFaddend involve the
momenta of the top and\DIFaddbegin \DIFadd{/or }\DIFaddend anti-top quarks \DIFdelbegin \DIFdel{thus requiring }\DIFdelend \DIFaddbegin \DIFadd{and thus require }\DIFaddend the full
reconstruction of the kinematics of the individual $t$ and $\tbar$
systems in order to be measured. Although challenging due to the
presence of \DIFaddbegin \DIFadd{the }\DIFaddend two neutrinos in the final state, this can \DIFdelbegin \DIFdel{be in principle }\DIFdelend \DIFaddbegin \DIFadd{in principle be
}\DIFaddend done by applying a kinematic reconstruction method such \DIFdelbegin \DIFdel{that }\DIFdelend \DIFaddbegin \DIFadd{as }\DIFaddend the
neutrino weighting technique\DIFaddbegin \DIFadd{~}\DIFaddend \cite{atlasconf,atlascharge}. Another
possibility is to define observables that \DIFdelbegin \DIFdel{are not dependent }\DIFdelend \DIFaddbegin \DIFadd{do not depend }\DIFaddend on the $t$
and $\tbar$ momenta but \DIFaddbegin \DIFadd{instead }\DIFaddend make use of the \DIFdelbegin \DIFdel{quarks }\DIFdelend \DIFaddbegin \DIFadd{momenta of the
}\DIFaddend $b$ and $\bbar$ \DIFaddbegin \DIFadd{quarks to which the $t$ and $\tbar$ decay}\DIFaddend . In
order to construct such observables we will \DIFdelbegin \DIFdel{modify here }\DIFdelend \DIFaddbegin \DIFadd{take as our starting
point }\DIFaddend the most
sensitive observables studied in Secs.~\DIFdelbegin \DIFdel{\ref{table3} }\DIFdelend \DIFaddbegin \DIFadd{\ref{sec3} }\DIFaddend and \ref{sec4},
\DIFdelbegin \DIFdel{namely the combinations }\DIFdelend \DIFaddbegin {\bf \DIFadd{(KK: note that I changed the reference from table3 to sec3.)}}
\DIFadd{namely those associated with the TPs }\DIFaddend $\epsilon_4$ and $\epsilon_8$\DIFdelbegin \DIFdel{respectively. }\DIFdelend \DIFaddbegin \DIFadd{,
respectively.
}

\DIFaddend Let us first consider the \DIFaddbegin \DIFadd{TP }\DIFaddend combination $\epsilon_8$\DIFaddbegin \DIFadd{, which is
}\DIFaddend defined in Eq.~(\ref{eq32})\DIFdelbegin \DIFdel{and replace the top and antitop momenta by
the
bottom and antibottom momenta respectively. Thus, we define
}\DIFdelend \DIFaddbegin \DIFadd{.  Replacing the momenta of the
$t$ and $\tbar$ quarks by the momenta of the $b$ and $\bbar$ quarks,
respectively, we have a new TP,
}\DIFaddend %
\beq
\label{eq33}
\epsilon_9 = \epsilon(b+\bbar+H,b-\bbar,\ell^+,\ell^-).
\eeq
%
Note that \DIFdelbegin \DIFdel{now }\DIFdelend the sign of \DIFdelbegin \DIFdel{the TP }\DIFdelend $\epsilon_9$ is \DIFdelbegin \DIFdel{set }\DIFdelend \DIFaddbegin \DIFadd{determined }\DIFaddend by the sign of
the quantity $(\vec{b}-\vec{\bbar})\cdot(\vec{\ell}^+\!\times
\vec{\ell}^-)$ in the $b\bbar H$ rest frame. This \DIFdelbegin \DIFdel{quantity is in fact
used in \mbox{%DIFAUXCMD
\cite{Guadagnoli} }%DIFAUXCMD
but in }\DIFdelend \DIFaddbegin \DIFadd{combination
of three vectors (determined in }\DIFaddend the lab frame \DIFdelbegin \DIFdel{in order }\DIFdelend \DIFaddbegin \DIFadd{instead
of the $b\bbar H$ rest frame) is
used in Ref.~\mbox{%DIFAUXCMD
\cite{Guadagnoli} }%DIFAUXCMD
}\DIFaddend to define a
$\mathrm{CP}$-odd observable that only depends on lab frame
variables. The values of the asymmetry for $\epsilon_9$ are listed in
Table\DIFdelbegin \DIFdel{\ref{table9}. By comparing Tables}\DIFdelend \DIFaddbegin \DIFadd{~\ref{table9}. Comparing Tables~}\DIFaddend \ref{table8} and \ref{table9}
we see that the use of the \DIFdelbegin \DIFdel{$b,\bbar$ instead of $t,\tbar$ }\DIFdelend \DIFaddbegin \DIFadd{$b$ and $\bbar$ momenta
instead of the $t$ and $\tbar$ momenta }\DIFaddend leads to a
\DIFdelbegin \DIFdel{decrese }\DIFdelend \DIFaddbegin \DIFadd{decrease }\DIFaddend in the sensitivity of the asymmetry by $\sim 5\sigma$ for
$\kp=1,\kpt=\pm 1$. \DIFdelbegin \DIFdel{However}\DIFdelend \DIFaddbegin \DIFadd{Nevertheless}\DIFaddend , the observable \DIFdelbegin \DIFdel{is still
capable of
discriminating }\DIFdelend \DIFaddbegin \DIFadd{can still
discriminate }\DIFaddend not only between the two $\mathrm{CP}$-mixed \DIFdelbegin \DIFdel{hypotheses
}\DIFdelend \DIFaddbegin \DIFadd{scenarios
}\DIFaddend but also between these and the SM case.
%\newpage
\DIFdelbegin \DIFdel{We proceed now }\DIFdelend \DIFaddbegin 

%DIF > 
\begin{table}[th]
\caption{\DIFaddFL{Asymmetry for the TP $\epsilon_{9}$ for the SM case and the
  two $\mathrm{CP}$-mixed cases defined by $\kp=1,\kpt=\pm 1$. The
  values are obtained with $10^5$ simulated events.
  }{\bf \DIFaddFL{(KK: I repositioned the table in the file, please double check to make
  sure I didn't mess anything up :)}}}
\label{table9}
\begin{center}
\begin{tabular}{|C{1cm}|C{1cm}||C{2cm}|C{2cm}|}
%DIF > \begin{tabular}{|c|r||r|c||r|c||r|c|}
\hhline{|====|}
%DIF > \hhline{|--------|}
\DIFaddFL{$\kappa_t$}&\DIFaddFL{$\tilde{\kappa}_t$~~}&\DIFaddFL{$\mathcal{A}(\epsilon_9)$~~}&\DIFaddFL{$\mathcal{A}(\epsilon_9)/\sigma_{\mathcal{A}}$ }\\ 
\hhline{|====|} 
%DIF > \hhline{|--------|}
\DIFaddFL{$1$ }& \DIFaddFL{$-1$~~~ }& \DIFaddFL{$0.0171$~ }& \DIFaddFL{$5.4$~ }\\[0.6mm]
\hline
\DIFaddFL{$1$ }& \DIFaddFL{$0$ }& \DIFaddFL{$0.0010$~ }& \DIFaddFL{$0.3$~ }\\[0.6mm]
\hline
\DIFaddFL{$1$ }& \DIFaddFL{$1$ }& \DIFaddFL{$-0.0247$~~~ }& \DIFaddFL{$\,-7.8$~~~~ }\\[0.6mm]
\hhline{|====|}
%DIF > \hhline{|--------|}
\end{tabular}
\end{center} 
\end{table}
%DIF > 

\DIFadd{We proceed }\DIFaddend in a similar \DIFdelbegin \DIFdel{way with the combination }\DIFdelend \DIFaddbegin \DIFadd{manner with the TP }\DIFaddend $\epsilon_4$. \DIFdelbegin \DIFdel{By
using }\DIFdelend \DIFaddbegin \DIFadd{Starting from
}\DIFaddend Eq.~(\ref{eq24}) \DIFdelbegin \DIFdel{along with }\DIFdelend \DIFaddbegin \DIFadd{and using }\DIFaddend the definitions of the spin vectors
in Eqs.~(\ref{eq3})-(\ref{eq4}), we \DIFdelbegin \DIFdel{can write
}\DIFdelend \DIFaddbegin \DIFadd{have
}\DIFaddend %
\beq
\label{eq34}
\epsilon_4 = \frac{m^2_t}{(t\cdot\ell^+)\cdot(\tbar\cdot\ell^-)}\,\epsilon(Q,t-\tbar,\ell^-,\ell^+)+\frac{1}{(t\cdot\ell^+)}\,\epsilon(Q,t,\ell^+,\tbar)-\frac{1}{(\tbar\cdot \ell^-)}\,\epsilon(Q,\tbar,t,\ell^-).
\eeq
%
Since the asymmetry is not changed by \DIFaddbegin \DIFadd{the presence
of }\DIFaddend an overall positive definite \DIFdelbegin \DIFdel{factor, we will concentrate }\DIFdelend \DIFaddbegin \DIFadd{multiplicative
factor, let us concentrate instead }\DIFaddend on the following combination
\DIFdelbegin \DIFdel{arising from
the expression in Eq.~(\ref{eq34})}\DIFdelend \DIFaddbegin \DIFadd{of TPs}\DIFaddend ,
%
\beq
\label{eq35}
\epsilon(Q,t-\tbar,\ell^-,\ell^+)+\frac{(\tbar\cdot \ell^-)}{m^2_t}\epsilon(Q,t,\ell^+,\tbar)-\frac{(t\cdot\ell^+)}{m^2_t}\epsilon(Q,\tbar,t,\ell^-)\DIFdelbegin \DIFdel{,
}\DIFdelend \DIFaddbegin \DIFadd{.
}\DIFaddend \eeq
%DIF < 
\DIFdelbegin %DIFDELCMD < \begin{table}[H]
%DIFDELCMD < %%%
%DIFDELCMD < \caption{%
{%DIFAUXCMD
\DIFdelFL{Asymmetry for the TP $\epsilon_{9}$ for the SM case and the
  two $\mathrm{CP}$-mixed cases defined by $\kp=1,\kpt=\pm 1$. The
  values are obtained with $10^5$ simulated events.}}
%DIFAUXCMD
%DIFDELCMD < %DIFDELCMD < \label{table9}%%%
%DIFDELCMD < \begin{center}
%DIFDELCMD < \begin{tabular}{|C{1cm}|C{1cm}||C{2cm}|C{2cm}|}
%DIFDELCMD < %%%
%DIF < \begin{tabular}{|c|r||r|c||r|c||r|c|}
%DIFDELCMD < \hhline{|====|}
%DIFDELCMD < %%%
%DIF < \hhline{|--------|}
\DIFdelFL{$\kappa_t$}%DIFDELCMD < &%%%
\DIFdelFL{$\tilde{\kappa}_t$~~}%DIFDELCMD < &%%%
\DIFdelFL{$\mathcal{A}(\epsilon_9)$~~}%DIFDELCMD < &%%%
\DIFdelFL{$\mathcal{A}(\epsilon_9)/\sigma_{\mathcal{A}}$ }%DIFDELCMD < \\ 
%DIFDELCMD < \hhline{|====|} 
%DIFDELCMD < %%%
%DIF < \hhline{|--------|}
\DIFdelFL{$1$ }%DIFDELCMD < & %%%
\DIFdelFL{$-1$~~~ }%DIFDELCMD < & %%%
\DIFdelFL{$0.0171$~ }%DIFDELCMD < & %%%
\DIFdelFL{$5.4$~ }%DIFDELCMD < \\[0.6mm]
%DIFDELCMD < \hline
%DIFDELCMD < %%%
\DIFdelFL{$1$ }%DIFDELCMD < & %%%
\DIFdelFL{$0$ }%DIFDELCMD < & %%%
\DIFdelFL{$0.0010$~ }%DIFDELCMD < & %%%
\DIFdelFL{$0.3$~ }%DIFDELCMD < \\[0.6mm]
%DIFDELCMD < \hline
%DIFDELCMD < %%%
\DIFdelFL{$1$ }%DIFDELCMD < & %%%
\DIFdelFL{$1$ }%DIFDELCMD < & %%%
\DIFdelFL{$-0.0247$~~~ }%DIFDELCMD < & %%%
\DIFdelFL{$\,-7.8$~~~~ }%DIFDELCMD < \\[0.6mm]
%DIFDELCMD < \hhline{|====|}
%DIFDELCMD < %%%
%DIF < \hhline{|--------|}
%DIFDELCMD < \end{tabular}
%DIFDELCMD < \end{center} 
%DIFDELCMD < \end{table}
%DIFDELCMD < \noindent
%DIFDELCMD < %%%
%DIF < 
\DIFdel{and instead }\DIFdelend \DIFaddbegin \DIFadd{Instead }\DIFaddend of replacing $t$ and $\tbar$ directly by $b$ and $\bbar$,
we use \DIFdelbegin \DIFdel{their visible parts}\DIFdelend \DIFaddbegin \DIFadd{the visible contributions}\DIFaddend , namely $b+\ell^+$ and $\bbar +\ell^-$\DIFaddbegin \DIFadd{,
}\DIFaddend respectively. This results in the following definition
%
\beq
\label{eq36}
\epsilon_{10}=\epsilon(\tilde{Q},c_{b\bbar}\,,\ell^-,\ell^+)-w_1\,\epsilon(\tilde{Q},b,\bbar,\ell^+)+w_2\,\epsilon(\tilde{Q},b,\bbar,\ell^-),
\eeq
%
where $\tilde{Q}\equiv (b+\ell^+\!+\bbar +\ell^-)/2$ stands for the
visible part of $Q$, $c_{b\bbar}=(1-w_1)\,b-(1-w_2)\,\bbar$\DIFaddbegin \DIFadd{, }\DIFaddend and the
weights $w_{1,2}$ are given by $(\bbar\cdot \ell^-)/m^2_t $ and
$(b\cdot \ell^+)/m^2_t$\DIFaddbegin \DIFadd{, }\DIFaddend respectively. Also, the contribution
$m^2_{\ell}/m^2_t$ has been neglected both in $w_1$ and in $w_2$. Note
that if we set $w_1=w_2=0$, the combination $\epsilon_{10}$ reduces to
$\epsilon_9 /2$ and $\mathcal{A}(\epsilon_{10})$ becomes equal to
$\mathcal{A}(\epsilon_9)$. The results \DIFdelbegin \DIFdel{obatined }\DIFdelend \DIFaddbegin \DIFadd{obtained }\DIFaddend for the asymmetry of
$\epsilon_{10}$ are given in Table \ref{table10}. By comparing Tables
\ref{table2} and \ref{table10} we see again that the sensitivity of
the asymmetry decreases when $t$ and $\tbar$ are not included in the
TP. Nevertheless, the combination $\epsilon_{10}$ remains a useful
observable for discriminating the $\mathrm{CP}$ nature of the Higgs
boson, with the corresponding asymmetry having a sensitivity \DIFdelbegin \DIFdel{even
}\DIFdelend \DIFaddbegin \DIFadd{that is
}\DIFaddend higher than that of $\epsilon_9$\DIFdelbegin \DIFdel{, as can be checked by taking into
account Tables}\DIFdelend \DIFaddbegin \DIFadd{.
}

\DIFadd{Comparing
Tables~}\DIFaddend \ref{table9} and \ref{table10}\DIFdelbegin \DIFdel{. More precisely, }\DIFdelend \DIFaddbegin \DIFadd{, we see that }\DIFaddend the
separation between the $\mathrm{CP}$-mixed \DIFdelbegin \DIFdel{hypotheses }\DIFdelend \DIFaddbegin \DIFadd{scenarios }\DIFaddend is enhanced by
about $3\sigma$ \DIFaddbegin \DIFadd{for $\mathcal{A}(\epsilon_{10})$ compared
  to $\mathcal{A}(\epsilon_{9})$}\DIFaddend .
    This improvement in the asymmetry \DIFdelbegin \DIFdel{of $\epsilon_{10}$
with respect to $\epsilon_9$ }\DIFdelend may be due to two facts. In the first
place, as was pointed out in Sec.~\ref{sec4} when comparing the TPs
$\epsilon_4$ and $\epsilon_8$, the asymmetry appears to be higher when
the spin vectors are used instead of the lepton momenta\DIFdelbegin \DIFdel{and we }\DIFdelend \DIFaddbegin \DIFadd{.  We }\DIFaddend see
from Eqs.~(\ref{eq33}) and (\ref{eq36}) that $\epsilon_{10}$, being
obtained from $\epsilon_4$, contains the information on the spin
vectors\DIFdelbegin \DIFdel{, as opposed to }\DIFdelend \DIFaddbegin \DIFadd{; by way of contrast, }\DIFaddend $\epsilon_9$ \DIFdelbegin \DIFdel{that }\DIFdelend depends directly on the
lepton momenta because it is derived from $\epsilon_8$. In the second
place, in order to obtain $\epsilon_{10}$\DIFaddbegin \DIFadd{, }\DIFaddend we have replaced
\DIFdelbegin \DIFdel{in the expression for $\epsilon_4$ }\DIFdelend the top and antitop momenta by their 
visible \DIFdelbegin \DIFdel{part}\DIFdelend \DIFaddbegin \DIFadd{parts}\DIFaddend , while in the case of $\epsilon_9$ the bottom and
antibottom momenta have been used.
\DIFdelbegin %DIFDELCMD < \begin{table}[H]
%DIFDELCMD < %%%
\DIFdelendFL \DIFaddbeginFL 

%DIF > 
\begin{table}[th]
\DIFaddendFL \caption{Asymmetry for the TP $\epsilon_{10}$ for the SM case and the
  two $\mathrm{CP}$-mixed cases defined by $\kp=1,\kpt=\pm 1$. The
  values are obtained by using $10^5$ simulated events.}
\label{table10}
\begin{center}
\begin{tabular}{|C{1cm}|C{1cm}||C{2cm}|C{2cm}|}
%\begin{tabular}{|c|r||r|c||r|c||r|c|}
\hhline{|====|}
%\hhline{|--------|}
$\kappa_t$&$\tilde{\kappa}_t$~~&$\mathcal{A}(\epsilon_{10})$~~&$\mathcal{A}(\epsilon_{10})/\sigma_{\mathcal{A}}$ \\ 
\hhline{|====|} 
%\hhline{|--------|}
$1$ & $-1$~~~ & $-0.0213$~~~ & $-6.7$~~~ \\[0.6mm]
\hline
$1$ & $0$ & $0.0031$~ & $1.0$~ \\[0.6mm]
\hline
$1$ & $1$ & $0.0300$~ & $9.5$~ \\[0.6mm]
\hhline{|====|}
%\hhline{|--------|}
\end{tabular}
\end{center} 
\end{table}
\DIFdelbegin %DIFDELCMD < \par
%DIFDELCMD < %%%
\DIFdelend %
\DIFdelbegin \DIFdel{By using our set of simulated events we have also tested, for
}\DIFdelend \DIFaddbegin 

\DIFadd{For }\DIFaddend comparison purposes, \DIFaddbegin \DIFadd{we have also used our simulated events
to test the asymmetry corresponding to }\DIFaddend the \DIFdelbegin \DIFdel{asymmetry of the }\DIFdelend lab frame observable given
in \DIFdelbegin \DIFdel{\mbox{%DIFAUXCMD
\cite{Guadagnoli}}%DIFAUXCMD
. We have obtained }\DIFdelend \DIFaddbegin \DIFadd{Ref.~\mbox{%DIFAUXCMD
\cite{Guadagnoli}}%DIFAUXCMD
. }{\bf \DIFadd{(KK: Is this lab frame observable
  just a TP, like we have used?  It seems strange to talk about an
  asymmetry associated with an observable, when normally an asymmetry
  is itself considered to be an observable.  I'm wondering if we are
  considering an asymmetry associated with a TP, or if we are actually
  just looking at whatever observable they have defined\ldots.  Sorry for
  my confusion on this!)}}  \DIFadd{We have found }\DIFaddend that this observable
\DIFaddbegin {\bf \DIFadd{(KK: ``asymmetry''?)}}
\DIFaddend appears to
be slightly less sensitive than the combination $\epsilon_{10}$,
\DIFaddbegin {\bf \DIFadd{(KK: ``$\mathcal{A}(\epsilon_{10})$''?)}}
\DIFaddend giving rise to a separation between the $\mathrm{CP}$-mixed \DIFdelbegin \DIFdel{hypotheses
}\DIFdelend \DIFaddbegin \DIFadd{scenarios
that is }\DIFaddend smaller by about $1.4\sigma$.
\section{Experimental Feasibility}
\label{sec6}
%
By considering the mild selection cuts introduced in Sec.~\ref{sec3},
the SM cross section for $\ppprocess$, $\ell=e,\mu$ at
$14\,\mathrm{TeV}$ is $\sim 15.3\,\mathrm{fb}$ and hence the number of
events expected within the context of the HL-LHC is $\sim
15.3\,\mathrm{fb} \times 3000\,\mathrm{fb}^{-1} = 4.59\times
10^4$. This number is expected to be even larger in the case of
$\kp=1,\kpt\neq 0$ since the corresponding cross section is then
higher than the SM one. By taking into account NLO corrections to the
production process and considering a $K$-factor around $1.2$
\cite{Dawson,Beenakker,Dittmaier}, the number of events can be raised
up to $\sim 5.49 \times 10^4$. On the other hand, additional cuts as
well as the efficiency related to the reconstruction of particles
momenta will contribute to decrease this number. For instance, in
order to obtain the asymmetry of $\epsilon_4$, the $t$ and $\tbar$
momenta need to be reconstructed. This is challenging not only due to
the presence of two neutrinos in the final state which escape the
detector undetected but also because final state objects need to be
asocciated with the respective parent quark \cite{atlasconf}. As was
already mentioned in Sec.~\ref{sec5}, a possibility is to use the
neutrino weighting technique along with the kinematic equations
arising from kinematic constraints related to the top and $W$ masses
as well as from the energy-momentum conservation at each of the decay
vertices involved in the process. Within the context of $t\tbar$ this
procedure has been used, for instance, in order to obtain measurements
of spin correlation \cite{atlasconf} and charge asymmetry
\cite{atlascharge}. Also, events reconstructed with this technique has
been used in \cite{dosSantos} for analyzing angular distributions that
are useful for discriminating the signal from the backgrounds in
$t\tbar H (H\rightarrow b\bbar)$. In all these cases the corresponding
efficiency in the reconstruction of the momenta is up to $\sim 80
\%$.\par Based on the discussion given in the previous paragraph, we
have simulated sets of $5\times 10^4, 1\times 10^4$ and $5\times 10^3$
events and recalculated the most sensitive observable, namely
$\mathcal{A}(\epsilon_4)$, for each case. The results are displayed in
Table \ref{table11}, where it can be seen that for a number of events
close to that roughly estimated above within the context of the
HL-LHC, the observable is still sensitive to $\kpt$ allowing to
separate the $\mathrm{CP}$-mixed cases by $19\sigma$. As expected, the
sensitivity worsen as the number of events is reduced, but even with
$5\times 10^3$ events the separation between the $\mathrm{CP}$-mixed
hypotheses under consideration is around $6.5\sigma$. \par Although
the combination $\epsilon_{10}$ discussed in Sec.~\ref{sec5} avoids
the problem of reconstructing the top and antitop momenta, we have
also considered the respective asymmetry obtained for more
conservative numbers of events. In Table \ref{table12} we show the
results for $\mathcal{A}(\epsilon_{10})$ when the number of events is
reduced from $5\times 10^4$ to $1\times 10^4$. We see in this case
that even with $1\times 10^4$ simulated events the observable is
capable to discriminate the $\mathrm{CP}$-mixed cases by
$5.6\sigma$.\par Finally, it is important to mention that a realistic
analysis of the sensistivity of the observables discussed in this
paper requires the study of the impact of the backgrounds as well as
the hadronization of the quarks in the final state and the effects of
the detector. If we consider the dominant decay mode of the higgs,
$H\rightarrow b\bbar$, in order to maximize the cross section of the
process, the signature is given by $4$ $b$-jets, two leptons and
missing energy, while the main background arise from the production of
$t\tbar$ in asocciation with additional jets, being the dominant
source the production of $t\tbar + b\bbar$. By applying a small set of
cuts it is shown in \cite{chinos} that the
\begin{table}[H]
\caption{Asymmetry for the TP $\epsilon_4$ obtained by using $5\times
  10^4, 1 \times 10^4$ and $5\times 10^3$ simulated events for the SM
  case and the two $\mathrm{CP}$-mixed cases defined by
  $\kp=1,\kpt=\pm 1$. }
\label{table11}
\begin{center}
\begin{tabular}{|C{1cm}|C{1cm}||C{2cm}|C{2cm}||C{2cm}|C{2cm}||C{2cm}|C{2cm}|}
%\begin{tabular}{|c|r||r|c||r|c||r|c|}
\hhline{|========|}
%\hhline{|--------|}
\multirow{2}{*}{$\kappa_t$} & \multirow{2}{*}{$\tilde{\kappa}_t$} & \multicolumn{2}{c||}{$N_{\mathrm{ev}}=5\times 10^4$} & \multicolumn{2}{c||}{$N_{\mathrm{ev}}=1\times 10^4$} & \multicolumn{2}{c|}{$N_{\mathrm{ev}}=5\times 10^3$} \\ \cline{3-8}
& & $\mathcal{A}(\epsilon_4)$~~&$\mathcal{A}(\epsilon_4)/\sigma_{\mathcal{A}}$ &  $\mathcal{A}(\epsilon_4)$~~&$\mathcal{A}(\epsilon_4)/\sigma_{\mathcal{A}}$ &  $\mathcal{A}(\epsilon_4)$~~&$\mathcal{A}(\epsilon_4)/\sigma_{\mathcal{A}}$ \\
\hhline{|========|} 
%\hhline{|--------|}
$1$ & $-1$~~~ & $-0.0405$~~~ & $-9.1$~~~ & $-0.0426$~~~ & $-4.3$~~~ & $-0.0496$~~~ & $-3.5$~~~ \\[0.6mm]
\hline
$1$ & $0$ & $0.0004$~ & $0.1$~ & $-0.0084$~~~ & $-0.8$~~~ & $-0.0004$~~~ & $-0.03$~~~ \\[0.6mm]
\hline
$1$ & $1$ & $0.0443$~ & $9.9$~ & $0.0434$~ & $4.2$~ & $\,0.0420$~ & $3.0\,\,$ \\
\hhline{|========|}
%\hhline{|--------|}
\end{tabular}
\end{center} 
\end{table}
\begin{table}[H]
\caption{Asymmetry for the TP $\epsilon_{10}$ in the SM case and the
  two $\mathrm{CP}$-mixed cases defined by $\kp=1,\kpt=\pm 1$ when the
  number of simulated events is reduced from $5\times 10^4$ to $1
  \times 10^4$.}
\label{table12}
\begin{center}
\begin{tabular}{|C{1cm}|C{1cm}||C{2cm}|C{2cm}||C{2cm}|C{2cm}|}
%\begin{tabular}{|c|r||r|c||r|c||r|c|}
\hhline{|======|}
%\hhline{|--------|}
\multirow{2}{*}{$\kappa_t$} & \multirow{2}{*}{$\tilde{\kappa}_t$} & \multicolumn{2}{c||}{$N_{\mathrm{ev}}=5\times 10^4$} & \multicolumn{2}{c|}{$N_{\mathrm{ev}}=1\times 10^4$} \\ \cline{3-6}
& & $\mathcal{A}(\epsilon_{10})$~~&$\mathcal{A}(\epsilon_{10})/\sigma_{\mathcal{A}}$ &  $\mathcal{A}(\epsilon_{10})$~~&$\mathcal{A}(\epsilon_{10})/\sigma_{\mathcal{A}}$ \\
\hhline{|======|} 
%\hhline{|--------|}
$1$ & $-1$~~~ & $-0.0270$~~~ & $-6.0$~~~ & $-0.0184$~~~ & $-1.8$~~~  \\[0.6mm]
\hline
$1$ & $0$ & $0.0022$~ & $0.5$~ & $-0.0086$~~~ & $-0.9$~~~  \\[0.6mm]
\hline
$1$ & $1$ & $0.0313$~ & $7.0$~ & $0.0380$~ & $3.8\,$ \\
\hhline{|======|}
%\hhline{|--------|}
\end{tabular}
\end{center} 
\end{table}
\noindent
%
signal to background ratio is largely improved. On the other hand, a
rigorous treatment of the backgrounds and their impact with respect to
the signal is performed in \cite{atlasger} by using $20.3\,
\mathrm{fb}^{-1}$ of data at $\sqrt{s}=8\,\mathrm{TeV}$.  \par The
results shown in tables \ref{table11} and \ref{table12} reveal that
with $5\times 10^3$ and $1\times 10^4$ events respectively the
observables $\mathcal{A}(\epsilon_4)$ and $\mathcal{A}(\epsilon_{10})$
are still useful for testing $\kpt$. Without considering the lack of
events related to the experimental analysis, these numbers of events
correspond to a luminosity around $\sim
300\,$-$\,600\,\mathrm{fb}^{-1}$ for the SM and even smaller for the
$\mathrm{CP}$-mixed cases due to the larger cross section. This range
of luminosities is in principle achievable in the short term by the
LHC. We note that in order to be fully conclusive about the required
luminosity, it is important to include the effects of hadronization,
detector resolution, reconstruction eficciencies and so
forth. However, this kind of analysis is out of the scope of this
paper.
%%%%%%%%%%%% with Cuts
%51.46/17 -0.02 +- 0.08
%45.48/17  -1.09 +- 0.07
%65.60/17   0.96 +- 0.08
%28.84/17   0.02 +- 0.07
%%%%%%%
%9.13/17     0.01 +- 0.06
%9.05/17     -0.98 +- 0.07
%19.7/17     0.80 +- 0.07
%16.59/17    0.06 +- 0.07
%
%%%%%%%%%%%% without Cuts
%16.36/17    0.0022 +-0.06
%15.25/17    -0.82 +- 0.07
%12.80/17    0.81 +- 0.07
%19.38/17    0.11 +- 0.08
%%%%%%
%7.45/17    0.005 +- 0.06
%18.17/17   -0.77 +- 0.07
%9.59/17    0.73 +- 0.06
%12.39/17   0.09 +- 0.08
%%%%%%%%%%%%%%%%%%%%%%%%%%%%%%%%%%%%%%%%%%%%%%%%%
\section{Conclusions}
\label{sec7}
%
In this paper we have presented a collection of $\mathrm{CP}$-odd
observables based on triple product correlations that are useful for
disentangling the relative sign between the scalar ($\kp$) and a
potential pseudoscalar ($\kpt$) top-Higgs couplings in the $t\tbar H$
production with both tops decaying leptonically. We have tested the
sensitivity of the proposed observables by considering three types of
observables: asymmetries, mean values and angular distributions.\par

Through the use of spinor techniques we have written the expression
for the differential cross section of the full process in such a
manner that the production and the decay parts are separated, although
connected by the spin vectors of the top and antitop which are given
in terms of the momenta of the leptons in the final state. Moreover,
we have indentified the terms linear in $\kp$ and $\kpt$ as those
involving TPs. Among these, we have explored the three that do not
involve the momenta of the incoming quarks/gluons and at the same time
incorporate both spin vectors: $\epsilon_1\equiv \TPa$,
$\epsilon_2\equiv \TPb$ and $\epsilon_3\equiv \TPc$.\par We have found
that $\epsilon_{1,2,3}$ allow to distinguish between the
$\mathrm{CP}$-mixed hypotheses by more than $\sim 20\sigma$ in the
case of asymmetries and $\sim 10\sigma$ in the case of mean values
when $1 \times 10^5$ simulated events are used. Furthermore, we have
shown that the angular distributions asocciated with these TPs are
also sensitive to the value of $\kp$ and $\kpt$ exhibiting a phase
shift that varies according to the values taken by these couplings. On
the other hand, by exploring TPs that incorporate the momenta of the
Higgs and the leptons instead of the spin vectors we can conclude that
the observables studied here appear to be more sensitive when the spin
vectors are used.\par

On the other hand, we have proposed a combination of the TPs
considered in the first place, $\epsilon_4\equiv
\epsilon_3-\epsilon_2$, that exhibits the highest sensitivity with the
separation between the $\mathrm{CP}$-mixed hypotheses being increased
by at least $3\sigma$ in the asymmetry for $1 \times 10^5$ events with
respect to $\mathcal{A}(\epsilon_{1,2,3})$. Again, when a similar
combination is constructed by using the leptons momenta instead of the
spin vectors ($\epsilon_8$), the sensitivity in the asymmetry is
decreased by $\sim 3\sigma$ compared to $\mathcal{A}(\epsilon_4)$ for
the same number of events, giving values compatible with those
obtained for the asymmetry of $\epsilon_2$ and $\epsilon_3$. \par

Taking into account the challenge of reconstructing the top and
antitop momenta due to the presence of two neutrinos in the final
state, we have proposed and tested two TP correlations that avoid this
difficulty. The first one is obtained by replacing the $t$ and $\tbar$
by the $b$ and $\bbar$ momenta ($\epsilon_9$), whereas the second
includes the visible part of the $t$ and $\tbar$ momenta
($\epsilon_{10}$). We have encountered that the latter is the most
sensitive leading to a discrimination of the $\mathrm{CP}$-mixed cases
under analysis by up to $\sim 16\sigma$.\par

Finally we have discussed the feasibility of the most sensitive
observables proposed here. We have found that with $5\times 10^3$ and
$1\times 10^4$ events respectively the observables
$\mathcal{A}(\epsilon_4)$ and $\mathcal{A}(\epsilon_{10})$ are still
useful for testing the hypotheses $(\kp=1,\kpt=\pm 1)$ giving rise to
separations of order $\sim 6\sigma$. These required number of events
would be reachable in principle in the short term by the LHC and hence
the capability of the observables studied here for testing the sign of
$\kpt/\kp$ could be probed in that context.

%%%%%%%%%%%%%%%%%%%%%%%%%%%%%%%%%%%%%%%%%%%%%%%%%
\bigskip
\noindent
{\bf Acknowledgments}
\noindent 
This work has been partially supported by ANPCyT under grants No. PICT
2013-0433 and No. PICT 2013-2266, and by CONICET (NM, AS). The work of
KK was supported by the U.S.  National Science Foundation under Grant
PHY-1215785. KK also acknowledges sabbatical support from Taylor
University.
%%%%%%%%%%%%%%%%%%%%%%%%%%%%%%%%%%%%%%%%%%%%%%%%%
\section*{\refname}
\let\bibsection\relax

\setlength{\bibsep}{10pt}
\bibliography{PaperDraftbiblio}
\bibliographystyle{apsrev4-1}

\end{document}
