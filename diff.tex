\documentclass[aps,preprint,tightenlines,floatfix,superscriptaddress,nofootinbib,showpacs]{revtex4-1}
%DIF LATEXDIFF DIFFERENCE FILE
%DIF DEL ../manuscript_older/Paper_ttbarH_Draft_12_20_15.tex   Wed Jan 20 14:03:07 2016
%DIF ADD Paper_ttbarH_Draft.tex                                Wed Feb 10 14:22:29 2016
\pdfoutput=1
\usepackage{graphicx}
%\usepackage{dcolumn}
\usepackage{amsfonts}
\usepackage{hhline}
\usepackage{multirow}
\usepackage{array}
%\usepackage{longtable}
\usepackage{amsmath}
\usepackage{amssymb}
\usepackage{maybemath}
%\usepackage[italic]{hepparticles}
\usepackage{float}
\usepackage[lofdepth,lotdepth]{subfig}
\usepackage[countmax]{subfloat}
\usepackage{cancel}
%\usepackage[nodisplayskipstretch]{setspace}
%\setstretch{1.5}
%\usepackage{caption}
%\usepackage{subcaption}
\setlength{\oddsidemargin}{-1in}
\addtolength{\oddsidemargin}{24mm} \setlength{\textwidth}{170mm}
\setlength{\topmargin}{-0.55in} \setlength{\headheight}{10mm}
\setlength{\headsep}{0mm} \setlength{\textheight}{230mm}
\def\beq{\begin{equation}}
\def\eeq{\end{equation}}
\def\bea{\begin{eqnarray}}
\def\eea{\end{eqnarray}}
\def\nn{\nonumber}
\def\sss{\scriptstyle}
\def\lft{{\scriptstyle L}}
\def\rht{{\scriptstyle R}}
\def\roughly#1{\mathrel{\raise.3ex\hbox
{$#1$\kern-.75em\lower1ex\hbox{$\sim$}}}}
\def\lsim{\roughly<}
\def\gsim{\roughly>}
\def\tbslash{\tbar\hspace{-10pt}\not{}\hspace{4pt}}
\def\tslash{t\hspace{-10pt}\not{}\hspace{4pt}}
\def\lpslash{l^+\hspace{-10pt}\not{}\hspace{4pt}}
\def\lmslash{l^-\hspace{-10pt}\not{}\hspace{4pt}}
\def\nslash{n\hspace{-10pt}\not{}\hspace{4pt}}
\def\ntbslash{n_{\tbar}\hspace{-10pt}\not{}\hspace{4pt}}
%%%%%%%%%%%%%%%%%%%%%%%%%%%%%%%%%%%%%%%%%%%%%%%%%%%%%%%%%%%%%%%%%%%%%%%%%%%%%%%
\newcommand{\note}[1]{\marginpar{{\small\begin{center}{\it #1}\end{center}}}}
%%%%%%%%%%%%%%%%%%%%%%%%%%%%%%%%%%%%%%%%%%%%%%%%%%%%%%%%%%%%%%%%%%%%%%%%%%%%%%%
\def\tbar{\bar{t}}
%\def\tbar{{\overline{t}}}
\def\bbar{\bar{b}}
\def\qbar{\bar{q}}
\def\nubar{{\bar{\nu}}_{\ell}}
\def\tbbc{t \to b \bbar c}
\def\tauprocess{\tau\rightarrow K\pi\pi\nu_{\tau}}
\def\ppprocess{pp\to t\,\left(\rightarrow b {\ell}^+ \nu_{\ell}\right) \tbar\,\left(\rightarrow\bbar {\ell}^-\nubar\right)\,H}
\def\bppprocess{{\boldmath $pp\to t \tbar H\to \left(b {\ell}^+ \nu_{\ell}\right)\left(\bbar {\ell}^-\nubar\right)H$}}
\def\ggprocess{gg\to t \tbar H\to\left(b l^+ \nu_l\right)\left(\bbar l^-\nubar\right)H}
\def\qqprocess{q\qbar\to t \tbar H\to\left(b l^+ \nu_l\right)\left(\bbar l^-\nubar\right)H}
\def\kp{\kappa_t}
\def\kpt{\tilde{\kappa}_t}
\def\MG{\mbox{MadGraph 5}}
\def\tm#1{\texttt{TM-{#1}}}
\def\ex#1{\texttt{EX-{#1}}}
\def\Ahat{\hat A_i^{\sigma}}
\def\madg{MadGraph~5}
\def\TPa{\epsilon(t,\tbar,n_t,n_{\tbar})}
\def\TPb{\epsilon(Q,\tbar,n_t,n_{\tbar})}
\def\TPc{\epsilon(Q,t,n_t,n_{\tbar})}
%%%%%%%%%%%%%%%%%%%%%%%%%%%%%%%%%%%%%%%%%%%%%%%%%%%%%%%%%%%%%%%%%%%%%%%%
%%%%%
%for editing purposes
\usepackage[normalem]{ulem}
\usepackage{color}
\definecolor{BrickRed}{cmyk}{0,0.89,0.94,0.28}
\definecolor{DarkGreen}{cmyk}{1,0,1,0.5}
\definecolor{Blue}{cmyk}{1,1,0,0}
\definecolor{BurntOrange}{cmyk}{0,0.51,1,0}

\def\hldl#1{\textcolor{BrickRed}{\textsf{#1}}}
\def\hlps#1{\textcolor{DarkGreen}{\textsf{#1}}}
\def\hlkk#1{\textcolor{Blue}{\textsf{#1}}}
\def\hlas#1{\textcolor{BurntOrange}{\textsf{#1}}}
\def\soutdl{\bgroup\markoverwith{\textcolor{BrickRed}{\rule[0.5ex]{2pt}{0.4pt}}}\ULon}
\def\soutps{\bgroup\markoverwith{\textcolor{DarkGreen}{\rule[0.5ex]{2pt}{0.4pt}}}\ULon}
\def\soutkk{\bgroup\markoverwith{\textcolor{Blue}{\rule[0.5ex]{2pt}{0.4pt}}}\ULon}
\def\soutas{\bgroup\markoverwith{\textcolor{BurntOrange}{\rule[0.5ex]{2pt}{0.4pt}}}\ULon}

\def\mhldl#1{\textcolor{BrickRed}{\ensuremath{#1}}}
\def\mhlps#1{\textcolor{DarkGreen}{\ensuremath{#1}}}
\def\mhlkk#1{\textcolor{Blue}{\ensuremath{#1}}}
\def\mhlas#1{\textcolor{BurntOrange}{\ensuremath{#1}}}
\def\msoutdl#1{\text{\soutps{\ensuremath{#1}}}}
\def\msoutps#1{\text{\soutps{\ensuremath{#1}}}}
\def\msoutkk#1{\text{\soutps{\ensuremath{#1}}}}
\def\msoutas#1{\text{\soutps{\ensuremath{#1}}}}
%%%%%%%%%%%%%%%%%%%%%%%%%%%%%%%%%%%%%%%%%%%%%%%%%%%%%%%%%%%%%%%%%%%%%%%%
%DIF PREAMBLE EXTENSION ADDED BY LATEXDIFF
%DIF UNDERLINE PREAMBLE %DIF PREAMBLE
\RequirePackage[normalem]{ulem} %DIF PREAMBLE
\RequirePackage{color}\definecolor{RED}{rgb}{1,0,0}\definecolor{BLUE}{rgb}{0,0,1} %DIF PREAMBLE
\providecommand{\DIFadd}[1]{{\protect\color{blue}\uwave{#1}}} %DIF PREAMBLE
\providecommand{\DIFdel}[1]{{\protect\color{red}\sout{#1}}}                      %DIF PREAMBLE
%DIF SAFE PREAMBLE %DIF PREAMBLE
\providecommand{\DIFaddbegin}{} %DIF PREAMBLE
\providecommand{\DIFaddend}{} %DIF PREAMBLE
\providecommand{\DIFdelbegin}{} %DIF PREAMBLE
\providecommand{\DIFdelend}{} %DIF PREAMBLE
%DIF FLOATSAFE PREAMBLE %DIF PREAMBLE
\providecommand{\DIFaddFL}[1]{\DIFadd{#1}} %DIF PREAMBLE
\providecommand{\DIFdelFL}[1]{\DIFdel{#1}} %DIF PREAMBLE
\providecommand{\DIFaddbeginFL}{} %DIF PREAMBLE
\providecommand{\DIFaddendFL}{} %DIF PREAMBLE
\providecommand{\DIFdelbeginFL}{} %DIF PREAMBLE
\providecommand{\DIFdelendFL}{} %DIF PREAMBLE
%DIF END PREAMBLE EXTENSION ADDED BY LATEXDIFF

\begin{document}
\vspace*{2cm}

\title{{\boldmath Pseudoscalar top-Higgs coupling:  Exploration of $\mathrm{CP}$-odd observables to 
resolve the sign ambiguity
}}

\def\tayloru{\affiliation{\it Physics and Engineering Department,
    Taylor University, \\ 236 West Reade Ave., Upland, IN 46989, USA \vspace*{1mm}}}

%\def\tayloru{\affiliation{\it Physics and Engineering Department,
%    Taylor University, \\ 236 West Reade Ave., Upland, IN 46989, USA \vspace*{8mm}}}

\def\laplata{\affiliation{\it IFLP, CONICET -- Dpto. de F\'{\i}sica,
    Universidad Nacional de La Plata, C.C. 67, 1900 La Plata,
    Argentina \vspace*{8mm}}}

\author{Nicolas Mileo}
\email{mileo@fisica.unlp.edu.ar}
\laplata

\author{Ken Kiers}
\email{knkiers@taylor.edu}
\tayloru

\author{Alejandro Szynkman}
\email{szynkman@fisica.unlp.edu.ar}
\laplata

\author{Daniel Crane \vspace*{4mm}}
\DIFdelbegin %DIFDELCMD < \email{\hlkk{dkcrane@mtu.edu}}
%DIFDELCMD < %%%
\DIFdelend \DIFaddbegin \email{dkcrane@mtu.edu}
\DIFaddend \tayloru

\author{Ethan Gegner}
\DIFdelbegin %DIFDELCMD < \email{ethan_gegner@taylor.edu}
%DIFDELCMD < %%%
\DIFdelend \DIFaddbegin \email{ethan\_gegner@taylor.edu}
\DIFaddend \tayloru

\date{\vspace*{2mm}\today \\\bigskip\bigskip}

\begin{abstract}
\vspace*{4mm} We present a collection of $\mathrm{CP}$-odd observables
for the process $\ppprocess$ that are linearly dependent on the scalar
($\kp$) and pseudoscalar ($\kpt$) top-Higgs coupling and hence
sensitive to the corresponding relative sign. The proposed observables
are based on triple \DIFdelbegin \DIFdel{products }\DIFdelend \DIFaddbegin \DIFadd{product }\DIFaddend (TP) structures that we extract
from the expression of the differential cross section in terms of the
spin vectors of the top and antitop quarks. In order to explore other
possibilities, we progressively modify these TPs, \DIFdelbegin \DIFdel{firstly, }\DIFdelend \DIFaddbegin \DIFadd{first
}\DIFaddend by combining them\DIFaddbegin \DIFadd{, }\DIFaddend and then by replacing the spin vectors by the
lepton momenta or the $t$ and $\tbar$ momenta by their visible parts.
\DIFdelbegin \DIFdel{We
obtain that the }\DIFdelend \DIFaddbegin \DIFadd{Assuming an integrated luminosity that is consistent with that
  envisioned for the HL-LHC, we find that the }\DIFaddend most
promising observable can disentangle the hypotheses $\kp=1,\kpt=\pm 1$
by more than $\sim 20\sigma$\DIFdelbegin \DIFdel{for $10^5$
events that is within the expected order of magnitude of events at the
HL-LHC}\DIFdelend .  In
the case of observables that do not need the reconstruction of the $t$
and $\tbar$ momenta, the power of discrimination is up to $\sim
16\sigma$ for the same \DIFdelbegin \DIFdel{ammount }\DIFdelend \DIFaddbegin \DIFadd{number }\DIFaddend of events. We also show that the
capability of the most promising observables for separating the
$\mathrm{CP}$-mixed hypotheses preveals even when a number of events
plausible within the short term LHC is considered. \DIFaddbegin {\bf \DIFadd{(KK: reword last sentence a bit.)}}
\DIFaddend 

%We also show that the sensitivity of the most promising observables is not spoiled when a number of events affordable by the short term LHC is considered.

%We also show that even when the number of events is reduced by an order of magnitude the most promising observables are still useful for testing the couplings
\end{abstract}
\maketitle
%%%%%%%%%%%%%%%%%%%%%%%%%%%%%%%%%%%%%%%%%%%%%%%%%
\section{Introduction}
\label{sec1}
After the discovery of a new boson $H$ by the ATLAS \cite{atlasH} and
CMS \cite{cmsH} collaborations, it has become of crucial importance
 \DIFdelbegin \DIFdel{its characterization, namely the determination of its }\DIFdelend \DIFaddbegin \DIFadd{to determine its }\DIFaddend physical
 properties with the highest possible precision.
 \DIFdelbegin \DIFdel{Among the features }\DIFdelend \DIFaddbegin \DIFadd{The study }\DIFaddend of the new boson\DIFdelbegin \DIFdel{that are under investigation, the study of its couplings
 to the }\DIFdelend \DIFaddbegin \DIFadd{'s couplings
 to }\DIFaddend fermions is of great relevance
 \DIFdelbegin \DIFdel{, allowing us to study the nature
of $H$ under the discrete symmetry }\DIFdelend \DIFaddbegin \DIFadd{and will allow us to better understand this particle's }\DIFaddend $\mathrm{CP}$\DIFdelbegin \DIFdel{and to clarify }\DIFdelend \DIFaddbegin \DIFadd{-transformation
   properties, as well as }\DIFaddend the
extent to which this \DIFdelbegin \DIFdel{new }\DIFdelend particle is consistent \DIFdelbegin \DIFdel{to }\DIFdelend \DIFaddbegin \DIFadd{with }\DIFaddend the Higgs boson
predicted by the Standard Model (SM) of particle physics.
\DIFdelbegin \DIFdel{In
particular, the test of the Higgs coupling to
the heaviest fermion,
the top quark, is a priority since on the one hand, it rules the main production mechanism through gluon fusionand }\DIFdelend \DIFaddbegin \DIFadd{It is particularly important to test the coupling of the putative (}{\bf \DIFadd{(KK: at some point
  should we just start calling this the Higgs?  How do we make that transition?)}}
\DIFadd{Higgs boson to
the top quark.  This coupling governs
the main Higgs boson production mechanism (which proceeds via gluon fusion)
and it }\DIFaddend contributes to the
important \DIFaddbegin \DIFadd{Higgs boson }\DIFaddend decay mode to two photons\DIFdelbegin \DIFdel{, and on the other hand, it is }\DIFdelend \DIFaddbegin \DIFadd{.
It is also
}\DIFaddend involved in the scalar-field naturalness problem \DIFaddbegin \DIFadd{-- }\DIFaddend giving rise to the
leading dependence on the cut-off energy scale in the corrections to
the Higgs mass \DIFaddbegin \DIFadd{-- }\DIFaddend and it may play an important role in the EWSB \DIFdelbegin \DIFdel{mechanism.}\DIFdelend \DIFaddbegin {\bf \DIFadd{KK: define EWSB?}}
\DIFadd{mechanism.  }{\bf \DIFadd{(KK: I reworded a lot of this paragraph -- is it still
  correct, or did I accidentally alter the meaning?)}}\DIFaddend \par

Given \DIFdelbegin \DIFdel{the fact }\DIFdelend that the main Higgs boson production
process \DIFdelbegin \DIFdel{, the gluon fusion, }\DIFdelend is dominated by a top quark loop and that
the diphoton and digluon decay channels are also mediated by a top
loop, these processes provide constraints on the scalar and
pseudoscalar $tH$ couplings, $\kp$ and $\kpt$
\cite{constraints1,constraints2, constraints3,constraints4}.
     \DIFaddbegin {\bf \DIFadd{(KK: OK to drop ``gluon fusion'' here, since we already said it was
       the main production process above?  Alternatively, we could say it here
       and skip it above.)}}
     \DIFaddend However,
these constraints assume that there are no other sources contributing
to the corresponding effective couplings\DIFdelbegin \DIFdel{and}\DIFdelend \DIFaddbegin \DIFadd{; furthermore}\DIFaddend , in the case of the
diphoton decay channel \DIFaddbegin \DIFadd{(}\DIFaddend which also involves a $W$ boson loop\DIFdelbegin \DIFdel{, }\DIFdelend \DIFaddbegin \DIFadd{), it is also
assumed }\DIFaddend that the
coupling of the Higgs boson to the $W$ is standard. In this sense, the
constraints derived from measurements of Higgs rates are indirect
constraints. \DIFdelbegin \DIFdel{In this direction, electric }\DIFdelend \DIFaddbegin \DIFadd{Electric }\DIFaddend dipole moments (EDM) can also
impose stringent indirect constraints on $\kpt$ by assuming that there
are no new physics (NP) particles contributing to the loops of the
relevant diagrams and that the electron-Higgs coupling is that
predicted by the SM \cite{constraints1,edm}. In order to probe
\DIFdelbegin \DIFdel{directly }\DIFdelend the $tH$ coupling \DIFaddbegin \DIFadd{directly}\DIFaddend , processes with smaller cross sections need
to be taken into account.  \par

In contrast to the $\tau H$ coupling which can be studied trough the
decay $H\rightarrow \tau^+\tau^-$ \cite{tau}, the $tH$ coupling is
tested directly by making use of the production processes since the
phase space forbids the Higgs to decay to the pair $t\tbar$. The
production of the Higgs boson in association with the pair $t\tbar$
and the production in association with a single top or antitop are the
only two direct ways of probing the $tH$ coupling. The cross section
of the single top (antitop) production is smaller and it is given by
the interference between a diagram in which the Higgs is radiated from
the top (antitop) leg and one with the Higgs emmited from the
intermediate virtual $W$ boson. This then implies that the contraints
over $\kp$ and $\kpt$ derived from $tH$ and $\tbar H$ production are
dependent on the asummption made on the coupling of the Higgs to the
gauge boson $W$, $\kappa_W$. However, it is important to note that the
interference between the above mentioned diagrams present in $tH$
($\tbar H$) production can be exploited in order to determine the
relative sign between $\kp$ and $\kappa_W$ (see for example
\cite{tHmaltoni}). From the point of view of $t\tbar H$ production,
several observables sensitive to the couplings $\kp$ and $\kpt$ has
been proposed. Examples of such observables are the cross section, the
invariant mass distributions, the transverse Higgs momentum
distribution or the azymuthal-angle separation between $t$ and $\tbar$
just to name a few \citep{Guadagnoli}. An approach based on weighted
moments and optimal observables has been developed in \cite
{Gunion1,*Gunion2,*Gunion3,*Gunion} to discriminate the hypothesis of
a $\mathrm{CP}$-even Higgs from that of a $\mathrm{CP}$-mixed state
within the context of a $e^+ e^-$ as well as a $pp$ collider. All the
mentioned observables are $\mathrm{CP}$-even and hence are not
sensitive to relative sign between the scalar and pseudoscalar
couplings $\kp$ and $\kpt$. Such observables are quadratically
dependent on these couplings and then only provide an indirect measure
of $\mathrm{CP}$ violation. In order to be sensitive to the relative
sign between $\kp$ and $\kpt$, $\mathrm{CP}$-odd observables must be
considered. \par

Since the quark top decays before it can hadronize, its spin
information is passed on to the angular distributions of the decay
products in such a way that these particles work as spin analyzers
being the lepton the most powerful. On the other hand, it has been
shown that in $t\tbar$ production the top quark and antiquark spins
are strongly correlated \cite{Mahlon1,*Mahlon2,*Mahlon3,*atwood} and
this manifests in the double angular distributions of the decay
products of the $t$ and $\tbar$ systems. These correlations are
dependent in turn on the production mechanism and in the case of
$t\tbar H$ associated production are sensitive to the manner in which
the top couples to the Higgs boson. In fact, observables that exploit
the differences on the $t\tbar$ spin configurations are used in
\cite{Biswas} to improve the discrimination of the $t\tbar H$ signal
from the dominant irreducible background $t\tbar b\bbar$, in which the
Higgs boson is not involved. \par

In this paper, we define a set of observables linearly dependent on
$\kp$ and $\kpt$ and hence sensitive to the relative sign of these
couplings. The proposed observables are based on a particular set of
triple product structures (TP) that we extract naturally from the
expression of the differential cross section, which makes use in turn
of the fact that the kinematics of the decay products contain the
information on the $t$ and $\tbar$ spins and thus is sensitive to the
nature of the $tH$ coupling, just as was discussed in the previous
paragraph. By using spinor thechniques we simply relate the top quark
and antiquark spin vectors and the corresponding final state particle
momenta and separate the production process from the decay which
allows to identify straightforwardly the contributions linearly
sensitive to the couplings. Further, the TPs correlations in these
contributions incorporate the $t$ and $\tbar$ spin vectors. Starting
with these TPs, we not only recover the observables given in
\cite{Ellis,Guadagnoli} but also propose additional possibilities that
increase the sensitivity. In order to establish a hierarchy in the
sensitivity of the TPs under analysis we investigate three different
types of observables by using simulated events: asymmetries, mean
values and angular distributions. We note that TP correlations has
been used in \cite{Valencia1,*Valencia2} in the context of top-quark
production and decay and in \cite{Valencia3} in the framework of
anomalous color dipole operators.  \par

The remainder of this paper is organized as follows. In
Sec.~\ref{sec2} we study the theoretical framework for the process
$\ppprocess$ and derive a general expression for the differential
cross section from which a first set of TP correlations is
extracted. In Sec.~\ref{sec3} we probe the sensitivity of these TPs to
the $tH$ coupling by using various $\mathrm{CP}$-odd
observables. Subsequent sections are dedicated to explore another
possibilities of $\mathrm{CP}$-odd observables. In particular,
observables based on TPs that incorporate the Higgs momentum are
discussed in Sec.~\ref{sec4}, whereas observables obtained without
using the $t$ and $\tbar$ momenta are studied in
Sec.~\ref{sec5}. Finally, Sec.~\ref{sec6} is devoted to the discussion
on the experimental feasibility of the most promising observables
encountered here. The main conclusions are summarized in
Sec.~\ref{sec7}.

%%%%%%%%%%%%%%%%%%%%%%%%%%%%%%%%%%%%%%%%%%%%%%%%%
%\newpage
%%%%%%%%%%%%%%%% Quitarlo %%%%%%%%%%%%%%%%%%%%%%%
\setlength{\abovedisplayskip}{10.6pt}
\setlength{\belowdisplayskip}{10.6pt}
%%%%%%%%%%%%%%%%%%%%%%%%%%%%%%%%%%%%%%%%%%%%%%%%%
%\section{Process \MakeLowercase{{\boldmath $pp\to t \bar{t}$}}$H\to$\MakeLowercase{{\boldmath $\left(b l^+ \nu_l \right) \left(\bar{b} l^-\bar{\nu}_l\right)$}}$H$. Theoretical framework}
\section{Process \MakeLowercase{{\boldmath $pp\to t(\to b {\ell}^+ \nu_{\ell})\,\bar{t}(\to\bar{b} {\ell}^-\bar{\nu}_{\ell})$}}$\,H$. Theoretical framework}
\label{sec2}

%
At the LHC the $t\bar{t}H$ production proceeds via $q\bar{q}$
annihilation and $gg$ fusion processes. The relevant Feynman diagrams
at leading order are displayed in Fig.~\ref{fig1}, where the two first
rows correspond respectively to $q\bar{q}$ and $gg$ s-channels whereas
the last one depicts the $gg$ t-channels. By exchanging the gluon
lines in this row, three more $gg$-initiated diagrams can be
obtained. We describe the $tH$ cupling with the effective lagrangian
%
\beq
\label{eq1}
\mathcal{L}_{t\bar{t}H}=-\frac{m_t}{v}(\kp \tbar t+i\kpt \tbar
\gamma_5 t)H,
\eeq
%
where $v=246~\mathrm{GeV}$ is the SM Higgs vacuum
expectation value, and the coefficients $\kp$ and $\kpt$ determine the
weigth of the scalar and pseudoscalar interaction, respectively. The
SM case is obtained for $\kp=1$ and $\kpt=0$, while the values $\kp=0$
and $\kpt\neq 0$ parameterizes a $\mathrm{CP}$-odd Higgs. \par
%%%%%%%%%%%%%%%%%%%% Figura 1%%%%%%%%%%%%%%%%%%%%%%%%%%%%%
%/home/nico/jaxodraw-2.1-0/
\begin{center}
\begin{figure}[H]
\centering
%\hspace*{-0.4cm}
\subfloat{\includegraphics[scale=0.45]{qqs1_II.pdf}}
\hspace*{0.05\textwidth}
%\label{fig1a}}
\subfloat{\includegraphics[scale=0.45]{qqs2_II.pdf}}
%\label{fig1b}}
\\[0.032\textwidth]
\subfloat{\includegraphics[scale=0.45]{ggs1_II.pdf}}
\hspace*{0.05\textwidth}
%\label{fig1b}}
\subfloat{\includegraphics[scale=0.45]{ggs2_II.pdf}}
%\label{fig1c}}
\\[0.032\textwidth]
\subfloat{\includegraphics[scale=0.45]{ggt1_II.pdf}}
%\label{fig1b}}
\hspace*{0.025\textwidth}
\subfloat{\includegraphics[scale=0.45]{ggt2_II.pdf}}
%\label{fig1c}}
\hspace*{0.025\textwidth}
\subfloat{\includegraphics[scale=0.45]{ggt3_II.pdf}}
%\label{fig1b}}
\vspace*{0.02\textwidth}
\caption{Tree level Feynman diagrams contributing to $t\tbar H$ production at the LHC. Three more diagrams are obtained by exchanging the gluon lines in the t-channel diagrams.}
\label{fig1}
\end{figure}
\end{center}
%%%%%%%%%%%%%%%%%%%%%%%%%%%%%%%%%%%%%%%%%%%%%%%%%%%%%%%%%%%%
Since the total cross-section is dominated by the $gg$ contribution,
let us focus on the $gg$-initiated production followed by the leptonic
SM decay of both $t$ and $\tbar$. By using the narrow-width
approximation for the top quark, it can be shown that the unpolarized
differential cross section for the full process $gg\to t(\to
b{\ell}^+\nu_{\ell})\,\tbar(\to \bbar {\ell}^- \nubar)\,H$ takes the
following form (we sketch the proof below)
%
\beq
\label{eq2}
%dd\sigma(gg\to (bl^+\nu_l)(\bbar l^- \nubar) H)=\sum_{\substack{bl^+\nu_l \\ \tiny{\mathrm{spins}}}}\,\sum_{\substack{\bbar l^-\nubar \\ \mathrm{\tiny{spins}}}}\left(\frac{2}{\Gamma_t}\right)^2\,d\sigma(gg\to t(n_t)\tbar (n_{\tbar})H)\,d\Gamma(t(n_t)\to bl^+\nu_l)\,d\Gamma(\tbar (n_{\tbar})\to \bbar l^-\nubar)
d\sigma =\sum_{\substack{b{\ell}^+\nu_l \\ \tiny{\mathrm{spins}}}}\,\sum_{\substack{\bbar {\ell}^-\nubar \\ \mathrm{\tiny{spins}}}}\left(\frac{2}{\Gamma_t}\right)^2\,d\sigma(gg\to t(n_t)\tbar (n_{\tbar})H)\,d\Gamma(t(n_t)\to b{\ell}^+\nu_{\ell})\,d\Gamma(\tbar (n_{\tbar})\to \bbar {\ell}^-\nubar),
\eeq  
%
where $d\sigma(gg\to t(n_t)\tbar (n_{\tbar})H)$ is the differential
cross section of the production of a top quark with spin vector $n_t$
plus an anti-top quark with spin vector $n_{\bar{t}}$ and a Higgs
boson, whereas $d\Gamma(t(n_t)\to b{\ell}^+\nu_{\ell})$ and
$d\Gamma(\tbar (n_{\tbar})\to \bbar {\ell}^-\nubar)$ are the
differential decay widths of a top with spin vector $n_t$ and an
anti-top with spin vector $n_{\tbar}$, respectively. Also, the vectors
$n_t$ and $n_{\tbar}$ are given by particular combinations of the
momenta $t,\ell^+$ and $\tbar,\ell^-$ respectively \cite{Arens},
%
\bea
\label{eq3}
n_t&=&-\frac{p_t}{m_t}+\frac{m_t}{(p_t\cdot p_{{\ell}^+})}p_{{\ell}^+}\\
\label{eq4}
n_{\tbar}&=&\,\frac{p_{\tbar}}{m_t}-\frac{m_t}{(p_{\tbar}\cdot p_{{\ell}^-})}p_{{\ell}^-}.
\eea
%
Expressions similar to Eq.~(\ref{eq2}) have been derived for the
production of short-lived particles in $e^-e^+$ colliders
\cite{kawasaki} and in particular for $t\tbar$ production both in
$e^-e^+$ colliders \cite{Arens} and $pp$ colliders
\cite{ale1,*ale2,*ale3,*ale4}. \par We will outline now how to derive
Eq.~(\ref{eq2}) and to obtain the above definitions for the spin
vectors. The process $g_ag_b\to t(\to
b_i{\ell}^+\nu_{\ell})\,\tbar(\to \bbar_j {\ell}^- \nubar)\,H$ can be
drawn as in Fig.~\ref{fig2}, where $a$ and $b$ denote the gluons and
$i$ and $j$ are the top and anti-top colours.
%%%%%%%%%%%%%%%%% FIGURA 2 %%%%%%%%%%%%%%%%%%%%%%
\begin{center}
\begin{figure}[H]
\centering
\includegraphics[scale=0.6]{esquematico_II.pdf}
\vspace*{0.02\textwidth}
\caption{Schematic representation of the process $g_ag_b\to t(\to
  b_i{\ell}^+\nu_{\ell})\tbar(\to \bbar_j {\ell}^- \nubar) H$. The
  indices $i,j$ denote the colour of the quarks while $a,b$ are gluon
  indices.}
\label{fig2}
\end{figure}
\end{center}
%%%%%%%%%%%%%%%%%%%%%%%%%%%%%%%%%%%%%%%%%%%%%%%%%
\par
%
The diagrams displayed in Fig.~\ref{fig1} (second and third rows)
contribute to the quantity $A^{ab,ij}$, so that
%
\beq
\label{eq5}
\mathcal{A}^{ab,ij}\equiv A^{ab,ij}_{\mu\nu}(\epsilon_{\lambda_a})^{\mu}(\epsilon_{\lambda_b})^{\nu}=\sum_{k=1}^8 \mathcal{A}^{ab,ij}_k =\kp \sum_{k=1}^8 \mathcal{S}^{ab,ij}_k + i\kpt \sum_{k=1}^8 \mathcal{P}^{ab,ij}_k, 
\eeq 
%
where $\epsilon_{\lambda_a}$ and $\epsilon_{\lambda_b}$ are the
polarization vectors of $g_a$ and $g_b$ respectively and in the last
equality the scalar piece of each diagram has been separated from the
pseudoscalar one. The quantity $\mathcal{A}^{ab,ij}$ can be linked to
the final states particles by using two spinors, $\psi_{t}$ and
$\psi_{\tbar}$, that will contain the information on the decay of the
virtual top and anti-top, respectively. Hence, the amplitude for the
complete process depicted in Fig.~\ref{fig2} can be witten as
%
\beq
\label{eq6}
\mathcal{M}^{ab,ij}=\bar{\psi}_t\,\mathcal{A}^{ab,ij}\,\psi_{\tbar}\,.
\eeq
%
If we assume $b,l$ and $\nu_l$ to be massless, we can make use of the
spinor techniques\footnote{These spinors techniques can also be used
  for massive particles. However, the derivation of Eq.~(\ref{eq2}) is
  much simpler by assuming the final state particles to be massless
  which is a sensible approximation in view of the energy involved in
  the process.} developed in \cite{Kleiss} to write $\bar{\psi}_t$ and
$\psi_{\tbar}$ as follows
%
\bea
\label{eq7}
\bar{\psi}_t &=& -g^2\,\frac{1}{(t^2-m^2_t+im_t\Gamma_t)}\,\frac{1}{((t-b)^2-M^2_W+im_W\Gamma_W)}\,\langle b-|\nu_{\ell}+\rangle \langle {\ell}^+\!+|(\tslash+m_t)\\
\label{eq8}
\psi_{\tbar} &=& -g^2\,\frac{1}{(\tbar^2-m^2_t+im_t\Gamma_t)}\frac{1}{((\tbar-\bbar)^2-M^2_W+im_W\Gamma_W)}\,\langle\nubar+|\bbar-\rangle (\tbslash-m_t)|{\ell}^-+\rangle,
\eea
%
where $|i+(-)\rangle \equiv P_{R(L)}\,\psi_i$ ($P_{R,L}=(1\pm
\gamma^5)/2$) represents a right-handed(left-handed) chiral spinor for
particle $i$, $\langle i+(-)|$ is the corresponding adjoint spinor and
we have denoted the momenta through the symbols that represents the
name of the particles \cite{Mangano}. The relevant quantity is
$|\mathcal{M}^{ab,ij}|^2$, which from Eqs.~(\ref{eq6})-(\ref{eq8}) is
proportional to $((t^2-m^2_t)^2+m^2_t\Gamma^2_t)^{-1}$ and
$((\tbar^2-m^2_t)^2+m^2_t\Gamma^2_t)^{-1}$. Under the narrow-width
approximation we can replace these factors by
$(\pi/m_t\Gamma_t)\delta(t^2-m^2_t)$ and
$(\pi/m_t\Gamma_t)\delta(\tbar^2-m^2_t)$ respectively. Hence, we can
take directly $t^2=\tbar^2=m^2_t$ in the amplitude and write
%
\beq
\label{eq9}
\mathcal{M}^{ab,ij}=\!-g^4\,\mathbb{P}_t(t)\,\mathbb{P}_{\tbar}(\tbar)\,\mathbb{P}_W(t-b)\,\mathbb{P}_W(\tbar-\bbar)\,\langle b-|\nu_{\ell}+\rangle \langle\nubar+|\bbar-\rangle \sqrt{\strut2(t\cdot {\ell}^+)}\sqrt{\strut2(\tbar\cdot {\ell}^-)}\,[\bar{\phi}_t \mathcal{A}^{ab,ij}\phi_{\tbar}],
%\frac{1}{(t^2-m^2_t+im_t\Gamma_t)}\,\frac{1}{(\tbar^2-m^2_t+im_t\Gamma_t)}\,\frac{1}{((t-b)^2-M^2_W)}\frac{1}{((\tbar-\bbar)^2-M^2_W)}\,\langle b-|\nu+\rangle\,\langle\nubar+|\bbar-\rangle
\eeq
%
where we have used the notation
$\mathbb{P}_t(t)=(t^2-m^2_t+im_t\Gamma_t)^{-1}$,
$\mathbb{P}_{\tbar}(\tbar)=(\tbar^2-m^2_t+im_t\Gamma_t)^{-1}$,
$\mathbb{P}_W(t-b)=((t-b)^2-M^2_W+im_W\Gamma_W)^{-1}$ and
$\mathbb{P}_W(\tbar-\bbar)=((\tbar-\bbar)^2-M^2_W+im_W\Gamma_W)^{-1}$. The
spinors $\phi_{t}$ and $\phi_{\tbar}$ are defined as follows
%
\beq
\label{eq10}
\phi_t=\frac{(\tslash +m_t)}{\sqrt{\strut2(t\cdot {\ell}^+)}}|{\ell}^++\rangle
\eeq
%
\beq
\label{eq11}
\phi_{\tbar}=\frac{(\tbslash -m_t)}{\sqrt{\strut2(\tbar\cdot {\ell}^-)}}|{\ell}^-+\rangle
\eeq
%
and describe a top quark with spin vector $n_t$ and an anti-top quark
with spin vector $n_{\tbar}$, since the corresponding projection
operators are given by
%
\beq
\label{eq12}
\phi_t\,\bar{\phi}_t=\frac{1}{2}(1+\nslash_{\!t}\gamma^5)(\tslash +m_t)
\eeq
%
\beq
\label{eq13}
\phi_{\tbar}\,\bar{\phi}_{\tbar}=\frac{1}{2}(1+\nslash_{\!\tbar}\gamma^5)(\tbslash -m_t),
\eeq
%
with $n_t$ and $n_{\tbar}$ being the vectors defined in
Eq.~(\ref{eq3})-(\ref{eq4}). Moreover, if we use these spinors to
compute the amplitude for a top(anti-) with spin vector
$n_t$($n_{\tbar}$) to decay into
$b{\ell}^+\nu_{\ell}$($\bar{b}{\ell}^-\bar{\nu}_{\ell}$) we obtain
%
\beq
\label{eq14}
\mathcal{M}(t(n_t)\to b{\ell}^+\nu_{\ell})=ig^2\mathbb{P}_W(t-b)\langle b-|\nu_{\ell}+\rangle\sqrt{\strut2(t\cdot {\ell}^+)}
\eeq
%
\beq
\label{eq15}
\mathcal{M}(\tbar(n_{\tbar})\to \bar{b}{\ell}^-\bar{\nu}_{\ell})=ig^2\mathbb{P}_W(\tbar-\bar{b})\langle \bar{\nu}_{\ell}+|\bar{b}-\rangle\sqrt{\strut2(\tbar\cdot {\ell}^-)}.
\eeq
%
On the other hand, the term inside the square brackets in
Eq.~(\ref{eq9}) is nothing but the amplitude for producing a top quark
with spin vector $n_t$, along with an anti-top with spin vector
$n_{\tbar}$ and a Higgs boson, i.e.
%
\beq
\label{eq16}
\mathcal{M}(g_ag_b \to t^i(n_t)\tbar^j(n_{\tbar})H)=\bar{\phi}_t \mathcal{A}^{ab,ij}\phi_{\tbar}.
\eeq
%
By combining Eqs.~(\ref{eq14})-(\ref{eq16}), we can write Eq.~(\ref{eq9}) as follows
%
\beq
\label{eq17}
\mathcal{M}^{ab,ij}=\mathbb{P}_t(t)\mathbb{P}_{\tbar}(\tbar)\,\mathcal{M}(t(n_t)\to b{\ell}^+\nu_{\ell})\,\mathcal{M}(\tbar(n_{\tbar})\to \bar{b}{\ell}^-\bar{\nu}_{\ell})\,\mathcal{M}(g_ag_b \to t^i(n_t)\tbar^j(n_{\tbar})H).
\eeq
%
From Eq.~(\ref{eq17}), Eq.~(\ref{eq2}) can be derived. It's important
to note that, even though Eqs.~(\ref{eq2}) and (\ref{eq9}) appear to
be factorized into production and decay parts, these are not
independent, since both parts share kinematical variables due to the
presence of the spin vectors $n_t$ and $n_{\tbar}$ (see
Eqs.~(\ref{eq14})-(\ref{eq16}) along with the definitions given in
Eqs.~(\ref{eq3}) and (\ref{eq4})).\par
%%%%%%%%%%%%%%%%%%% Quitarlo %%%%%%%%%%%%%%%%%%%%
\setlength{\abovedisplayskip}{10.2pt}
\setlength{\belowdisplayskip}{10.2pt}
%%%%%%%%%%%%%%%%%%%%%%%%%%%%%%%%%%%%%%%%%%%%%%%%%
Let us focus now in the production amplitude in Eq.~(\ref{eq17})
needed for computing the production part of the unpolarized
differential cross-section. By summing over colour and gluon indices
we have
%
\beq
\label{eq18}
\sum_{\substack{a,b \\ i,j}}|\mathcal{M}(g_ag_b \to t^i(n_t)\tbar^j(n_{\tbar})H)|^2=\sum_{\substack{a,b \\ i,j}}\left|\sum^{8}_{k=1}C^{ab,ij}_k\,\bar{\phi}_t (\kp\mathcal{S}_k+i\kpt\mathcal{P}_k)\phi_{\tbar}\right|^2,
\eeq
%
where we have separated the colour structure of each diagram by using
the definitions $\mathcal{S}^{ab,ij}= C^{ab,ij}_k \mathcal{S}$ and
$\mathcal{P}^{ab,ij}= C^{ab,ij}_k \mathcal{P}$. Also, the factors
$g^2_s m_t/v$ and $-ig^2_s m_t/v$ arising from the vertices of the t-
and s-channel diagrams respectively have been included in the
definition of $C^{ab,ij}_k$ for convenience. The terms linear in $\kp$
and $\kpt$ can be written as
%
\beq
\label{eq19}
\mathcal{O}(\kp\kpt)\to \frac{1}{2}\kp\kpt \sum_{k,r}\mathbb{C}_{kr}\mathrm{Im}
\left\lbrace \mathrm{Tr}\left[ (1+\nslash_t \gamma^5)(\tslash+m_t)\mathcal{S}_k(1+\nslash_{\tbar}\gamma^5
 )(\tbslash -m_t)\tilde{\mathcal{P}}_r \right] \right\rbrace ,
\eeq
%
where the factor $\mathbb{C}_{kr}=\sum_{ab,ij}C^{ab,ij}_k
C^{ab,ij*}_r$ is real and $\tilde{\mathcal{P}}_r = \gamma^0
\mathcal{P}_r \gamma^0$. The only contributing terms are those with an
odd number of $\gamma^5$ matrices and will lead to triple-products
(TP) structures, i.e. contractions between the levi-civita tensor
$\epsilon_{\alpha\beta\gamma\delta}$ with four momenta. In contrast,
it can be seen from Eq.~(\ref{eq18}) that the terms proportional to
$\kp^2$ and $\tilde{\kappa}^2_t$ contain an even number of $\gamma^5$
matrices and can be written in terms of scalar products of the
available momenta.\par
%\vspace*{-2mm}

In order to state this more clearly, we will consider the form of the
differential cross-section in terms of the momenta $q=(q_1-q_2)/2$,
$Q=(q_1+q_2)/2$, $t$, $\bar{t}$, $n_t$ and $n_{\tbar}$, where
$q_{1,2}$ denote the momenta of the gluons. Note that with this
choice, $q\cdot Q=0$ and $Q^2=-q^2=M^2_{t\tbar H}/4$, where $M_{t\tbar
  H}$ is the invariant mass of the system $t\tbar H$. From these six
vectors, fifteen TPs can be built\footnote{ We note that these fifteen
  TPs are not linearly independent (see the epsilon relations
  discussed in \cite{identities}).}, and the general form of the
production differential cross-section will be given by
%
\beq
\label{eq20}
d\sigma(gg\to t(n_t)\tbar (n_{\tbar})H)= \kp^2\,f_1(p_i\cdot p_j)+\tilde{\kappa}^2_t\,f_2(p_i\cdot p_j)+\kp\kpt\,\sum^{15}_{l=1}g_l(p_i\cdot p_j)\,\epsilon_l,
\eeq   
%
where
$\epsilon_l=\epsilon_{\alpha\beta\gamma\delta}\,p^{\alpha}_ap^{\beta}_bp^{\gamma}_cp^{\delta}_d$
denotes the $l$th TP, the Roman indices refer to any of the involved
momenta and we adopt the convention $\epsilon_{0123}=+1$. The
functions $f_{1,2}$ and $g_k$ depend only on the possible scalar
products and are therefore even under a parity transformation
($\mathrm{P}$). However, the terms linear in $\kp\kpt$ are
$\mathrm{P}$-odd due to the presence of the $\mathrm{P}$-odd
TPs. Hence, only the functions $f_{1,2}$ will contribute to the total
cross-section, whereas the TPs terms will be sensitive to the sign of
the anomalous coupling $\kpt$. From the fifteen TPs mentioned above,
we will focus on those that do not include $q$ and contain both spin
vectors $n_t$ and $n_{\tbar}$. The former is motivated in the fact
that $q$ cannot be eliminated in terms of the momenta of final state
particles (like $Q$ in virtue of momentum-energy conservation). The
latter takes into account that for $t\tbar$ production at hadron
colliders the top and antitop are not polarized but their spins are
strongly correlated and observables combining the decay products of
$t$ and $\tbar$ will be sensitive to this spin correlation
\cite{Bernreuther}.  A similar behaviour is expected in $t\tbar H$
production where it can be shown that the single-spin asymmetries
vanish \cite{Ellis,Biswas}. Hence, in order to construct observables
sensitive to the structure of the $tH$ coupling, we will prefer those
TPs that include information on the decay products of both top and
anti-top quarks. Only five of the fifteen TPs in Eq.~(\ref{eq20}) do
not involve the vector $q$ and, among these, we will focus on the
three that include both $n_t$ and $n_{\tbar}\,$:
%
\beq
\label{eqa1}
\epsilon_1\equiv\epsilon(t,\tbar,n_t,n_{\tbar}),\,\,
\vspace*{1mm}
\eeq
%
\beq
\label{eqa2}
\epsilon_2\equiv\epsilon(Q,\tbar,n_t,n_{\tbar}),
\vspace*{1mm}
\eeq
%
\beq
\label{eqa3}
\epsilon_3\equiv\epsilon(Q,t,n_t,n_{\tbar}).
\vspace*{1mm}
\eeq
%
\par  

Finally, we remark that even though all the above discussion was given
within the context of $gg$-initiated production, similar conclusions
are obtained for $q\qbar$-initiated production. In particular, the
definitions of the spin vectors in Eqs.~(\ref{eq3})-(\ref{eq4}) and
the general form of $d\sigma$ introduced in Eq.~(\ref{eq20}) are valid
in both cases.
%%%%%%%%%%%%%%%%%%%%%%%%%%%%%%%%%%%%%%%%%%%%%%%%%
%\newpage
%%%%%%%%%%%%%%%%%%%%%%%%%%%%%%%%%%%%%%%%%%%%%%%%%
%\setlength{\belowdisplayskip}{10pt} \setlength{\belowdisplayshortskip}{10pt}
%\setlength{\abovedisplayskip}{10pt} \setlength{\abovedisplayshortskip}{10pt}
\bigskip
\section{$\mathrm{\mathbf{CP}}$-odd observables}
\label{sec3}
We present below three types of observables based on the TPs discussed
in Sec.~\ref{sec2}, mean values, asymmetries and angular
distributions. These observables are sensitive not only to the
magnitude of the pseudoscalar coupling $\kpt$ but also to its
sign. With the aim of testing them, we have used in all the cases a
set of $10^5$ events $\ppprocess$ simulated at parton level with
$\mathtt{MadGraph5\_aMC@NLO}$ \cite{Madgraph} at a center-of-mass
energy of $14\,\mathrm{TeV}$ for different values of the couplings
$\kp$ and $\kpt$\footnote{Although we use the same number of events in
  all the cases, the required luminosities are different since the
  cross section depends on the value of $\kpt$.}. We have used this
somewhat large number of events with the aim of determining clearly
the extent to which the proposed observables are sensitive to the NP
coupling. However, we will discuss the corresponding experimental
viability in Sec.~\ref{sec6}. Finally, we have also imposed the
following set of cuts: $p_T$ of leptons $> 10\,\mathrm{GeV}$, $|\eta|$
of leptons $< 2.5$, $|\eta|$ of b jets $< 2.5$ and $\Delta
R_{\ell\ell}>0.4$.\par Under the assumption of no other new physics
sources but the inclusion of the pseudoscalar coupling $\kpt$,
indirect contraints based on the Higgs production from gluon fusion
followed by its decay to two photons disfavour $\kp < 0$ without
resolving the degeneracy in the sign of $\kpt$ \cite{Guadagnoli}. On
the other hand, by assuming the tensor structure of the Lagrangian to
be the same as in the SM and considering a parameterization with one
universal Higgs coupling to vector bosons, $\kappa_V$, and one
universal Higgs coupling to fermions, $\kappa_f$, the measured signal
strengths provided by ATLAS and CMS collaborations are compatible with
the values predicted by the SM, ($\kappa_f=1,\kappa_V=1$). Based on
these facts, we will set the value of the scalar coupling to its SM
value, $\kp=1$, and concentrate in modifications of the value as well
as the sign of the pseudoscalar coupling. In particular, we have
analized the cases $\kpt=0,\pm 0.25, \pm 0.5, \pm 0.75,\pm
1$. However, we will give some results and comments regarding the pure
$\mathrm{CP}$-odd hypothesis ($\kp=0,\kpt=1$).
\DIFdelbegin %DIFDELCMD < \newpage
%DIFDELCMD < %%%
\DIFdelend %DIF > \newpage
%%%%%%%%%%%%%%%%%%%%%%%%%%%%%%%%%%%%%%%%%%%%%%%%%
\subsection{Asymmetry}
%\bigskip
\label{sec3.1}
The first type of $\mathrm{CP}$-odd observable we will consider is the
asymmetry. For a certain TP, we define an asymmetry which gives the
balance between the number of events with $\epsilon > 0$ and those for
which $\epsilon < 0$:
%
\beq
\label{eq21}
\mathcal{A}(\epsilon)=\frac{N(\epsilon > 0)-N(\epsilon < 0)}{N(\epsilon > 0)+N(\epsilon < 0)}.
\eeq 
%
Based on the general expression given in Eq.~(\ref{eq20}), we expect
the following shape for the asymmetry,
%
\beq
\label{eq22}
\mathcal{A}(\epsilon)=\frac{A\kp\kpt}{B\kappa^2_t+C\tilde{\kappa}^2_t},
\eeq 
%
which for $\kp=1$ can be parameterized as
%
\beq
\label{eq23}
\mathcal{A}(\epsilon)=\frac{a\kpt}{1+b\tilde{\kappa}^2_t},
\eeq 
%
where the parameter $a\equiv A/B$ determines the sensitivity to the
pseudoscalar coupling, whereas $b\equiv C/B$ quantifies the deviation
from the linear behaviour.  The results for $\kp=1$ and $\kpt=0,\pm 1$
are shown in Table \ref{table1}, where the values in terms of the
corresponding statistical uncertainty are also included. It can be
seen that the considered asymmetries are useful to separate the
$\mathrm{CP}$-mixed cases from the SM case and also to discriminate
the different sign of $\kpt$, giving deviations of order $10\sigma$ in
the former case and $20\sigma$ in the latter.\par The sensitivity of
the asymmetry is quite similar for the three TPs, as can be seen by
including other values of $\kpt$ and using the expression in
Eq.~(\ref{eq23}) as fitting function (see Fig.~\ref{fig3}). In fact,
we obtain the fit parameters $(a=-0.057\pm 0.006, b=0.5\pm 0.2),
(a=-0.056\pm 0.006, b=0.5 \pm 0.2)$ and $(a=0.058\pm 0.006, b=0.6 \pm
0.2)$ for $\epsilon_1$, $\epsilon_2$ and $\epsilon_3$,
respectively. In addition, we include in Fig.~\ref{fig4} the asymmetry
for $\epsilon_1$ isolating the $gg$-initiated, $q\qbar$-initiated and
$pp$ production. We see that the asymmetry is enhanced for
$gg$-initiated production, while it is reduced and of opposite sign
for the $q\qbar$-initiated case. The asymmetry for the $pp$ case is
then dominated by the $gg$ contribution, albeit smaller due to the
$q\qbar$ contribution.
\newcolumntype{C}[1]{>{\centering\arraybackslash}p{#1}}
\renewcommand{\arraystretch}{1.4}
\begin{table}[H]
\caption{Asymmetries obtained by using $10^5$ simulated events for the
  TPs $\epsilon_1=\TPa, \epsilon_2=\TPb$ and $\epsilon_3=\TPc$ in the
  SM case and the two $\mathrm{CP}$-mixed cases defined by
  $\kp=1,\kpt=\pm 1$.}
\label{table1}
\begin{center}
\begin{tabular}{|C{1cm}|C{1cm}||C{2cm}|C{2cm}||C{2cm}|C{2cm}||C{2cm}|C{2cm}|}
%\begin{tabular}{|c|r||r|c||r|c||r|c|}
\hhline{|========|}
%\hhline{|--------|}
$\kappa_t$&$\tilde{\kappa}_t$~~&$\mathcal{A}(\epsilon_1)$~~&$\mathcal{A}(\epsilon_1)/\sigma_{\mathcal{A}}$& $\mathcal{A}(\epsilon_2)$~~&$\mathcal{A}(\epsilon_2)/\sigma_{\mathcal{A}}$&$\mathcal{A}(\epsilon_3)$~~&$\mathcal{A}(\epsilon_3)/\sigma_{\mathcal{A}}$  \\ 
\hhline{|========|} 
%\hhline{|--------|}
$1$ & $-1$~~~ & $0.0315$~ & $10.0$ & $0.0332$~ & $10.5$~ & $-0.0307$~~~ & $-9.7$~~~\\[0.6mm]
\hline
$1$ & $0$ & $-0.0021$~~~ & $-0.7$~~~ & $0.0009$~ & $0.3$~ & $-0.0011$~~~ & $-0.3$~~~\\[0.6mm]
\hline
$1$ & $1$ & $-0.0379$~~~ & $-12.0$~~~ & $-0.0411$~~~& $-13.0$~~~ & $\,0.0378$~ & $12.0$ \\[0.6mm]
\hhline{|========|}
%\hhline{|--------|}
\end{tabular}
\end{center} 
\end{table}
%%%%%%%%%%%%%%%%%%%%%%%%% FIGURA 3 %%%%%%%%%%%%%%%%%%%%%%%%%
\begin{center}
\vspace*{2.5mm}
\begin{figure}[H]
%\centering
\hspace*{-0.45cm}
\subfloat{\includegraphics[scale=0.45]{ATP1_nuevo.pdf}}
\hspace*{0.002\textwidth}
\subfloat{\includegraphics[scale=0.45]{ATP2_nuevo.pdf}} \\
%\hspace*{0.03\textwidth}
\centering
\subfloat{\includegraphics[scale=0.45]{ATP3_nuevo.pdf}}
\caption{Asymmetries for the TPs
  $\epsilon_1=\epsilon(t,\tbar,n_t,n_{\tbar})$ (top-left),
  $\epsilon_2=\epsilon(Q,\tbar,n_t,n_{\tbar})$ (top-right) and
  $\epsilon_3=\epsilon(Q,t,n_t,n_{\tbar})$ (bottom). The points
  represent the values for $\kpt=0,\pm 0.25, \pm 0.5, \pm 0.75,\pm 1$
  and the red solid line is the fitting curve.}
\label{fig3}
\end{figure}
\end{center}
%%%%%%%%%%%%%%%%%%%%%%%%%%%%%%%%%%%%%%%%%%%%
%
%%%%%%%%%%%%%%%%%%%%% FIGURA 4  %%%%%%%%%%%%%%%%%%%%%%%%%%%%
\begin{center}
\vspace*{-4mm}
\begin{figure}[H]
\centering
\includegraphics[scale=0.45]{ATP1juntos_nuevo.pdf}
\caption{Asymmetry for the TP
  $\epsilon_1=\epsilon(t,\tbar,n_t,n_{\tbar})$. The dashed line (red)
  corresponds to $gg$-initiated production, the dot-dashed line (grey)
  to $q\qbar$-initiated production and the solid line (blue) to $pp$
  production.}
\label{fig4}
\end{figure}
\end{center}
%%%%%%%%%%%%%%%%%%%%%%%%%%%%%%%%%%%%%%%%%%%%%%%%%%%%%%%%%%%%%
\vspace*{-6mm} \par On the other hand, we have also tested various
linear combinations of the TPs $\epsilon_{1,2,3}$, with the result
that the asymmetry is enhanced in the following case:
%
\beq
\setlength{\abovedisplayskip}{9.5pt}
\setlength{\belowdisplayskip}{9.5pt}
\label{eq24}
\epsilon_4=\epsilon_3-\epsilon_2=\epsilon(Q,t-\tbar,n_t,n_{\tbar}).
\eeq
%
Note that in the $Q$ rest frame $\epsilon_4=Q^0
(\vec{t}-\vec{\tbar}\,)\cdot(\vec{n}_t\times \vec{n}_{\tbar})$ and the
sign of this TP is determined by the quantity
$(\vec{t}-\vec{\tbar}\,)\cdot(\vec{n}_t\times \vec{n}_{\tbar})$. The
values obtained for the asymmetry of this TP are shown in
\ref{table2}. We see by comparing the results in Tables \ref{table1}
and \ref{table2} that the capability of the asymmetry to separate the
two $\mathrm{CP}$-mixed hypotheses is increased by at least $3\sigma$.
\vspace{4mm}
\begin{table}[H]
\caption{Asymmetry for the TP $\epsilon_{4}$ for the SM case and the
  two $\mathrm{CP}$-mixed cases defined by $\kp=1,\kpt=\pm 1$. The
  values are obtained by using $10^5$ simulated events.}
\label{table2}
\begin{center}
\begin{tabular}{|C{1cm}|C{1cm}||C{2cm}|C{2cm}|}
%\begin{tabular}{|c|r||r|c||r|c||r|c|}
\hhline{|====|}
%\hhline{|--------|}
$\kappa_t$&$\tilde{\kappa}_t$~~&$\mathcal{A}(\epsilon_4)$~~&$\mathcal{A}(\epsilon_4)/\sigma_{\mathcal{A}}$ \\ 
\hhline{|====|} 
%\hhline{|--------|}
$1$ & $-1$~~~ & $-0.0371$~~~ & $-12$~~~ \\[0.6mm]
\hline
$1$ & $0$ & $0.0004$~ & $0.1$ \\[0.6mm]
\hline
$1$ & $1$ & $0.0461$~ & $14\,$ \\[0.6mm]
\hhline{|====|}
%\hhline{|--------|}
\end{tabular}
\end{center} 
\end{table}
\par Finally, it is worth noting that the asymmetry is not useful for
discriminating between the SM hypothesis ($\kp=1,\kpt=0$) and the pure
pseudoscalar hypothesis ($\kp=0,\kpt=1$) because, being linear in both
$\kp$ and $\kpt$, it is expected to vanish in these cases. However, we
will show in the next section that there exist angular distributions
derived from the TPs that are actually suitable to distinguish these
two hyphothesis.
%%%%%%%%%%%%%%%%%%%%%%%%%%%%%%%%%%%%%%%%%%%%%%%%%
\subsection{Angular Distributions}
\label{sec3.2}
Given a certain TP, it is possible to define asocciated angular
distributions that are also sensitive to the pseudoscalar coupling
$\kpt$. In order to clarify this, let us first consider the TP
$\epsilon(t,\tbar,n_t,n_{\tbar})$. This TP can be written as
$\epsilon(t+\tbar,\tbar,n_t,n_{\tbar})$ so that in the reference frame
defined by $\vec{t}+\vec{\tbar} =0$ and $\vec{\tbar}\parallel \hat{z}$
we have
%
\beq
\label{eq25}
\epsilon(t+\tbar,\tbar,n_t,n_{\tbar})=M_{t\tbar}\,|\vec{\tbar}|\,(\vec{n}_t\times \vec{n}_{\tbar})_z=M_{t\tbar}\,|\vec{\tbar}||\vec{n}_t||\vec{n}_{\tbar}|\sin\theta_{n_t}\sin\theta_{n_{\tbar}}\sin \Delta \phi(n_t,n_{\tbar}),
\eeq
%
where $M_{t\tbar}$ is the invariant mass of the $t\tbar$ pair, the
angles $\theta_{n_t}$ and $\theta_{n_{\tbar}}$ denote the polar angles
of $\vec{n}_t$ and $\vec{n}_{\tbar}$ respectively and finally
$\Delta\phi(n_t,n_{\tbar})$ is the angular difference between the
projections of $\vec{n}_t$ and $\vec{n}_{\tbar}$ onto the plane
perpendicular to $\vec{\bar{t}}$. If we define this angle within the
range $[-\pi,\pi]$, we see from Eq.~(\ref{eq25}) that its sign will
determine the sign of the TP. In this sense, the distribution of the
number of the events with respect to the angle
$\Delta\phi(n_t,n_{\tbar})$ is related to the asymmetry of the TP,
%
\beq
\label{eq26}
\mathcal{A}(\epsilon)=1-2\frac{N(\epsilon < 0)}{N_T}\,\,\mbox{ and }\,\,\frac{N(\epsilon < 0)}{N_T}=\int^0_{-\pi}\frac{1}{N_T}\frac{dN}{d\Delta\phi(n_t,n_{\tbar})}\,d\Delta\phi(n_t,n_{\tbar}),
\eeq  
%
where $N_T$ is the total number of events. Moreover, from a certain TP
one can derive different angular distributions by considering
different reference frames, but all of these will satisfy
Eq.~(\ref{eq26}). In particular, we will analyze the following angular
distributions based on the TPs defined in Sec.~\ref{sec2}:
 \begin{itemize} 
\item Related to $\epsilon_1 = \TPa$ we consider
  $d\sigma/d\Delta\phi_1(n_t,n_{\tbar})$ in the rest frame of $t\tbar$
  with $\vec{\tbar}$ in the $z$-axis. The angle
  $\Delta\phi_1(n_t,n_{\tbar})$ is the angular difference between the
  projection of the spin vectors in the plane perpendicular to
  $\vec{\tbar}$.
\item Related to $\epsilon_2 = \TPb$ we consider
  $d\sigma/d\Delta\phi_2(n_t,n_{\tbar})$ in the rest frame of $Q$ with
  $\vec{\tbar}$ in the $z$-axis. The angle
  $\Delta\phi_2(n_t,n_{\tbar})$ is the angular difference between the
  projection of the spin vectors in the plane perpendicular to
  $\vec{\tbar}$.
\item Related to $\epsilon_3 = \TPc$ we consider
  $d\sigma/d\Delta\phi_3(n_t,n_{\tbar})$ in the rest frame of $Q$ with
  $\vec{t}$ in the $z$-axis. The angle $\Delta\phi_3(n_t,n_{\tbar})$
  is the angular difference between the projection of the spin vectors
  in the plane perpendicular to $\vec{t}$.
 \end{itemize} 
\par
%
In Fig.~\ref{fig5} we present the normalized distributions obtained
for the first case listed above with $\kp= 1,\kpt=0$
($\mathrm{CP}$-even Higgs boson), $\kp= 1,\kpt=\pm 1$
($\mathrm{CP}$-mixed Higgs boson) and $\kp= 0,\kpt=1$
($\mathrm{CP}$-odd Higgs boson). In Fig.~\ref{fig6} we display in turn
the analogous distributions for $\epsilon_2$. The distributions
correponding to $\epsilon_3$ are similar to those of $\epsilon_2$ but
with the opposite shift for $\kp=\kpt=1$ and $\kp=-\kpt=1$ and then we
do not include them here.
%%%%%%%%%%%%%%%%%%%%%%%%% FIGURA 5 %%%%%%%%%%%%%%%%%%%%%%%%%
%%%% ruta turin facultad: /home/nico/Documentos/ttbarH/Material_final/ todas las figs para abajo
\begin{center}
\vspace*{1.4mm}
\begin{figure}[H]
%\centering
\hspace*{-0.52cm}
\subfloat{\includegraphics[scale=0.45]{TP1_10_nuevo.pdf}}
\hspace*{-0.006\textwidth}
\subfloat{\includegraphics[scale=0.45]{TP1_01_nuevo.pdf}} \\
\hspace*{-0.45cm}
%\hspace*{0.03\textwidth}
\subfloat{\includegraphics[scale=0.45]{TP1_11_nuevo.pdf}}
\hspace*{-0.006\textwidth}
\subfloat{\includegraphics[scale=0.45]{TP1_1-1_nuevo.pdf}}
\caption{Angular distributions associated with the TP
  $\epsilon(t,\tbar,n_t,n_{\tbar})$ for various values of
  $(\kp,\kpt)$. The error bars correspond to the statistical
  uncertainties.}
\label{fig5}
\end{figure}
\end{center}
%%%%%%%%%%%%%%%%%%%%%%%%% FIGURA 6 %%%%%%%%%%%%%%%%%%%%%%%%%
\begin{center}
%\vspace*{-4mm}
\begin{figure}[H]
%\centering
\hspace*{-0.52cm}
\subfloat{\includegraphics[scale=0.45]{TP2_10_nuevo.pdf}}
\hspace*{-0.006\textwidth}
\subfloat{\includegraphics[scale=0.45]{TP2_01_nuevo.pdf}} \\
\hspace*{-0.45cm}
%\hspace*{0.03\textwidth}
\subfloat{\includegraphics[scale=0.45]{TP2_11_nuevo.pdf}}
\hspace*{-0.006\textwidth}
\subfloat{\includegraphics[scale=0.45]{TP2_1-1_nuevo.pdf}}
\caption{Angular distributions associated with the TP $\TPb$ for
  various values of $(\kp,\kpt)$. The error bars correspond to the
  statistical uncertainties.}
\label{fig6}
\end{figure}
\end{center}
\par As can be seen from Figs.~\ref{fig5} and \ref{fig6}, the peak of
the distributions is shifted according to the values of the couplings
$\kp, \kpt$. This shift appears to be approximately the same but with
opposite sign for $\kp=\kpt=1$ and $\kp=-\kpt=1$, distinguishing then
the sign of the pseudoscalar coupling. The last is the expected
behaviour since the asymmetry is linear in $\kpt$ (see
Sec.~\ref{sec3.1}) and the quantity $N(\epsilon < 0)/N_T$ is related
to the angular distribution according to Eq.~(\ref{eq26}). The
distributions for the SM and the pure pseudoscalar hypotheses are
clearly different. While in the former case the distributions exhibit
a minimum at $\Delta\phi_{1,2}(n_t,n_{\tbar})=0$, in the latter the
similar distributions display a peak, so that the two hypotheses can
be discriminated, in contrast to the asymmetry that vanishes for both
hypotheses. In order to quantify the mentioned shifts, we have fitted
the simulated distributions with the following function proposed in
\cite{Ellis},
%
\beq
\label{eq27}
\frac{1}{\sigma}\frac{d\sigma}{d\Delta\phi_i(n_t,n_{\tbar})}=a_0 + a_1\cos(\Delta\phi_i(n_t,n_{\tbar})+\delta),\qquad\, i=1,2,3.
\eeq
%
To the extent that this equation is exact we note that
Eq.~(\ref{eq29}) gives $\mathcal{A}(\epsilon_i)=4a_1 \sin\delta$. With
this fitting function, we obtain phase shifts approximately between
$0.9$ and $1$ ($-1$ and $-0.9$) for $\kp=\kpt=1$ ($\kp=-\kpt=1$) both
for $\epsilon_1$ and $\epsilon_2$. However, the quality of the fits is
not good enough, being worse in the case of the $\epsilon_1$
distributions for which $\chi^2/\mathrm{d.o.f}$ is in the range
$1.69$-$3.86$, in contrast to the range $0.53$-$1.16$ obtained for
$\epsilon_2$. The main impact on the deviation from the functional
form proposed in Eq.~(\ref{eq27}) appears to be caused by the $\Delta
R_{ll}$ cut we have imposed. In fact, when this cut is turned off the
above ranges for $\chi^2/\mathrm{d.o.f}$ modify to $0.75$-$1.14$ and
$0.44$-$1.07$ for the $\epsilon_1$ and $\epsilon_2$ distributions
respectively. In Tables \ref{table3} and \ref{table4} we list the
results of the fits obtained when the $\Delta R_{\ell\ell}$ cut is
relaxed. The results for the TP $\epsilon_3$ are pretty similar to
those for $\epsilon_2$ but with opposite sign in the phase shifts of
the $\mathrm{CP}$-mixed hypotheses and hence we do not include them
here. In addition, we show in Fig.~\ref{fig7} the angular
distributions along with the fit curves for both $\mathrm{CP}$-mixed
cases when no cut is imposed in $\Delta R_{\ell\ell}$.
\renewcommand{\arraystretch}{1.2}
\begin{table}[H]
\caption{Fit results for the angular distribution $d\sigma/(\sigma
  d\Delta\phi_1(n_t,n_{\tbar}))$ related to the TP
  $\epsilon_1=\epsilon(t,\tbar,n_t,n_{\tbar})$ when the $\Delta
  R_{\ell\ell}$ is turned off. Note that the parameter $a_1$ change
  its sign for $\kp=0,\kp=1$.}
\label{table3}
\begin{center}
\begin{tabular}{|C{1cm}|C{1cm}||C{3cm}|C{3cm}|C{3cm}|}
%\begin{tabular}{|c|r||r|c||r|c||r|c|}
\hhline{|=====|}
%\hhline{|--------|}
$\kappa_t$&$\tilde{\kappa}_t$~~&$a_0$~~&$a_1$& $\delta$~~ \\ 
\hhline{|=====|} 
%\hhline{|--------|}
$1$ & $-1$~~~ & $0.1592 \pm 0.0006$ & $-0.0139 \pm 0.0008$ & $0.81 \pm 0.07$ \\[0.6mm]
\hline
$1$ & $0$ & $0.1595 \pm 0.0006$ & $-0.0181 \pm 0.0008$ & $0.002 \pm 0.06\,\,$ \\[0.6mm]
\hline
$1$ & $1$ & $0.1591 \pm 0.0006$ & $-0.0131 \pm 0.0008 $ & $\,-0.82 \pm 0.07\quad$  \\[0.6mm]
\hline
$0$ & $1$ & $0.1591 \pm 0.0006$ & ~~$\,0.0102 \pm 0.0008$ & $0.11 \pm 0.08$ \\
\hhline{|=====|}
%\hhline{|--------|}
\end{tabular}
\end{center} 
\end{table}
%%%%%%%%%%%%%%%%%%%%%%%%%%%%%%%%%%%%%%%%%%%%%%%%%%%%%
\begin{table}[H]
\caption{Fit results for the angular distribution $d\sigma/(\sigma
  d\Delta\phi_2(n_t,n_{\tbar}))$ related to the TP
  $\epsilon_2=\epsilon(Q,\tbar,n_t,n_{\tbar})$ when the $\Delta
  R_{\ell\ell}$ is turned off. Note that the parameter $a_1$ change
  its sign for $\kp=0,\kp=1$.}
\label{table4}
\begin{center}
\begin{tabular}{|C{1cm}|C{1cm}||C{3cm}|C{3cm}|C{3cm}|}
%\begin{tabular}{|c|r||r|c||r|c||r|c|}
\hhline{|=====|}
%\hhline{|--------|}
$\kappa_t$&$\tilde{\kappa}_t$~~&$a_0$~~&$a_1$& $\delta$~~ \\ 
\hhline{|=====|} 
%\hhline{|--------|}
$1$ & $-1$~~~ & $0.1591 \pm 0.0006$ & $-0.0146 \pm 0.0008$ & $0.73 \pm 0.06$ \\[0.6mm]
\hline
$1$ & $0$ & $0.1594 \pm 0.0007$ & $-0.0190 \pm 0.0008$ & $0.005 \pm 0.06\,\,$ \\[0.6mm]
\hline
$1$ & $1$ & $0.1592 \pm 0.0006$ & $-0.0136 \pm 0.0008 $ & $\,-0.77 \pm 0.07\quad$  \\[0.6mm]
\hline
$0$ & $1$ & $0.1591 \pm 0.0006$ & ~~$\,0.0113 \pm 0.0008$ & $0.09 \pm 0.08$ \\
\hhline{|=====|}
%\hhline{|--------|}
\end{tabular}
\end{center} 
\end{table}
\par From Tables \ref{table3} and \ref{table4} we see that the
parameter $\delta$ is sensitive not only to the modulus of $\kpt$ but
also to its sign, as expected if Eq.~(\ref{eq26}) is taken into
account. The phase shift $\delta$ for the distribution of the angle
$\Delta\phi_1$ appears to exhibit a slightly higher sensitivity than
that obtained from the $\Delta\phi_2$-distribution, albeit both are
still compatible within the statistical uncertainties. However, it is
important to stress that the fits of the $\Delta\phi_2$-distributions
give allways smaller values for $\chi^2/\mathrm{d.o.f}$.
%%%%%%%%%%%%%%%%%%%%%%%%% FIGURA 7 %%%%%%%%%%%%%%%%%%%%%%%%%
\begin{center}
%\vspace*{-4mm}
\begin{figure}[H]
%\centering
\hspace*{-0.52cm}
\subfloat{\includegraphics[scale=0.45]{TP1_11_nocuts_nuevo.pdf}}
\hspace*{-0.006\textwidth}
\subfloat{\includegraphics[scale=0.45]{TP1_1-1_nocuts_nuevo.pdf}} \\
\hspace*{-0.52cm}
%\hspace*{0.03\textwidth}
\subfloat{\includegraphics[scale=0.45]{TP2_11_nocuts_nuevo.pdf}}
\hspace*{-0.006\textwidth}
\subfloat{\includegraphics[scale=0.45]{TP2_1-1_nocuts_nuevo.pdf}}
\caption{Angular distributions $d\sigma/(\sigma
  d\Delta\phi_1(n_t,n_{\tbar}))$ (top) and $d\sigma/(\sigma
  d\Delta\phi_2(n_t,n_{\tbar}))$ (bottom) associated with the TPs
  $\epsilon_1=\TPa$ and $\epsilon_2=\TPb$ respectively for the
  $\mathrm{CP}$-mixed cases $\kp=\kpt=1$ (left) and $\kp=-\kpt=1$
  (right) when the $\Delta R_{\ell\ell}$ cut is turned off. Also, the
  corresponding fit curves are displayed in red.}
\label{fig7}
\end{figure}
\end{center}
\par Concerning the angular distributions that can be related to the
combination $\epsilon_4$, we have analized the
$\Delta\phi(n_t,n_{\tbar})$ distribution in the $Q$ rest frame with
$H$ in the $z$-axis for various values of $\kp$ and $\kpt$. We have
found that these distributions are not described by Eq.~(\ref{eq27})
and their range of variation is larger than that of the distributions
displayed in Figs.~\ref{fig5} and \ref{fig6}. Unlike the distributions
related to $\epsilon_1$-$\epsilon_3$, those arising from $\epsilon_4$
exhibit small changes in their shapes for all the considered
hypotheses and for this reason we have not included the corresponding
plots here. However, the larger range of variation of the $\epsilon_4$
distributions leads to higher values for the asymmetry (as can be sees
from Tables \ref{table1} and \ref{table2}) even when the changes in
the respective shapes are smaller than in the case of the
distributions described by Eq.~(\ref{eq27}).  \par
%%%%%%%%%%%%%%%%%%%%%%%%%%%%%%%%%%%%%%%%%%%%%%%%%%%%%%%%%%%%%%%%%%%%%%%%%%
\subsection{Mean value}
\label{sec3.3}
We turn now to consider the last type of observables constructed from
the TPs that are sensitive to $\kpt$, the mean value. Given a certain
TP, we define its mean value in the following manner,
%
\beq
\label{eq28}
\langle \epsilon \rangle = \frac{\int\epsilon\, \{d\sigma (pp\to b\,\ell^+\nu_{\ell}\,\bbar\,\ell^-\bar{\nu}_{\ell}H)/ d\Phi\}\,d\Phi}{\int \{d\sigma (pp\to b\,\ell^+\nu_{\ell}\,\bbar\,\ell^-\bar{\nu}_{\ell}H)/ d\Phi\}\,d\Phi},
\eeq
%
where $\Phi$ is the Lorentz invariant phase space corresponding to the
final state
$b\,\ell^+\nu_{\ell}\,\bbar\,\ell^-\bar{\nu}_{\ell}H$. From
Eq.~(\ref{eq20}) we see that only the terms linear in $\kp$ and $\kpt$
will contribute to the mean value, so that we expect this observable
to be sensitive not only to the value but also to the relative sign of
the couplings.\par The results obtained for the TPs $\epsilon_1=
\epsilon(t,\tbar,n_t,n_{\tbar})$, $\epsilon_2=
\epsilon(Q,\tbar,n_t,n_{\tbar})$ and $\epsilon_3
=\epsilon(Q,t,n_t,n_{\tbar})$ introduced in Sec.~\ref{sec2} are
displayed in Table \ref{table5}, where we list the deviation of the
mean values from the SM case ($\kpt = 0$) in terms of the
corresponding statistical uncertainty of the estimator of $\langle
\epsilon \rangle$, $\bar{\epsilon}$.
\renewcommand{\arraystretch}{1.6}
\begin{table}[H]
\caption{Mean values obtained for the TPs $\epsilon_{1,2,3}$ for the
  SM case and two $\mathrm{CP}$-mixed cases with opposite sign in the
  pseudoscalar coupling. The values are obtained by using $10^5$
  simulated events.}
\label{table5}
\begin{center}
\begin{tabular}{|C{1cm}|C{1cm}||C{2cm}||C{2cm}||C{2cm}|}
%\hhline{|=====|}
\hhline{|-----|}
$\kappa_t$ & $\tilde{\kappa}_t$ & $\langle \epsilon_1 \rangle /\sigma_{\bar{\epsilon}_1}$ & $\langle \epsilon_2 \rangle /\sigma_{\bar{\epsilon}_2}$ & $\langle \epsilon_3 \rangle /\sigma_{\bar{\epsilon}_3}$ \\ 
\hhline{|=====|} 
%\hhline{|-----|}
%\vspace*{.5mm}
\renewcommand{\arraystretch}{1.0}
$1$~ & $-1$~~~ & $4.26$~ & $4.94$~ & $-5.81$~~~\\[0.6mm]
\hline
$1$ & $0$~ & $-0.91$~~~ & $-0.22$~~~ & $1.25$~\\[0.6mm]
\hline
$1$ & $1$~ & $-7.98$~~~ & $-8.83$~~~& $8.75$~\\[0.6mm]
\hhline{|=====|}
%\hhline{|-----|}
\end{tabular}
\end{center} 
\end{table}
We see that the three observables are capable of disinguishing the SM
case from both $\mathrm{CP}$-mixed cases. Further, the observables are
sensitive to the sign of $\kpt$ and the two $\mathrm{CP}$-mixed cases
are also clearly disentangled. The observables $\langle \epsilon_2
\rangle$ and $\langle \epsilon_3 \rangle$ appear to be slightly more
sensitive than $\langle \epsilon_1 \rangle$. On the other hand, the
mean value for the combination $\epsilon_4$ introduced in
Sec.~\ref{sec3.1} gives slightly smaller values than those listed in
Table \ref{table5}, with $-4.32, 1.11$ and $7.23$ for the cases
$(\kp=1,\kpt=-1,0,1)$ respectively. As with the asymmetry, the
hypotheses $\mathrm{CP}$-even and $\mathrm{CP}$-odd cannot be
distinguished by the mean value that is also linear in both $\kp$ and
$\kpt$ (see Eqs.~(\ref{eq20}) and (\ref{eq28})). By taking into
account the results given in Sec.~\ref{sec3.1}, we can conclude that
the sensitivity to the NP contribution is smaller for the mean values
than for the asymmetries of the TPs under consideration.
%%%%%%%%%%%%%%%%%%%%%%%%%%%%%%%%%%%%%%%%%%%%%%%%% 
\section{{\boldmath $\mathrm{CP}$}-odd observables not depending on \MakeLowercase{{\boldmath $t$}} and \MakeLowercase{{\boldmath $\tbar$}} spin vectors}
\label{sec4}
So far we have considered three TPs involving the momenta $t,\tbar$
and $Q$ and the spin vectors $n_t$ and $n_{\tbar}$ given in
Eqs.~(\ref{eq3})-(\ref{eq4}). In fact, we have described the general
form of the differential cross-section in terms of these vectors in
Eq.~(\ref{eq20}). Here, we will take into account another
possibilities for the choice of the vectors from which the
$\mathrm{CP}$-odd observables can be constructed. From the definitions
in Eqs.~(\ref{eq3})-(\ref{eq4}), we see that the TPs
$\epsilon_{1,2,3}$ can be written as follows,
%
\beq
\label{eq29}
\TPa = \frac{m^2_t}{(t\cdot \ell^+)(\tbar\cdot \ell^-)}\,\epsilon(t,\tbar,\ell^-\!,\ell^+),
\eeq
%
\vspace*{1mm}
\beq
\label{eq30}
\TPb = \frac{m^2_t}{(t\cdot \ell^+)(\tbar\cdot \ell^-)}\left(\epsilon(t,\tbar,\ell^-\!,\ell^+)+\epsilon(H,\tbar,\ell^-\!,\ell^+)+\frac{(t\cdot\ell^+)}{m^2_t}\epsilon(H,\tbar,t,\ell^-\!)\right)
\eeq
%
\vspace*{1mm}
\beq
\label{eq31}
\TPc = \frac{m^2_t}{(t\cdot \ell^+)(\tbar\cdot \ell^-)}\left(-\epsilon(t,\tbar,\ell^-\!,\ell^+)+\epsilon(H,t,\ell^-\!,\ell^+)+\frac{(\tbar\cdot\ell^-)}{m^2_t}\epsilon(H,\tbar,t,\ell^+\!)\right).
\vspace*{2mm}
\eeq
%
The above equations express the TPs studied in the last sections as
linear combination of TPs involving the momenta $t,\tbar,H,\ell^+$ and
$\ell^-$, with coefficients that are functions of phase space
variables. These five momenta give rise to five TPs whose sensitivity
can also be tested by means of the observables introduced in
Secs.~\ref{sec3.1}-\ref{sec3.3}. In the first place, we have found
that TPs not including both lepton and anti-lepton momenta present a
negligible sensitivity, and then we will concentrate here on the
results obtained for the remaining TPs,
$\epsilon_5\equiv\epsilon(t,\tbar,\ell^-\!,\ell^+),
\epsilon_6\equiv\epsilon(H,t,\ell^-\!,\ell^+)$ and $\epsilon_7\equiv
\epsilon(H,\tbar,\ell^-\!,\ell^+)$. The Tables \ref{table6} and
\ref{table7} summarize the results for these TPs.
\renewcommand{\arraystretch}{1.6}
\begin{table}[H]
\caption{Mean values obtained for the TPs $\epsilon_{5,6,7}$ for the
  SM case and two $\mathrm{CP}$-mixed cases with opposite sign in the
  pseudoscalar coupling. The values correspond to $10^5$ simulated
  events.}
\label{table6}
\begin{center}
\begin{tabular}{|C{1cm}|C{1cm}||C{2cm}||C{2cm}||C{2cm}|}
%\hhline{|=====|}
\hhline{|-----|}
$\kappa_t$ & $\tilde{\kappa}_t$ & $\langle \epsilon_5 \rangle /\sigma_{\bar{\epsilon}_5}$ & $\langle \epsilon_6 \rangle /\sigma_{\bar{\epsilon}_6}$ & $\langle \epsilon_7 \rangle /\sigma_{\bar{\epsilon}_7}$ \\ 
\hhline{|=====|} 
%\hhline{|-----|}
%\vspace*{.5mm}
\renewcommand{\arraystretch}{1.0}
$1$~ & $-1$~~~ & $3.98$~ & $-1.96$~~~ & $1.69$~ \\[0.6mm]
\hline
$1$ & $0$~ & $-0.43$~~~ & $1.25$~ & $0.74$~ \\[0.6mm]
\hline
$1$ & $1$~ & $-6.76$~~~ & $3.46$~ & $-3.29$~~~~\\[0.6mm]
\hhline{|=====|}
%\hhline{|-----|}
\end{tabular}
\end{center} 
\end{table}
\renewcommand{\arraystretch}{1.4}
\begin{table}[H]
\caption{Asymmetries for the TPs $\epsilon_{5,6,7}$ for the SM case
  and the two $\mathrm{CP}$-mixed cases defined by $\kp=1,\kpt=\pm
  1$. The values correspond to $10^5$ simulated events.}
\label{table7}
\begin{center}
\begin{tabular}{|C{1cm}|C{1cm}||C{2cm}|C{2cm}||C{2cm}|C{2cm}||C{2cm}|C{2cm}|}
%\begin{tabular}{|c|r||r|c||r|c||r|c|}
\hhline{|========|}
%\hhline{|--------|}
$\kappa_t$&$\tilde{\kappa}_t$~~&$\mathcal{A}(\epsilon_5)$~~&$\mathcal{A}(\epsilon_5)/\sigma_{\mathcal{A}}$& $\mathcal{A}(\epsilon_6)$~~&$\mathcal{A}(\epsilon_6)/\sigma_{\mathcal{A}}$&$\mathcal{A}(\epsilon_7)$~~&$\mathcal{A}(\epsilon_7)/\sigma_{\mathcal{A}}$  \\ 
\hhline{|========|} 
%\hhline{|--------|}
$1$ & $-1$~~~ & $0.0315$~ & $10.0$~ & $-0.0134$~~~ & $-4.2$~~~ & $0.0111$~ & $3.5$~\\[0.6mm]
\hline
$1$ & $0$ & $-0.0021$~~~ & $-0.7$~~~ & $-0.0011$~~~ & $-0.3$~~~ & $0.0009$~& $0.3$~\\[0.6mm]
\hline
$1$ & $1$ & $-0.0379$~~~ & $-12.0$~~~ & $0.0143$~ & $4.5$~ & $-0.0137$~~~ & $\,-4.3$~~~  \\[0.6mm]
\hhline{|========|}
%\hhline{|--------|}
\end{tabular}
\end{center} 
\end{table}
\par
%
We see that the TP $\epsilon_5$ gives rise to asymmetries and mean
values that are clearly higher than those obtained from $\epsilon_6$
and $\epsilon_7$. This is in contrast to the TPs $\epsilon_{1,2,3}$,
for which the asymmetries and mean values are comparable (see Tables
\ref{table1} and \ref{table5}). We also note that the asymmetry for
$\epsilon_5$ is exactly the same as for $\epsilon_1$ as was expected
from Eq.~(\ref{eq29}) since the proportionality factor relating them
is positive definite. Regarding the mean values, we see by comparing
Tables \ref{table5} and \ref{table6} that the TPs $\epsilon_{1,2,3}$
appear to have a higher sensitivity to the pseudoscalar coupling than
$\epsilon_{5,6,7}$.  \par It is important to mention that in the
$t\tbar$ rest frame the sign of the TP $\epsilon_5$ is defined through
the angle $\Delta\phi_{\ell^-\ell^+}$ (recall the discussion on
Eq.~(\ref{eq28})), which is the angular difference between the
projections of the leptons momenta onto the plane perpendicular to
$\vec{\tbar}$. By a similar argument to that discussed at the
beginning of Sec.~\ref{sec3.2}, we can construct an associated angular
distribution that, being constrained by the $\mathcal{A}(\epsilon_5)$,
will be also sensitive to the sign of the pseudoscalar coupling. This
angular variable is nothing but that proposed in \cite{Ellis} as a
useful $\mathrm{CP}$-odd observable. Moreover, it is shown in
\cite{Ellis} that this angular distribution follows the functional
form given in Eq.~(\ref{eq27}). Due to the fact that this distribution
is constrained by $\mathcal{A}(\epsilon_5)$ which is equal to
$\mathcal{A}(\epsilon_1)$ and smaller than
$\mathcal{A}(\epsilon_{2,3})$, the corresponding shifts ($\delta$)
obtained for different values of $\kpt$ are expected to be of the same
order than those exhibited by the $\Delta\phi_1(n_t,n_{\tbar})$
distribution and somewhat smaller than those observed in the
$\Delta\phi_2(n_t,n_{\tbar})$ distribution.  \par In analogy to the
combination of TPs considered in Sec.~\ref{sec3}, we have found a
combination of the TPs $\epsilon_{5,6,7}$ for which the asymmetry is
enhanced with respect to $\epsilon_5$-$\epsilon_7$,
%
\beq
\label{eq32}
\epsilon_8 = 2\epsilon_5 -\epsilon_6 +\epsilon_7 = \epsilon(t+\tbar+H,t-\tbar,\ell^+,\ell^-).
%2\epsilon(t,\tbar,\ell^-,\ell^+)+\epsilon(H,\tbar,\ell^-,\ell^+)-\epsilon(H,t,\ell^-,\ell^+)
\eeq
%
We see from Eq.~(\ref{eq32}) that in the $t\tbar H$ rest frame
$\epsilon_8=M_{t\tbar H}(\vec{t}-\vec{\tbar})\cdot (\vec{\ell}^+\!
\times \vec{\ell}^-)$, where $M_{t\tbar H}$ is the invariant mass of
the system $t\tbar H$. Hence, in this reference frame the sign of the
combination $\epsilon_8$ is set by the quantity
$(\vec{t}-\vec{\tbar})\cdot (\vec{\ell}^+\! \times
\vec{\ell}^-)$. Taking into account Eqs.~(\ref{eq24}) and (\ref{eq32})
along with the definition $Q=(t+\tbar+H)/2$, we see that the only
relevant difference between $\epsilon_4$ and $\epsilon_8$ is that in
the latter the spin vectors $n_t$ and $n_{\tbar}$ have been replaced
by the momenta of the leptons $\ell^+$ and $\ell^-$ respectively. The
values obtained for $\mathcal{A}(\epsilon_8)$ are shown in Table
\ref{table8}. Compared to the TPs $\epsilon_5$-$\epsilon_7$ and
$\epsilon_1, \epsilon_3$ (see Tables \ref{table1} and \ref{table7}),
the chosen combination exhibits a slightly higher sensitivity in the
asymmetry for resolving the $\mathrm{CP}$-mixed cases. This is not
true in the case of the TPs $\epsilon_2$ and $\epsilon_4$, for which
the asymmetry is larger (see Tables \ref{table1}, \ref{table2} and
\ref{table8}). In particular, the comparison between $\epsilon_4$ and
$\epsilon_8$ indicates that the use of the momenta of the leptons
instead of the spin vectors produces a decrease in the sensitivity of
the asymmetry. \par From the TPs $\epsilon_5$-$\epsilon_7$ one can
derive various angular distributions in the same manner we have
discussed in Sec.~\ref{sec3.2} for $\epsilon_1$-$\epsilon_3$. Of
course, in this case the corresponding angle will be defined in terms
of the momenta of the leptons instead of using the spin vectors. These
distributions have the same behaviour than those derived from
$\epsilon_1$-$\epsilon_3$, but only the shift obtained in the case of
$\epsilon_5$ is compatible to the values given in Tables for
$\epsilon_1$-$\epsilon_3$. The shifts resulting from the fit of the
distributions related to $\epsilon_6$ and $\epsilon_7$ are
smaller. Taking into account these facts we do not give further
details of the angular distributions for the TPs discussed in this
section.\par The mean value of $\epsilon_8$ for the hypotheses under
consideration is comparable with the values listed in Table
\ref{table6} for $\epsilon_5$. Concerning the associated angular
distributions, their range of variation is larger than in the case of
the distributions related to $\epsilon_5$-$\epsilon_7$ but exhibit
smaller changes in their shapes for the different hypothesis and for
this reason we do not include here the corresponding plots.
\begin{table}[H]
\caption{Asymmetry for the TP $\epsilon_{8}$ for the SM case and the
  two $\mathrm{CP}$-mixed cases defined by $\kp=1,\kpt=\pm 1$. The
  values are obtained with $10^5$ simulated events.}
\label{table8}
\begin{center}
\begin{tabular}{|C{1cm}|C{1cm}||C{2cm}|C{2cm}|}
%\begin{tabular}{|c|r||r|c||r|c||r|c|}
\hhline{|====|}
%\hhline{|--------|}
$\kappa_t$&$\tilde{\kappa}_t$~~&$\mathcal{A}(\epsilon_8)$~~&$\mathcal{A}(\epsilon_8)/\sigma_{\mathcal{A}}$ \\ 
\hhline{|====|} 
%\hhline{|--------|}
$1$ & $-1$~~~ & $0.0331$~ & $10.5$~ \\[0.6mm]
\hline
$1$ & $0$ & $0.0023$~ & $0.7$~ \\[0.6mm]
\hline
$1$ & $1$ & $-0.0403$~~~ & $\,-12.7$~~~~ \\[0.6mm]
\hhline{|====|}
%\hhline{|--------|}
\end{tabular}
\end{center} 
\end{table}
\section{{\boldmath{$\mathrm{CP}$}}-odd observables not depending on \MakeLowercase{{\boldmath $t$}} and \MakeLowercase{{\boldmath $\tbar$}} momenta}
\label{sec5}
All the observables discussed in the above sections involve the
momenta of the top and anti-top quarks thus requiring the full
reconstruction of the kinematics of the individual $t$ and $\tbar$
systems in order to be measured. Although challenging due to the
presence of two neutrinos in the final state, this can be in principle
done by applying a kinematic reconstruction method such that the
neutrino weighting technique \cite{atlasconf,atlascharge}. Another
possibility is to define observables that are not dependent on the $t$
and $\tbar$ momenta but make use of the quarks $b$ and $\bbar$. In
order to construct such observables we will modify here the most
sensitive observables studied in Secs.~\ref{table3} and \ref{sec4},
namely the combinations $\epsilon_4$ and $\epsilon_8$
respectively. Let us first consider the combination $\epsilon_8$
defined in Eq.~(\ref{eq32}) and replace the top and antitop momenta by
the bottom and antibottom momenta respectively. Thus, we define
%
\beq
\label{eq33}
\epsilon_9 = \epsilon(b+\bbar+H,b-\bbar,\ell^+,\ell^-).
\eeq
%
Note that now the sign of the TP $\epsilon_9$ is set by the sign of
the quantity $(\vec{b}-\vec{\bbar})\cdot(\vec{\ell}^+\!\times
\vec{\ell}^-)$ in the $b\bbar H$ rest frame. This quantity is in fact
used in \cite{Guadagnoli} but in the lab frame in order to define a
$\mathrm{CP}$-odd observable that only depends on lab frame
variables. The values of the asymmetry for $\epsilon_9$ are listed in
Table \ref{table9}. By comparing Tables \ref{table8} and \ref{table9}
we see that the use of the $b,\bbar$ instead of $t,\tbar$ leads to a
decrese in the sensitivity of the asymmetry by $\sim 5\sigma$ for
$\kp=1,\kpt=\pm 1$. However, the observable is still capable of
discriminating not only between the two $\mathrm{CP}$-mixed hypotheses
but also between these and the SM case.
%\newpage
We proceed now in a similar way with the combination $\epsilon_4$. By
using Eq.~(\ref{eq24}) along with the definitions of the spin vectors
in Eqs.~(\ref{eq3})-(\ref{eq4}), we can write
%
\beq
\label{eq34}
\epsilon_4 = \frac{m^2_t}{(t\cdot\ell^+)\cdot(\tbar\cdot\ell^-)}\,\epsilon(Q,t-\tbar,\ell^-,\ell^+)+\frac{1}{(t\cdot\ell^+)}\,\epsilon(Q,t,\ell^+,\tbar)-\frac{1}{(\tbar\cdot \ell^-)}\,\epsilon(Q,\tbar,t,\ell^-).
\eeq
%
Since the asymmetry is not changed by an overall positive definite
factor, we will concentrate \DIFdelbegin \DIFdel{in }\DIFdelend \DIFaddbegin \DIFadd{on }\DIFaddend the following combination arising from
the expression in Eq.~(\ref{eq34}),
%
\beq
\label{eq35}
\epsilon(Q,t-\tbar,\ell^-,\ell^+)+\frac{(\tbar\cdot \ell^-)}{m^2_t}\epsilon(Q,t,\ell^+,\tbar)-\frac{(t\cdot\ell^+)}{m^2_t}\epsilon(Q,\tbar,t,\ell^-),
\eeq
%
\begin{table}[H]
\caption{Asymmetry for the TP $\epsilon_{9}$ for the SM case and the
  two $\mathrm{CP}$-mixed cases defined by $\kp=1,\kpt=\pm 1$. The
  values are obtained with $10^5$ simulated events.}
\label{table9}
\begin{center}
\begin{tabular}{|C{1cm}|C{1cm}||C{2cm}|C{2cm}|}
%\begin{tabular}{|c|r||r|c||r|c||r|c|}
\hhline{|====|}
%\hhline{|--------|}
$\kappa_t$&$\tilde{\kappa}_t$~~&$\mathcal{A}(\epsilon_9)$~~&$\mathcal{A}(\epsilon_9)/\sigma_{\mathcal{A}}$ \\ 
\hhline{|====|} 
%\hhline{|--------|}
$1$ & $-1$~~~ & $0.0171$~ & $5.4$~ \\[0.6mm]
\hline
$1$ & $0$ & $0.0010$~ & $0.3$~ \\[0.6mm]
\hline
$1$ & $1$ & $-0.0247$~~~ & $\,-7.8$~~~~ \\[0.6mm]
\hhline{|====|}
%\hhline{|--------|}
\end{tabular}
\end{center} 
\end{table}
\noindent
%
and instead of replacing $t$ and $\tbar$ directly by $b$ and $\bbar$,
we use their visible parts, namely $b+\ell^+$ and $\bbar +\ell^-$
respectively. This results in the following definition
%
\beq
\label{eq36}
\epsilon_{10}=\epsilon(\tilde{Q},c_{b\bbar}\,,\ell^-,\ell^+)-w_1\,\epsilon(\tilde{Q},b,\bbar,\ell^+)+w_2\,\epsilon(\tilde{Q},b,\bbar,\ell^-),
\eeq
%
where $\tilde{Q}\equiv (b+\ell^+\!+\bbar +\ell^-)/2$ stands for the
visible part of $Q$, $c_{b\bbar}=(1-w_1)\,b-(1-w_2)\,\bbar$ and the
weights $w_{1,2}$ are given by $(\bbar\cdot \ell^-)/m^2_t $ and
$(b\cdot \ell^+)/m^2_t$ respectively. Also, the contribution
$m^2_{\ell}/m^2_t$ has been neglected both in $w_1$ and in $w_2$. Note
that if we set $w_1=w_2=0$, the combination $\epsilon_{10}$ reduces to
$\epsilon_9 /2$ and $\mathcal{A}(\epsilon_{10})$ becomes equal to
$\mathcal{A}(\epsilon_9)$. The results obatined for the asymmetry of
$\epsilon_{10}$ are given in Table \ref{table10}. By comparing Tables
\ref{table2} and \ref{table10} we see again that the sensitivity of
the asymmetry decreases when $t$ and $\tbar$ are not included in the
TP. Nevertheless, the combination $\epsilon_{10}$ remains a useful
observable for discriminating the $\mathrm{CP}$ nature of the Higgs
boson, with the corresponding asymmetry having a sensitivity even
higher than that of $\epsilon_9$, as can be checked by taking into
account Tables \ref{table9} and \ref{table10}. More precisely, the
separation between the $\mathrm{CP}$-mixed hypotheses is enhanced by
about $3\sigma$. This improvement in the asymmetry of $\epsilon_{10}$
with respect to $\epsilon_9$ may be due to two facts. In the first
place, as was pointed out in Sec.~\ref{sec4} when comparing the TPs
$\epsilon_4$ and $\epsilon_8$, the asymmetry appears to be higher when
the spin vectors are used instead of the lepton momenta and we see
from Eqs.~(\ref{eq33}) and (\ref{eq36}) that $\epsilon_{10}$, being
obtained from $\epsilon_4$, contains the information on the spin
vectors, as opposed to $\epsilon_9$ that depends directly on the
lepton momenta because it is derived from $\epsilon_8$. In the second
place, in order to obtain $\epsilon_{10}$ we have replaced in the
expression for $\epsilon_4$ the top and antitop momenta by their
visible part, while in the case of $\epsilon_9$ the bottom and
antibottom momenta have been used.
\begin{table}[H]
\caption{Asymmetry for the TP $\epsilon_{10}$ for the SM case and the
  two $\mathrm{CP}$-mixed cases defined by $\kp=1,\kpt=\pm 1$. The
  values are obtained by using $10^5$ simulated events.}
\label{table10}
\begin{center}
\begin{tabular}{|C{1cm}|C{1cm}||C{2cm}|C{2cm}|}
%\begin{tabular}{|c|r||r|c||r|c||r|c|}
\hhline{|====|}
%\hhline{|--------|}
$\kappa_t$&$\tilde{\kappa}_t$~~&$\mathcal{A}(\epsilon_{10})$~~&$\mathcal{A}(\epsilon_{10})/\sigma_{\mathcal{A}}$ \\ 
\hhline{|====|} 
%\hhline{|--------|}
$1$ & $-1$~~~ & $-0.0213$~~~ & $-6.7$~~~ \\[0.6mm]
\hline
$1$ & $0$ & $0.0031$~ & $1.0$~ \\[0.6mm]
\hline
$1$ & $1$ & $0.0300$~ & $9.5$~ \\[0.6mm]
\hhline{|====|}
%\hhline{|--------|}
\end{tabular}
\end{center} 
\end{table}
\par
%
By using our set of simulated events we have also tested, for
comparison purposes, the asymmetry of the lab frame observable given
in \cite{Guadagnoli}. We have obtained that this observable appears to
be slightly less sensitive than the combination $\epsilon_{10}$,
giving rise to a separation between the $\mathrm{CP}$-mixed hypotheses
smaller by about $1.4\sigma$.
\section{Experimental Feasibility}
\label{sec6}
%
By considering the mild selection cuts introduced in Sec.~\ref{sec3},
the SM cross section for $\ppprocess$, $\ell=e,\mu$ at
$14\,\mathrm{TeV}$ is $\sim 15.3\,\mathrm{fb}$ and hence the number of
events expected within the context of the HL-LHC is $\sim
15.3\,\mathrm{fb} \times 3000\,\mathrm{fb}^{-1} = 4.59\times
10^4$. This number is expected to be even larger in the case of
$\kp=1,\kpt\neq 0$ since the corresponding cross section is then
higher than the SM one. By taking into account NLO corrections to the
production process and considering a $K$-factor around $1.2$
\cite{Dawson,Beenakker,Dittmaier}, the number of events can be raised
up to $\sim 5.49 \times 10^4$. On the other hand, additional cuts as
well as the efficiency related to the reconstruction of particles
momenta will contribute to decrease this number. For instance, in
order to obtain the asymmetry of $\epsilon_4$, the $t$ and $\tbar$
momenta need to be reconstructed. This is challenging not only due to
the presence of two neutrinos in the final state which escape the
detector undetected but also because final state objects need to be
asocciated with the respective parent quark \cite{atlasconf}. As was
already mentioned in Sec.~\ref{sec5}, a possibility is to use the
neutrino weighting technique along with the kinematic equations
arising from kinematic constraints related to the top and $W$ masses
as well as from the energy-momentum conservation at each of the decay
vertices involved in the process. Within the context of $t\tbar$ this
procedure has been used, for instance, in order to obtain measurements
of spin correlation \cite{atlasconf} and charge asymmetry
\cite{atlascharge}. Also, events reconstructed with this technique has
been used in \cite{dosSantos} for analyzing angular distributions that
are useful for discriminating the signal from the backgrounds in
$t\tbar H (H\rightarrow b\bbar)$. In all these cases the corresponding
efficiency in the reconstruction of the momenta is up to $\sim 80
\%$.\par Based on the discussion given in the previous paragraph, we
have simulated sets of $5\times 10^4, 1\times 10^4$ and $5\times 10^3$
events and recalculated the most sensitive observable, namely
$\mathcal{A}(\epsilon_4)$, for each case. The results are displayed in
Table \ref{table11}, where it can be seen that for a number of events
close to that roughly estimated above within the context of the
HL-LHC, the observable is still sensitive to $\kpt$ allowing to
separate the $\mathrm{CP}$-mixed cases by $19\sigma$. As expected, the
sensitivity worsen as the number of events is reduced, but even with
$5\times 10^3$ events the separation between the $\mathrm{CP}$-mixed
hypotheses under consideration is around $6.5\sigma$. \par Although
the combination $\epsilon_{10}$ discussed in Sec.~\ref{sec5} avoids
the problem of reconstructing the top and antitop momenta, we have
also considered the respective asymmetry obtained for more
conservative numbers of events. In Table \ref{table12} we show the
results for $\mathcal{A}(\epsilon_{10})$ when the number of events is
reduced from $5\times 10^4$ to $1\times 10^4$. We see in this case
that even with $1\times 10^4$ simulated events the observable is
capable to discriminate the $\mathrm{CP}$-mixed cases by
$5.6\sigma$.\par Finally, it is important to mention that a realistic
analysis of the sensistivity of the observables discussed in this
paper requires the study of the impact of the backgrounds as well as
the hadronization of the quarks in the final state and the effects of
the detector. If we consider the dominant decay mode of the higgs,
$H\rightarrow b\bbar$, in order to maximize the cross section of the
process, the signature is given by $4$ $b$-jets, two leptons and
missing energy, while the main background arise from the production of
$t\tbar$ in asocciation with additional jets, being the dominant
source the production of $t\tbar + b\bbar$. By applying a small set of
cuts it is shown in \cite{chinos} that the
\begin{table}[H]
\caption{Asymmetry for the TP $\epsilon_4$ obtained by using $5\times
  10^4, 1 \times 10^4$ and $5\times 10^3$ simulated events for the SM
  case and the two $\mathrm{CP}$-mixed cases defined by
  $\kp=1,\kpt=\pm 1$. }
\label{table11}
\begin{center}
\begin{tabular}{|C{1cm}|C{1cm}||C{2cm}|C{2cm}||C{2cm}|C{2cm}||C{2cm}|C{2cm}|}
%\begin{tabular}{|c|r||r|c||r|c||r|c|}
\hhline{|========|}
%\hhline{|--------|}
\multirow{2}{*}{$\kappa_t$} & \multirow{2}{*}{$\tilde{\kappa}_t$} & \multicolumn{2}{c||}{$N_{\mathrm{ev}}=5\times 10^4$} & \multicolumn{2}{c||}{$N_{\mathrm{ev}}=1\times 10^4$} & \multicolumn{2}{c|}{$N_{\mathrm{ev}}=5\times 10^3$} \\ \cline{3-8}
& & $\mathcal{A}(\epsilon_4)$~~&$\mathcal{A}(\epsilon_4)/\sigma_{\mathcal{A}}$ &  $\mathcal{A}(\epsilon_4)$~~&$\mathcal{A}(\epsilon_4)/\sigma_{\mathcal{A}}$ &  $\mathcal{A}(\epsilon_4)$~~&$\mathcal{A}(\epsilon_4)/\sigma_{\mathcal{A}}$ \\
\hhline{|========|} 
%\hhline{|--------|}
$1$ & $-1$~~~ & $-0.0405$~~~ & $-9.1$~~~ & $-0.0426$~~~ & $-4.3$~~~ & $-0.0496$~~~ & $-3.5$~~~ \\[0.6mm]
\hline
$1$ & $0$ & $0.0004$~ & $0.1$~ & $-0.0084$~~~ & $-0.8$~~~ & $-0.0004$~~~ & $-0.03$~~~ \\[0.6mm]
\hline
$1$ & $1$ & $0.0443$~ & $9.9$~ & $0.0434$~ & $4.2$~ & $\,0.0420$~ & $3.0\,\,$ \\
\hhline{|========|}
%\hhline{|--------|}
\end{tabular}
\end{center} 
\end{table}
\begin{table}[H]
\caption{Asymmetry for the TP $\epsilon_{10}$ in the SM case and the
  two $\mathrm{CP}$-mixed cases defined by $\kp=1,\kpt=\pm 1$ when the
  number of simulated events is reduced from $5\times 10^4$ to $1
  \times 10^4$.}
\label{table12}
\begin{center}
\begin{tabular}{|C{1cm}|C{1cm}||C{2cm}|C{2cm}||C{2cm}|C{2cm}|}
%\begin{tabular}{|c|r||r|c||r|c||r|c|}
\hhline{|======|}
%\hhline{|--------|}
\multirow{2}{*}{$\kappa_t$} & \multirow{2}{*}{$\tilde{\kappa}_t$} & \multicolumn{2}{c||}{$N_{\mathrm{ev}}=5\times 10^4$} & \multicolumn{2}{c|}{$N_{\mathrm{ev}}=1\times 10^4$} \\ \cline{3-6}
& & $\mathcal{A}(\epsilon_{10})$~~&$\mathcal{A}(\epsilon_{10})/\sigma_{\mathcal{A}}$ &  $\mathcal{A}(\epsilon_{10})$~~&$\mathcal{A}(\epsilon_{10})/\sigma_{\mathcal{A}}$ \\
\hhline{|======|} 
%\hhline{|--------|}
$1$ & $-1$~~~ & $-0.0270$~~~ & $-6.0$~~~ & $-0.0184$~~~ & $-1.8$~~~  \\[0.6mm]
\hline
$1$ & $0$ & $0.0022$~ & $0.5$~ & $-0.0086$~~~ & $-0.9$~~~  \\[0.6mm]
\hline
$1$ & $1$ & $0.0313$~ & $7.0$~ & $0.0380$~ & $3.8\,$ \\
\hhline{|======|}
%\hhline{|--------|}
\end{tabular}
\end{center} 
\end{table}
\noindent
%
signal to background ratio is largely improved. On the other hand, a
rigorous treatment of the backgrounds and their impact with respect to
the signal is performed in \cite{atlasger} by using $20.3\,
\mathrm{fb}^{-1}$ of data at $\sqrt{s}=8\,\mathrm{TeV}$.  \par The
results shown in tables \ref{table11} and \ref{table12} reveal that
with $5\times 10^3$ and $1\times 10^4$ events respectively the
observables $\mathcal{A}(\epsilon_4)$ and $\mathcal{A}(\epsilon_{10})$
are still useful for testing $\kpt$. Without considering the lack of
events related to the experimental analysis, these numbers of events
correspond to a luminosity around $\sim
300\,$-$\,600\,\mathrm{fb}^{-1}$ for the SM and even smaller for the
$\mathrm{CP}$-mixed cases due to the larger cross section. This range
of luminosities is in principle achievable in the short term by the
LHC. We note that in order to be fully conclusive about the required
luminosity, it is important to include the effects of hadronization,
detector resolution, reconstruction eficciencies and so
forth. However, this kind of analysis is out of the scope of this
paper.
%%%%%%%%%%%% with Cuts
%51.46/17 -0.02 +- 0.08
%45.48/17  -1.09 +- 0.07
%65.60/17   0.96 +- 0.08
%28.84/17   0.02 +- 0.07
%%%%%%%
%9.13/17     0.01 +- 0.06
%9.05/17     -0.98 +- 0.07
%19.7/17     0.80 +- 0.07
%16.59/17    0.06 +- 0.07
%
%%%%%%%%%%%% without Cuts
%16.36/17    0.0022 +-0.06
%15.25/17    -0.82 +- 0.07
%12.80/17    0.81 +- 0.07
%19.38/17    0.11 +- 0.08
%%%%%%
%7.45/17    0.005 +- 0.06
%18.17/17   -0.77 +- 0.07
%9.59/17    0.73 +- 0.06
%12.39/17   0.09 +- 0.08
%%%%%%%%%%%%%%%%%%%%%%%%%%%%%%%%%%%%%%%%%%%%%%%%%
\section{Conclusions}
\label{sec7}
%
In this paper we have presented a collection of $\mathrm{CP}$-odd
observables based on triple product correlations that are useful for
disentangling the relative sign between the scalar ($\kp$) and a
potential pseudoscalar ($\kpt$) top-Higgs couplings in the $t\tbar H$
production with both tops decaying leptonically. We have tested the
sensitivity of the proposed observables by considering three types of
observables: asymmetries, mean values and angular distributions.\par

Through the use of spinor techniques we have written the expression
for the differential cross section of the full process in such a
manner that the production and the decay parts are separated, although
connected by the spin vectors of the top and antitop which are given
in terms of the momenta of the leptons in the final state. Moreover,
we have indentified the terms linear in $\kp$ and $\kpt$ as those
involving TPs. Among these, we have explored the three that do not
involve the momenta of the incoming quarks/gluons and at the same time
incorporate both spin vectors: $\epsilon_1\equiv \TPa$,
$\epsilon_2\equiv \TPb$ and $\epsilon_3\equiv \TPc$.\par We have found
that $\epsilon_{1,2,3}$ allow to distinguish between the
$\mathrm{CP}$-mixed hypotheses by more than $\sim 20\sigma$ in the
case of asymmetries and $\sim 10\sigma$ in the case of mean values
when $1 \times 10^5$ simulated events are used. Furthermore, we have
shown that the angular distributions asocciated with these TPs are
also sensitive to the value of $\kp$ and $\kpt$ exhibiting a phase
shift that varies according to the values taken by these couplings. On
the other hand, by exploring TPs that incorporate the momenta of the
Higgs and the leptons instead of the spin vectors we can conclude that
the observables studied here appear to be more sensitive when the spin
vectors are used.\par

On the other hand, we have proposed a combination of the TPs
considered in the first place, $\epsilon_4\equiv
\epsilon_3-\epsilon_2$, that exhibits the highest sensitivity with the
separation between the $\mathrm{CP}$-mixed hypotheses being increased
by at least $3\sigma$ in the asymmetry for $1 \times 10^5$ events with
respect to $\mathcal{A}(\epsilon_{1,2,3})$. Again, when a similar
combination is constructed by using the leptons momenta instead of the
spin vectors ($\epsilon_8$), the sensitivity in the asymmetry is
decreased by $\sim 3\sigma$ compared to $\mathcal{A}(\epsilon_4)$ for
the same number of events, giving values compatible with those
obtained for the asymmetry of $\epsilon_2$ and $\epsilon_3$. \par

Taking into account the challenge of reconstructing the top and
antitop momenta due to the presence of two neutrinos in the final
state, we have proposed and tested two TP correlations that avoid this
difficulty. The first one is obtained by replacing the $t$ and $\tbar$
by the $b$ and $\bbar$ momenta ($\epsilon_9$), whereas the second
includes the visible part of the $t$ and $\tbar$ momenta
($\epsilon_{10}$). We have encountered that the latter is the most
sensitive leading to a discrimination of the $\mathrm{CP}$-mixed cases
under analysis by up to $\sim 16\sigma$.\par

Finally we have discussed the feasibility of the most sensitive
observables proposed here. We have found that with $5\times 10^3$ and
$1\times 10^4$ events respectively the observables
$\mathcal{A}(\epsilon_4)$ and $\mathcal{A}(\epsilon_{10})$ are still
useful for testing the hypotheses $(\kp=1,\kpt=\pm 1)$ giving rise to
separations of order $\sim 6\sigma$. These required number of events
would be reachable in principle in the short term by the LHC and hence
the capability of the observables studied here for testing the sign of
$\kpt/\kp$ could be probed in that context.

%%%%%%%%%%%%%%%%%%%%%%%%%%%%%%%%%%%%%%%%%%%%%%%%%
\bigskip
\noindent
{\bf Acknowledgments}
\noindent 
This work has been partially supported by ANPCyT under grants No. PICT
2013-0433 and No. PICT 2013-2266, and by CONICET (NM, AS). The work of
KK was supported by the U.S.  National Science Foundation under Grant
PHY-1215785. KK also acknowledges sabbatical support from Taylor
University.
%%%%%%%%%%%%%%%%%%%%%%%%%%%%%%%%%%%%%%%%%%%%%%%%%
\section*{\refname}
\let\bibsection\relax

\setlength{\bibsep}{10pt}
\bibliography{PaperDraftbiblio}
\bibliographystyle{apsrev4-1}

\end{document}
